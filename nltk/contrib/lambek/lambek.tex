%
% Ling 5554 Final Project: Lambek Calculus Theorem Prover
% Edward Loper
% Created [12/12/00 12:46 AM]
%
% $Id$
%

\documentclass[11pt]{article}
\usepackage{fullpage}
\usepackage{boxedminipage}

\begin{document}

\title{Lambek Calculus Theorem Prover}
\author{Edward Loper}
\date{\today}
\maketitle

\section{Introduction}

Applicative Categorial Grammar provides a compositional description of
the syntactic structure of natural language.  It assigns a denotation
to each expression in the language.  Each expression's
\emph{denotation} consists of a lambda calculus term and a category.
A denotation's \emph{term} specifies the ``meaning'' of the
expression.  A denotation's \emph{category} (or \emph{type}) describes
how it can be combined with other denotations.

The Lambek Calculus extends Categorial Grammar by extending the
meanings of categories, thereby providing new ways to combine
denotations.  In Categorial Grammar, a denotation $D$ can combine in
the manner specified by a category $C$ if $D$ has category $C$.  But
Lambek calculus adds the converse assertion: a denotation $D$ has a
category $C$ if $D$ can combine in the mannar specified by $C$.

In Lambek Calculus, theorems consist of statements that one
expression's denotation can be reduced to another expression's
denotation.  This paper describes a system which will find all
proofs\footnote{More precisely, all cut-free proofs.  But since a
theorem is provable if and only if it has a cut-free proof, the system
is still guaranteed to find a proof if and only if a proof exists.}
for any theorem in Lambek Calculus.

\subsection{Lambek Calculus}

This section gives a brief overview of the Lambek Calculus.  I used
Carpenter(1997)'s description of Lambek Calculus as the basis for my
theorem prover.  I tried to stay consistant with Carpenter's
terminology, where possible.  See Carpenter(1997) for more details on
the Lambek Calculus.

\subsubsection{Terms}
Lambek Calculus terms give the ``meanings'' of expressions, in terms
of the functional composition of a fixed finite set of basic terms,
\textbf{BasTerm}.  The internal structure of the terms in
\textbf{BasTerm} are left unanalyzed.

Terms are written using lower case Greek letters, such as $\alpha$ and
$\beta$.  Term constants are bold faced words, such as \textbf{eat} and
\textbf{cat}.  Terms can also be written out in lambda calculus, such
as $\lambda x.\textbf{eat}(x)(\alpha)$

\newpage
\subsubsection{Categories}
Lambek Calculus categories specify how a denotation can combine with
other denotations.  They consist of a fixed finite set of base
categories, \textbf{BasCat}, and the closure of those categories under
the operations $\backslash$, $/$, and $\cdot$.  These operations specify how
denotations can be combined (where linear order of the denotations
combined is signifigant):

\begin{itemize}
  \item A denotation with category $A$ can be combined with a
  denotation of category $A \backslash B$ to produce a denotation of category
  $B$.
  \item A denotation with category $B / A$ can be combined with a
  denotation of category $A$ to produce a denotation of category $B$.
  \item A denotation with category $A$ can be combined with a
  denotation of category $B$ to produce a denotation of category
  $A \cdot B$.
\end{itemize}

Categories are written using upper case italic letters, such as $A$
and $B$.

\subsection{Denotations}

A denotation is a pair consisting of a term and a category.  It
therefore specifies both the meaning of an expression, and how that
expression can combine with other expressions.  

Denotations are written as terms and categories joined with a colon,
such as $\alpha:A$ and $\lambda x.\alpha:B/A$.  The variable $\Delta$ can also stand
for a denotation.

\subsubsection{Expressions}

Expressions represent string of natural language words.  Expressions
consist of a fixed finite set of base expressions, \textbf{BasExp},
and their closure under concatination.  A lexicon, \textbf{Lex}, maps
each expression in \textbf{BasExp} to its denotation.

Expressions are written in Roman text, such as ``John likes Mary.''

\subsubsection{Sequents}

A \emph{sequent} is a theorem in Lambek Calculus which asserts that
one sequence of denotations can be reduced to another sequence of 
denotations. 

Sequents are written as a comma separated list of denotations,
followed by the symbol $\Rightarrow$, followed by another comma separated list
of denotations, such as:

$$\Gamma_1, \alpha:A, \lambda x.\textbf{like}(x):A\backslash B \Rightarrow \Gamma_1, \textbf{like}(\alpha): B$$

\noindent
The variables $\Gamma_1$, $\Gamma_2$, $\ldots$ are be used to represent
zero or more denotations.

\subsubsection{Proofs}

A proof states that one sequent (the \emph{conclusion}) is true iff
zero or more sequents (the \emph{assumptions}) are true.  Each step of
a proof is written as a line, with the assumptions above, the
conclusion below, and the name of he rule used to the right.  Each
step of a proof must match one of the Lambek calculus rule schemes.
An example proof step is:

$$
\frac{\textrm{$ 
  {{\textit{a}: \textrm{b} \Rightarrow \textit{a}: \textrm{b}}} \quad 
  {{\textit{b}(\textit{a}): \textrm{a} \Rightarrow \textit{b}(\textit{a}): \textrm{a}} } \quad $}}
      {\textit{b}: \textrm{a/b}, \textit{c}: \textrm{b} \Rightarrow \textit{b}(\textit{a}): \textrm{a}} \textrm{/L}
$$

\subsubsection{Rule Schemes}

The Lambek calculus uses the following rule schemes.  See
Carpenter(1997) for a detailed description of these schemes.


% /L
$$\frac{\Delta \Rightarrow \beta:B \quad \Gamma_1, \alpha(\beta):A, \Gamma_2 \Rightarrow \gamma:C}
{\Gamma_1, \alpha:A/B, \Delta, \Gamma_2 \Rightarrow \gamma:C} /L
\hspace{16mm}
% \L
\frac{\Delta \Rightarrow \beta:B \quad \Gamma_1, \alpha(\beta):A, \Gamma_2 \Rightarrow \gamma:C}
{\Gamma_1, \Delta, \alpha:B\backslash A, \Gamma_2 \Rightarrow \gamma:C} \backslash L$$

% /R
$$\frac{\Gamma, x:A \Rightarrow \alpha:B} {\Gamma \Rightarrow \lambda x.\alpha:B/A} /R
\hspace{35mm}
% \R
\frac{x:A, \Gamma \Rightarrow \alpha:B} {\Gamma \Rightarrow \lambda x.\alpha:A\backslash B} \backslash R$$

% *L
$$\frac{\Gamma_1, \pi_1(\alpha):A, \pi_2(\alpha):B, \Gamma_2 \Rightarrow \gamma:C}
{\Gamma_1, \alpha:A \cdot B, \Gamma_2 \Rightarrow \gamma:C} \cdot L
\hspace{16mm}
% *R
\frac{\Gamma_1, \Rightarrow \alpha:A \quad \Gamma_2 \Rightarrow \beta:B}
{\Gamma_1, \Gamma_2 \Rightarrow <\alpha, \beta>:A \cdot B} \cdot R\hspace{5mm}$$

\subsubsection{Summary of Terminology}
The following table sumarizes the terminology:

\textit{
\begin{tabbing}
\qquad\=Term\qquad\quad
             \== {\rm \bf BasTerm} $|$ $\lambda x.$Term $|$ Term(Term)\\
\>BasTerm    \>= {\rm \bf eat} $|$ {\rm \bf sleep} $|$ $\ldots$\\
\\
\>Cat        \>= {\rm \bf BasCat} $|$ (Cat $\backslash$ Cat) $|$ 
               (Cat $/$ Cat) $|$ (Cat $\cdot$ Cat)\\
\>BasCat     \>= {\rm np} $|$ {\rm n} $|$ $\ldots$\\
\\
\>Exp        \>= {\rm \bf BasExp} $|$ Exp Exp\\
\>BasExp     \>= {\rm eat} $|$ {\rm sleep} $|$ $\ldots$\\
\\
\>Denotation \>= Term: Cat\\
\>Sequent    \>= Denotation* $\Rightarrow$ Denotation*\\
\end{tabbing}
}

\section{Usage}
%
% What does it do?
%

I designed and implemented a theorem prover for Lambek calculus.  This 
program interactively asks for sequents, and attempts to prove them.
It will find all proofs for any derivable sequent, and will report
that there are no proofs for any non-derivable sequent.  In addition,
it attempts to unify free variables in the sequent's terms.  For
example, given the sequent:

$$\textbf{John}:np, \textbf{sings}:np\backslash s \Rightarrow \alpha: s$$

\noindent
The program will derive that:

$$\alpha = \textbf{sings}(\textbf{John})$$

Figure \ref{fig:session} shows an example session with the theorem
prover.  Note that the theorem prover can generate \LaTeX\, output, and
can operate in either normal mode, or short-circuit mode, where it
returns the first proof it finds for each sequent.

\begin{figure}
\noindent
\begin{boxedminipage}{\textwidth}
{\footnotesize \tt
[edloper@syse ling554]\$ \textit{./lambek.py lexicon.txt}
\vspace{3mm}

>> \textit{[np/n] [n] => [np]}

\begin{verbatim}
%%%%%%%%%%%%%%%%%%%%%%%%%%%%%%%%%%%%%%%%%%%%%%%%%%%%%%%%%%%%
%% Proof(s) for a: np/n, b: n => c: np

------------ I   -------------------- I
c: n => c: n     a(c): np => a(c): np
------------------------------------- /L
      a: np/n, b: n => a(c): np
\end{verbatim}

>> \textit{dog sleeps => [(np/n)$\backslash$s]}

\begin{verbatim}
%%%%%%%%%%%%%%%%%%%%%%%%%%%%%%%%%%%%%%%%%%%%%%%%%%%%%%%%%%%%
%% Proof(s) for dog: n, sleeps: np\s => a: (np/n)\s

                     ------------------------ I   -------------------------------------- I
                     a(dog): np => a(dog): np     sleeps(a(dog)): s => sleeps(a(dog)): s
---------------- I   ------------------------------------------------------------------- \L
dog: n => dog: n                a(dog): np, sleeps: np\s => sleeps(a(dog)): s
---------------------------------------------------------------------------------------- /L
                   a: np/n, dog: n, sleeps: np\s => sleeps(a(dog)): s
---------------------------------------------------------------------------------------- \R
                 dog: n, sleeps: np\s => (\x.sleeps(x(dog))): (np/n)\s
\end{verbatim}

>> \textit{[np/n] [n] => [s]}

\begin{verbatim}
%%%%%%%%%%%%%%%%%%%%%%%%%%%%%%%%%%%%%%%%%%%%%%%%%%%%%%%%%%%%
%% Can't prove a: np/n, b: n => c: s
\end{verbatim}

>> \textit{help}

\begin{verbatim}
% Lambek Calculus Theorem Proover
%
% Type a sequent you would like prooved.  Examples are:
%   [np/n] [n] => [np]
...
%   the city tom likes => [np*(s/np)]
%
% Other commands:
%   help         -- show this information
%   latexmode    -- toggle latexmode (outputs in LaTeX)
%   shortcircuit -- toggle shortcircuit mode (return just one proof)
%   lexicon      -- display the lexicon contents
%   quit         -- quit
\end{verbatim}
>> \textit{lex}
\begin{verbatim}
% Lexicon: 
%  gives:          gives: (np\s/np)/np
%  mary:           mary: np
%  john:           john: np
%  the:            the: np/n
...
\end{verbatim}}
\end{boxedminipage}
\caption{Example session with the Lambek calculus theorem prover}
\label{fig:session}
\end{figure}

\newpage
\subsection{Example Proofs}
\label{sec:examples}

This section lists a number of example proofs generated by the theorem 
prover.  These proofs demonstrate some of the system's capabilities.

\input{examples.tex}

\newpage
\section{Design}

This section gives a detailed description of the design of the Lambek
calculus theorem prover.  The theorem prover was written in Python.
It consists of five basic data classes, a lexicon, a lambda calculus
unifier, and a proof function.

\subsection{Data Classes}

The five basic data classes, and their subclasses, are:
\begin{itemize}
  \item \texttt{Term}: A lambda calculus expression.
  \begin{itemize}
    \item \texttt{Var}: A variable (e.g., $x$ or $\alpha$).
    \item \texttt{Const}: A constant (e.g., \textbf{run}).
    \item \texttt{Appl}: An application (e.g., $\alpha(\beta)$).
    \item \texttt{Abstr}: An abstraction (e.g., $\lambda x.x$).
    \item \texttt{Tuple}: A tuple (e.g., $\langle x,y\rangle$).
  \end{itemize}
  \item \texttt{Type}: A category.
  \begin{itemize}
    \item \texttt{LSlash}: A left-slash category (e.g., A$\backslash$B).
    \item \texttt{RSlash}: A right-slash category (e.g., A$/$B).
    \item \texttt{Dot}: A product category (e.g., A$\cdot$B).
    \item \texttt{BaseType}: A base type (e.g., np).
  \end{itemize}
  \item \texttt{TypedTerm}: A denotation (e.g., $\alpha$:A).
  \item \texttt{Sequent}: A sequent (e.g., $\alpha$:A $\Rightarrow$ $\alpha$:A).
  \item \texttt{Proof}: A Lambek calculus proof.
\end{itemize}

\newpage
\subsubsection{\texttt{Term}}

Lambda calculus expressions are represented by trees of objects of
type \texttt{Term}.  The constructors and accessors for each subtype
of
\texttt{Term} are:

\vspace{3mm}
\begin{tabular}{l|lll}
Term \hspace{1cm} & Constructor \hspace{1cm} 
  & Accessor$_1$ & Accessor$_2$\\
\hline
$\alpha(\beta)$ & \texttt{Appl}($\alpha$, $\beta$) 
  & \texttt{Appl.func} = $\alpha$ & \texttt{Appl.arg} = $\beta$ \\
$\lambda x.\alpha$ & \texttt{Abstr}($x$, $\alpha$) 
  & \texttt{Abstr.var} = $x$ & \texttt{Abstr.arg} = $\alpha$ \\
$\langle\alpha,\beta\rangle$ & \texttt{Tuple}($\alpha$, $\beta$)
  & \texttt{Tuple.left} = $\alpha$ & \texttt{Tuple.right} = $\beta$\\
\textit{name} & \texttt{Const}(\textit{name}) 
  & \texttt{Const.name} = \textit{name} & \\
$x$& \texttt{Var}() & & \\
\end{tabular}

\vspace{3mm}\noindent
The following functions are also defined for lambda expressions:
\begin{itemize}

  \item \texttt{freevars}($\alpha$): Return a list of all free variables
  in the given lambda expression.

  \item \texttt{boundvars}($\alpha$): Return a list of all bound variables
  in the given lambda expression.

  \item \texttt{vars}($\alpha$): Return a list of all variables used in
  the given lambda expression.  vars($\alpha$) = \texttt{freevars}($\alpha$) +
  \texttt{boundvars}($\alpha$)

  \item \texttt{replace}($x$, $\alpha$, $\beta$): Replace all free occurances
  of the $x$ in $\beta$ with $\alpha$.

  \item \texttt{reduce}($\alpha$): Return the reduced form of $\alpha$
  (i.e., the form in which $\beta-$ and $\eta-$ reduction can not be
  applied to $\alpha$ or any of its subterms).

  \item \texttt{unify}($\alpha$, $\beta$, \textit{varmap}): Attempt to bind
  the free variables of $\alpha$ and $\beta$ such that $\alpha=\beta$, and return
  the resulting expression.  \textit{varmap} is a dictionary providing
  a partial mapping variables to values.  The bindings in
  \textit{varmap} must be obeyed by the unification.  \textit{varmap}
  is modified by this procedure.  See Section \ref{sec:unify} for more
  details.

  \item \texttt{parse\_term}(\textit{s}): Return the lambda expression
  corresponding to the string \textit{s}.  An example string for a
  lambda expression is ``$\backslash$?x.likes(x)''.
\end{itemize}

\subsubsection{\texttt{Type}}

Categories are represented by trees of objects of type
\texttt{Type}.  The constructors and accessors for each subtype of
\texttt{Type} are:

\vspace{3mm}
\begin{tabular}{l|lll}
Term \hspace{1cm} & Constructor \hspace{1cm} 
  & Accessor$_1$ & Accessor$_2$\\
\hline
A $\backslash$ B & \texttt{LSlash}(A, B) 
  & \texttt{LSlash.arg} = A & \texttt{LSlash.result} = B\\
B / A & \texttt{RSlash}(B, A) 
  & \texttt{LSlash.arg} = A & \texttt{RSlash.result} = B\\
A $\cdot$ B & \texttt{Dot}(A, B)
  & \texttt{Tuple.left} = A & \texttt{Tuple.right} = B\\
\textit{type} & \texttt{BaseType}(\textit{type}) 
  & BaseType.name = \textit{type} & \\
\end{tabular}

\vspace{3mm}\noindent
The following function is also defined for lambda expressions:
\begin{itemize}
  \item \texttt{parse\_type}(\textit{s}): Return the category
  corresponding to the string \textit{s}.  An example string for a
  category is ``np$\backslash$(s/(np*np))''.  The order of operations are as
  defined in Carpenter(1997).
\end{itemize}

\subsubsection{\texttt{TypedTerm}}

Denotations are represented by objects of class \texttt{TypedTerm}.
The following functions and accessors are defined for
\texttt{TypedTerm}s:

\begin{itemize}
  \item \texttt{TypedTerm}($\alpha$, A): Construct a \texttt{TypedTerm}
  representing the denotation $a$:A.
  \item \texttt{TypedTerm.type}: A \texttt{TypedTerm}'s type.
  \item \texttt{TypedTerm.term}: A \texttt{TypedTerm}'s term.
  \item \texttt{($\alpha$:A).unify}($\beta$:B, \textit{varmap}): If A=B,
  return $\left(\texttt{unify}(\alpha, \beta, \textit{varmap})\right)$:A.
  Otherwise, return \texttt{None}.  See Section \ref{sec:unify} for
  more details about unification of lambda expressions.
\end{itemize}

\subsubsection{\texttt{Sequent}}

Sequents are represented by objects of cass \texttt{Sequent}.  The
following functions and accessors are defined for \texttt{Sequent}s:

\begin{itemize}
  \item \texttt{Sequent}(\textit{left}, \textit{right}: Construct a
  \texttt{Sequent} with the given left and right sides, where
  \textit{left} and \textit{right} are lists of \texttt{TypedTerm}s.
  \item \texttt{Sequent.left}: A list containing the denotations on
  the left of the sequent.
  \item \texttt{Sequent.right}: A list containing the denotations on
  the right of the sequent.
\end{itemize}

\subsubsection{\texttt{Proof}}

Proofs are represented by trees of objects of cass \texttt{Proof}.
The following functions and accessors are defined for
\texttt{Proof}s:

\begin{itemize}
  \item \texttt{Proof}(\textit{rule, assumptions, conclusion,
  varmap}): Construct a new sub-proof.  \textit{assumptions} is a list
  of \texttt{Proof}s, whose conclusions are the assumptions of the
  sub-proof.  \textit{conclusion} is a \texttt{Sequent} which gives
  the conclusion of the proof.  \textit{rule} is a string naming the
  rule used to produce this proof.  And \textit{varmap} is a
  dictionary mapping \texttt{Var}s to \texttt{Term}s which specifies
  values for (some of) the free variables of the terms in the
  proof.

  \item \texttt{Proof.rule}: A string naming the rule used to produce
  the proof.

  \item \texttt{Proof.assumptions}: A list of \texttt{Proof}s, whose
  conclusions are the assumptions of the sub-proof.

  \item \texttt{Proof.conclusion}: A \texttt{Sequent} giving the
  conclusion of the proof.

  \item \texttt{Proof.varmap}: A dictionary mapping \texttt{Var}s to
  \texttt{Term}s which specifies values for (some of) the free
  variables of the expressions in the proof.

  \item \texttt{Proof.simplify()}: Return a simplified version of this
  proof, where the proof's terms have had the values of their free
  variables filled in and have been reduced.

  \item \texttt{Proof.pp()}: Return a text representation of this proof
  tree (as seen in Figure \ref{fig:session}).

  \item \texttt{Proof.to\_latex()}: Return a latex representation of
  this proof tree (as seen in Section \ref{sec:examples}).
 
\end{itemize}

\subsection{\texttt{Lexicon}}

The \texttt{Lexicon} class maintains a mapping from words to
\texttt{TypedTerm}s.  It defines the following functions:

\begin{itemize}
  \item \texttt{Lexicon}(): Construct a new (empty) lexicon.

  \item \texttt{Lexicon.load}(\textit{file}): Add the lexicon entries
  in the given file to this lexicon.  The file should have one lexicon
  entry per line, and lexicon entries should have the format
  ``word:$\alpha$:A''.  Any text to the right of a '\#' character is
  considered a comment.

  \item \texttt{Lexicon}[\textit{word}]: Return the \texttt{TypedTerm}
  corresponding to the given word.  If none exists, return
  \texttt{None}.

  \item \texttt{Lexicon.parse}(\textit{str}): Split the given string
  on whitespace, and return a list containing
  \texttt{Lexicon}[\textit{word}] for each \textit{word} in the
  string.

  \item \texttt{Lexicon.words}(): Return a list containing all words
  with values defined in the lexicon.
\end{itemize}

\subsection{The Lambda Calculus Unifier}
\label{sec:unify}

The lambda calculus unifier attempts to find a set of values for the
free values of two terms that will make them equal.  In addition,
these values must be consistant with a given set of free-variable
values.

The unifier is implemented by the function
\texttt{unify}($\alpha$,$\beta$,\textit{varmap}).  This function attempts to
unify $\alpha$ and $\beta$ in a way consistant with the priors specified by
\textit{varmap}.  It adds any new free-variable bindings to
\textit{varmap}, and returns a version of the term containing as few free
variables as possible.  As a special case, variables may be mapped by
\textit{varmap} to \texttt{None}, indicating that they cannot be
unified with anything but themselves.  This is used to ensure that
bound variables do not get assigned values.  If \texttt{unify} cannot
unify $\alpha$ and $\beta$, it returns \texttt{None} and does not modify
\textit{varmap}.  Note that the unifier is not currently guaranteed to
find a unification of two terms if one exists.  The unifier uses a
case-by-case analysis to unify $\alpha$ and $\beta$, based on their classes.
Figure \ref{fig:unify} describes the different cases.

\begin{figure}
\noindent
\begin{boxedminipage}{\textwidth}
\begin{itemize}
\item[If] $\alpha \in $\texttt{Var}
\begin{itemize}
  \item If \textit{varmap}[$\alpha$] == \texttt{None}: fail
  \item If $\alpha \in$ \texttt{freevars}($\beta$): fail
  \item If $\alpha$ is in \textit{varmap},
      \texttt{unify}(\textit{varmap}[$\alpha$], $\beta$
  \item else: \textit{varmap}[$\alpha$] = $\beta$
\end{itemize}
\item[If] $\alpha \in $\texttt{Tuple}, $\beta \in $\texttt{Tuple}
\begin{itemize}
  \item $\langle$\texttt{unify}($\pi_1(\alpha)$, $\pi_1(beta)$),
         \texttt{unify}($\pi_2(\alpha)$, $\pi_2(beta)$)$\rangle$
\end{itemize}
\item[If] $\alpha \in $\texttt{Abstr}, $\beta \in $\texttt{Abstr}
\begin{itemize}
\item $\lambda x.\texttt{unify}(\alpha.\texttt{body}[x/\alpha.\texttt{var}]., 
      \beta.\texttt{body}[x/\alpha.\texttt{var}])$
    Where $x$ is fresh.
\end{itemize}
\item[If] $\alpha \in $\texttt{Appl}, $\beta \in $\texttt{Appl}
\begin{itemize}
  \item \texttt{unify}($\alpha.\texttt{func}$, $beta.\texttt{func}$)
        (\texttt{unify}($\alpha.\texttt{var}$, $beta.\texttt{var}$))
  \item If that fails, and $\alpha = x(y)$, where $y$ is free, then
        \textit{varmap}[$x$] = $\lambda z.\alpha[z/y]$
\end{itemize}
\item[If] $\alpha \in $\texttt{Const}, $\beta \in $\texttt{Const}
\begin{itemize}
\item if $\alpha = \beta$: return $\alpha$
\item else: fail
\end{itemize}
\end{itemize}
\vspace{2mm}
\end{boxedminipage}
\caption[Pseudo-code for \texttt{unify}]{Pseudo-code for
\texttt{unify}.  This figure shows the case-by-case analysis
used to unify two terms.  Note that this algorithm is not guaranteed
to find a unification if one exists.}
\label{fig:unify}
\end{figure}

\subsection{The Prover}

The prover attempts to find all proofs for a given sequent.  It works
by matching the given sequent against the conclusion of each rule
scheme.  If a rule scheme matches the conclusion of a rule scheme, the
prover recursively attempts to find proofs for each of the rule
scheme's assumptions.  This process is guaranteed to terminate because 
each sequent has only a finite number of cut-free proofs (Carpenter,
1997).  

In addition to a sequent, the prover is given a set of free-variable
values.  These values are used to unify multiple occurances of the
same term in the rule scheme (for example, in the /L rule, the term
$\alpha$ appears both in the conclusion and in one of the assumptions).
If the terms cannot be unified, then the proof fails.  If the proof
succeeds, then the set of free-variable values used is recorded in its 
\texttt{varmap} field.

The algorithms for matching sequents against the rule schemes are
given in Figures \ref{fig:rslash_l}, \ref{fig:lslash_l},
\ref{fig:rslash_r}, \ref{fig:lslash_r}, \ref{fig:dot_l}, and
\ref{fig:dot_r}.  

\begin{figure}
\noindent
\begin{boxedminipage}{\textwidth}
\begin{itemize}
  \item For each $\Gamma \Rightarrow \gamma:\textrm{B/A}$ that matches the given $seq$:
  \begin{itemize}
    \item For each proof $p$ that 
          $\Gamma, x:\textrm{A} \Rightarrow \alpha_{assum}(x):\textrm{B}$ ($x$ fresh)
    \begin{itemize}
      \item unify($\lambda x.\alpha, \lambda x.\alpha_{assum}$)
      \item return Proof(``/R'', $[p]$, $seq$, \textit{varmap})
    \end{itemize}
  \end{itemize}
\end{itemize}
\vspace{2mm}
\end{boxedminipage}
\caption{Pesudo-code for matching the /R rule.}
\label{fig:rslash_r}
\end{figure}

\begin{figure}
\noindent
\begin{boxedminipage}{\textwidth}
\begin{itemize}
  \item For each $\Gamma \Rightarrow \gamma:\textrm{B$\backslash$A}$ that matches the given $seq$:
  \begin{itemize}
    \item For each proof $p$ that 
          $x:\textrm{A},\Gamma \Rightarrow \alpha_{assum}(x):\textrm{B}$ ($x$ fresh)
    \begin{itemize}
      \item unify($\lambda x.\alpha, \lambda x.\alpha_{assum}$)
      \item return Proof(``$\backslash$R'', $[p]$, $seq$, \textit{varmap})
    \end{itemize}
  \end{itemize}
\end{itemize}
\vspace{2mm}
\end{boxedminipage}
\caption{Pesudo-code for matching the $\backslash$R rule.}
\label{fig:lslash_r}
\end{figure}

\begin{figure}
\noindent
\begin{boxedminipage}{\textwidth}
\begin{itemize}
  \item For each $\Gamma_1, \alpha:\textrm{A/B}, \Delta, \Gamma_2 \Rightarrow \gamma:C$ that matches the
  given $seq$:
  \begin{itemize}
    \item For each proof $p_{left}$ that $\Delta \Rightarrow \beta_{left}:$B
    \begin{itemize}
      \item For each proof $p_{right}$ that 
            $\Gamma_1, \alpha(\beta_{left}):\textrm{A}, \Gamma_2 \Rightarrow \gamma_{right}:C$
      \begin{itemize}
        \item unify($\gamma$, $\gamma_{right}$)
        \item return Proof(``/L'', $\left[p_{left}, p_{right}\right]$, 
              $seq$, varmap)
      \end{itemize}
    \end{itemize}
  \end{itemize}
\end{itemize}
\vspace{2mm}
\end{boxedminipage}
\caption{Pesudo-code for matching the /L rule.}
\label{fig:rslash_l}
\end{figure}

\begin{figure}
\noindent
\begin{boxedminipage}{\textwidth}
\begin{itemize}
  \item For each $\Gamma_1, \Delta, \alpha:\textrm{A$\backslash$B}, \Gamma_2 \Rightarrow \gamma:C$ that matches the
  given $seq$:
  \begin{itemize}
    \item For each proof $p_{left}$ that $\Delta \Rightarrow \beta_{left}:$B
    \begin{itemize}
      \item For each proof $p_{right}$ that 
            $\Gamma_1, \alpha(\beta_{left}):\textrm{A}, \Gamma_2 \Rightarrow \gamma_{right}:C$
      \begin{itemize}
        \item unify($\gamma$, $\gamma_{right}$)
        \item return Proof(``$\backslash$L'', 
              $\left[p_{left}, p_{right}\right]$, 
              $seq$, \textit{varmap})
      \end{itemize}
    \end{itemize}
  \end{itemize}
\end{itemize}
\vspace{2mm}
\end{boxedminipage}
\caption{Pesudo-code for matching the $\backslash$L rule.}
\label{fig:lslash_l}
\end{figure}

\begin{figure}
\noindent
\begin{boxedminipage}{\textwidth}
\begin{itemize}
  \item For each $\Gamma_1, \alpha:\textrm{A$\cdot$B}, \Gamma_2 \Rightarrow \gamma:C$ 
        that matches the given $seq$:
  \begin{itemize}
    \item For each proof $p$ that 
          $\Gamma_1, \alpha_{left}:\textrm{A}, \alpha_{right}:\textrm{B}, \Gamma_2 \Rightarrow \gamma_{assum}:C$
    \begin{itemize}
      \item unify($\alpha$, $\langle\alpha_{left}, \alpha_{right})\rangle$)
      \item unify($\gamma, \gamma_{assum}$)
      \item return Proof(``$\cdot$L'', $[p]$, $seq$, \textit{varmap})
    \end{itemize}
  \end{itemize}
\end{itemize}
\vspace{2mm}
\end{boxedminipage}
\caption{Pesudo-code for matching the $\cdot$L rule.}
\label{fig:dot_l}
\end{figure}

\begin{figure}
\noindent
\begin{boxedminipage}{\textwidth}
\begin{itemize}
  \item For each $\Gamma_1, \Gamma_2 \Rightarrow \langle\alpha, \beta\rangle: \textrm{A$\cdot$B}$ 
  that matches the given $seq$:
  \begin{itemize}
    \item For each proof $p_{left}$ that $\Gamma_1 \Rightarrow \alpha_{assum}:$A
    \begin{itemize}
      \item For each proof $p_{right}$ that $\Gamma_1 \Rightarrow \beta_{assum}:$B
      \begin{itemize}
        \item unify($\langle\alpha, \beta\rangle$, $\langle\alpha_{assum}, \beta_{assum}\rangle$)
        \item return Proof(``$\cdot$R'',
              $\left[p_{left}, p_{right}\right]$, 
              $seq$, \textit{varmap})
      \end{itemize}
    \end{itemize}
  \end{itemize}
\end{itemize}
\vspace{2mm}
\end{boxedminipage}
\caption{Pesudo-code for matching the $\cdot$R rule.}
\label{fig:dot_r}
\end{figure}


\end{document}