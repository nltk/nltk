\documentclass{460}
\assignment{1}{Tokenization}{11am Monday 18 August 2003}

\begin{document}
\maketitle

Please read the tutorial on NLTK basics.
Using the Python interpreter in interactive mode, experiment
with words, texts, tokens, locations and tokenizers, and satisfy
yourself that you understand all the examples in the tutorial.
Now complete the following questions.
You can use the lab sessions to work on this assignment

\section*{Lab exercises -- not for submission}

\begin{enumerate}
\item
Describe the class of strings matched by the following regular expressions:

\begin{enumerate}
    \item \verb@[a-zA-Z]+@
    \item \verb@[A-Z][a-z]*@
    \item \verb@\d+(\.\d+)?@
    \item \verb@([bcdfghjklmnpqrstvwxyz][aeiou][bcdfghjklmnpqrstvwxyz])*@
    \item \verb@\w+|[^\w\s]+@
\end{enumerate}

\item
Write regular expressions to match the following classes of strings:

\begin{enumerate}
    \item A single determiner (assume that ``a,'' ``an,'' and ``the''
    are the only determiners).  
    \item An arithmetic expression using integers, addition, and
    multiplication, such as $2*3+8$.
\end{enumerate}

\item
Write Python code which evaluates and prints:

\begin{enumerate}
    \item a location beginning at position 5 and ending at position 6,
      having the unit \texttt{w} (word)
    \item a token ``parrot'' at this location
    \item the type of this token
    \item the location of this token
\end{enumerate}
\end{enumerate}

\pagebreak

\section{Word Level Tokenization}

Choose a text (e.g. from Project Gutenberg)
and save it to a local file called \texttt{corpus.txt}.  (There is no need
to submit this data file).

\begin{enumerate}
    \item Define a function \texttt{p1a()} which reads the file into a
      string, tokenizes the string using \texttt{WSTokenizer},
      and prints the tokens, one per line.
    \item Using \texttt{RETokenizer}, create a tokenizer \texttt{p1b()}
      which does a better
      job at handling punctuation than the one given in the tutorial, and
      explain your improvements in the inline documentation.
\end{enumerate}

\section{Zipf's Law}

Let $f(w)$ be the frequency of a word $w$ in free text.  Suppose that
all the words of a text are ranked according to their frequency, with
the most frequent word first.
Zipf's law states that the frequency of a word type is inversely
proportional to its rank (i.e. $f$*$r$=$k$, for some constant $k$).
For example, the 50th most common word type should
occur three times as frequently as the 150th most common word type.

Write a Python function \texttt{p2()}
to process a large text and plot word frequency
against word rank using the \texttt{nltk.draw.plot} module.
Do you confirm Zipf's law?  (Hint: it helps to
set the axes to log-log.)

\section*{Important Notes}

{\bf Accessing NLTK.}
Ensure you have your \texttt{PYTHONPATH} environment variable
set to \url{/home/subjects/460/local/nltk/}.

{\bf Marking.}  Points will be awarded for well-structured and
well-documented code.  Points will be deducted for late submissions
(20\% per day or part thereof).  Assignments must be strictly original
work.

{\bf Submission.}
Submission will be in the form of a single Python program,
submitted using the department's submit system
(instructions to follow).

\end{document}

