% Natural Language Toolkit Problem Set
% Problem Set 2: Using NLTK
%
% Copyright (C) 2001 University of Pennsylvania
% Author: Edward Loper <edloper@gradient.cis.upenn.edu>
% URL: <http://nltk.sf.net>
% For license information, see LICENSE.TXT
%
% $Id$

\documentclass{article}
\usepackage{fullpage}

\begin{document}
\title{Problem Set 2: Introduction to NLTK}
\subtitle{Due: sometime}
\maketitle

% Add comment here about how code should be pretty

\section{Regular Expressions}

Describe the class of strings matched by the following regular expressions:
\begin{enumerate}
    \item \texttt{[a-zA-Z]+}
    \item \texttt{[A-Z][a-z]*}
    \item \texttt{\\d+(\\.\\d+)?}
    \item \texttt{([bcdfghjklmnpqrstvwxyz][aeiou][bcdfghjklmnpqrstvwxyz])*}
    \item \texttt{\\w+|[\^\\w\\s]+}
\end{enumerate}

Write a regular expression to match the following classes of strings:

\begin{enumerate}
    \item A single determiner (assume that ``a,'' ``an,'' and ``the''
    are the only determiners).  
    \item An arithmatic expression using integers, addition, and
    multiplication, such as $2*3+8$.
\end{enumerate}

\section{Tokens and Locations}

\begin{enumerate}
    \item Write an expression that constructs 
\end{enumerate}

\section{Zipf's Law}

Zipf's Law relates a word type's frequency to its rank (the
\emph{rank} of the $r$ most common word type is $r$).  In particular,
it states that the frequency $f$ of a word type is inversely
proportional to its rank $r$.  In other words, $f$*$r$=$k$, for some
constant $k$.  For example, the 50th most common word type should
occur three times as frequently as the 150th most common word type.
See \emph{Foundations of Statistical Natural Language Processing}
(Manning & Schutze), pp. 23-24, for more information on Zipf's Law.

Write a Python program to demonstrate Zipf's law.

\section{Sentence Level Tokenization}

\section{The Switchboard Corpus}

Listen to one of the recorded telephone conversations on the
Switchboard CD-ROM (also on
\texttt{unagi:/spd25/cis530/cdrom/switchboard/speech-pc/}).  Identify an
utterance which is not grammatical, standard English.  Transcribe the
utterance and describe the way(s) in which it is ungrammatical or
non-standard.

\end{document}


  1. some regexp problems (what does xyz match?  write a regexp to match xyz?)
  2. basic nltk usage (construct locations/tokens.  what happens when you
     compare locations?  overlapping locations?  locations with different
     sources?  etc.)
  3. zipf problem: demonstrate zipf by plotting freq vs. rank.  Print out
     a log/log version of the output (still not sure how I want them to
     turn it in.. I'll figure that out soon)
  4. sentence tokenization: use RETokenizer to do good sentence tokenization
  5. your switchboard question
