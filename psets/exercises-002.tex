%
% Natural Language Toolkit for Python
% Peudo-Problem Set 2
% Edward Loper
% 
% Created [03/02/01 12:02 AM]
% $Id$
%

\documentclass[11pt]{article}
\usepackage{fullpage}
\usepackage{boxedminipage}

\begin{document}
\title{Psuedo-Problem Set 2}
\author{Edward Loper}
\maketitle

\section{Introduction to the NLP Toolkit}

  The NLP Toolkit is a Python package which simplifies the creation of
  natural language processing programs:

  \begin{enumerate}

    \item It defines data types for representing objects used by
    natural language processing tasks, such as words, syntax trees,
    and probability distributions.

    \item It defines standard interfaces for performing natural
    language processing tasks.  These standards ensure that all of the
    components you write can be combined to perform larger tasks.

    \item It defines several implementations for each task interface.
    Since tasks have standard interfaces, these implementations can be
    combined with your own components.

  \end{enumerate}

  An introduction to the toolkit can be read at:

  \texttt{http://www.cis.upenn.edu/\~edloper/nltk/tutorial}.

\subsection{Using the NLP Toolkit}

  The toolkit consists of several Python \emph{modules}.  Each module
  contains a collection of related data types and interfaces.  The
  modules defined by the NLP toolkit are:

  \begin{itemize}
    \item \texttt{nltk.token}: Data types and interfaces for processing
    individual elements of text, such as words or sentences.

    \item \texttt{nltk.tree}: Data types and interfaces for processing
    syntax trees.

    \item \texttt{nltk.probability}: Data types for representing
    frequency distributions and probability distributions.

    \item \texttt{nltk.wordnet}: Data types for accessing the WordNet
    lexical database.

    \item[etc.] (fill in more later)
  \end{itemize}

\subsubsection{Importing}

  Before you can use a module, you must import it.  The simplest way
  to import a module is to use the syntax:

  \begin{quote}
  \texttt{>>> from} \textit{nltk.token} \texttt{import *}
  \end{quote}

  \noindent If you import a module using this syntax, then you can
  access the objects it defines with their simple names (such as
  \texttt{Token}).

  You may also import a module using the syntax:

  \begin{quote}
  \texttt{>>> import} \textit{nltk.token}
  \end{quote}

  \noindent If you import a module with this syntax, then you must
  access its objects with fully qualified names (such as
  \texttt{nltk.token.Token}).

\subsubsection{Reference Documenation}

  To find out more about the contents of a module, refer to the
  reference documentation:

  \texttt{http://www.cis.upenn.edu/~edloper/nltk/ref}

  \noindent or use the pydoc help facility:

  \begin{quote}
  \texttt{>>> from pydoc import help}\\
  \texttt{>>> import} \textit{nltk.token}\\
  \texttt{>>> help(}\textit{nltk.token}\texttt{)}
  \end{quote}

\section{The Token Module}

  The token module defines several classes which are useful for
  processing individual elements of text, known as \emph{text types},
  or \emph{types} for short.  Words are the most common kind of text
  type, but types can also be used to represent sentences, morphemes,
  syntax trees, paragraphs, or any other element of text.
  \emph{Types} are often encoded using Python strings:

  \begin{quote}
  \texttt{>>> wordtypes = ["the", "big", "dog"]}
          \qquad \textit{\# Define a list of word types.}
  \end{quote}

  Occurances of types are known as \emph{text tokens}, or
  \emph{tokens} for short.  Several tokens may have the same type.
  For example, the following sentence contains seven word tokens, but
  only five word types:

  \begin{quote}
  The big dog chased the small dog.
  \end{quote}


\end{document}


