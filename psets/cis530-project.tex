\documentclass{cis530}
\usepackage{url}
\other{Mini-Project}
\due{1:30pm Wednesday 28 November 2001}

\begin{document}
\maketitle

Many topics and techniques have been discussed during this course.
Equally, many important topics and techniques have been omitted.
The mini-project is your chance to explore an area of your own interest.
You might like to gain a deeper understanding of one of the algorithms we
presented by implementing and testing it.  On the other hand, you may
like to explore a topic that was not covered in class.  In all cases,
we expect the result to be a working prototype, implemented in Python.

\section{Topic Ideas}

Here is an assortment of techniques that you might like to implement:
morphological analyzer; stemmer; Brill tagger; statistical chunker;
parsing with feature-based grammars; collocation
discovery using one of the methods in Manning \S5; word-sense
disambiguation using one of the methods in Manning \S7;
a text-retrieval task; PCFG parsing; ...
For further ideas, see the course readings and textbooks, or
the {\it HLT Survey} by Cole et al \url{http://cslu.cse.ogi.edu/HLTsurvey/}.

\section{The Design}

Before doing any programming, you should agree on the design with one
of the instructors.
Please write a short description (approx 200 words)
of the topic and your proposed approach, along with
a sketch of the top-level interface.  Be sure to make
it clear that the task is an appropriate size.
Please email this
to both instructors.  We will respond by email and/or
discuss details with you in office hours.

\section{The Form of the Result}

The resulting prototype will be a Python module consisting of one or
more classes and demonstration code.  The module should contain
detailed documentation at the module level, the class level, and the
member level.  You should be identified as the author in a comment
at the top of the module.

The demonstration code should run if the module is executed, using the
standard idiom: \verb@if __name__ == '__main__':@.

The Natural Language Toolkit contains many modules which you can use
as a model.

\section{Class Presentation}

In the class on Monday 3 December,
each student will be asked to give a brief presentation
of their work.

\section{Contributing the Code to NLTK}

We hope to contribute your modules to the Natural Language Toolkit,
either as examples of student projects, or as core modules.  This
means that students in later courses at Penn and elsewhere will benefit
from your work.

In some cases we may need to modify the code slightly.  You will
remain identified as author (unless you want to be anonymous), and
your code will be included under our open source license.

If you do not want your work to be made available in this way,
please include a comment to the effect at the top of your submission.

\section*{Notes}

{\bf Grading.}
Points will be awarded for well-structured and well-documented code,
and for original and insightful discussion.
Points will be deducted for late submissions.  Projects must be
strictly original work.  The value of this mini-project is approximately
the same as two assignments.

{\bf Submission.}  
Please email your completed assignment as a
\emph{single Python program} to both instructors
\texttt{sb@cis, edloper@gradient.cis}, as a MIME attachment
(mime type: \texttt{text/plain}).
\vspace{2ex}

\end{document}
