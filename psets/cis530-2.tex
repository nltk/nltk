\documentclass{cis530}
\num{2}\due{1:30pm Monday 24 September 2001}

\begin{document}
\maketitle

Please read the tutorial on NLTK basics.
Using the Python interpreter in interactive mode, experiment
with words, texts, tokens, locations and tokenizers, and satisfy
yourself that you understand all the examples in the tutorial.
Now complete the following questions.
You may use the second practicum to work on this assignment
(Moore 100B, Friday 21st, 2-5pm, entry code 8955*).

\emph{Submission will be in the form of a single Python program.}

\section{Regular Expressions}

Describe the class of strings matched by the following regular expressions:
\begin{enumerate}
    \item \texttt{[a-zA-Z]+}
    \item \texttt{[A-Z][a-z]*}
    \item \texttt{\\d+(\\.\\d+)?}
    \item \texttt{([bcdfghjklmnpqrstvwxyz][aeiou][bcdfghjklmnpqrstvwxyz])*}
    \item \texttt{\\w+|[\^\\w\\s]+}
\end{enumerate}

Write regular expressions to match the following classes of strings:

\begin{enumerate}
    \item A single determiner (assume that ``a,'' ``an,'' and ``the''
    are the only determiners).  
    \item An arithmetic expression using integers, addition, and
    multiplication, such as $2*3+8$.
\end{enumerate}

Save your answers as a formatted text string (using the triple quote
syntax), to be returned by a Python function \texttt{p1()}.

\section{Tokens and Locations}

Write a Python function \texttt{p2()} which evaluates and prints:

\begin{enumerate}
    \item a location beginning at position 5 and ending at position 6,
      having the unit \texttt{w} (word)
    \item a token ``parrot'' at this location
    \item the type of this token
    \item the location of this token
\end{enumerate}

\section{Word Level Tokenization}

\textit{Note, if you are working from your own copy of NLTK, it will be
  necessary to download the current version; see the course website for
  details.}

Choose a text from the resources CD-ROM or Project Gutenberg, or some other
source, and save it to a file called \texttt{corpus.txt}.  (There is no need
to submit this data file).

\begin{enumerate}
    \item Define a function \texttt{p3a()} which reads the file into a
      string, tokenizes the string using \texttt{WSTokenizer},
      and prints the tokens, one per line.
    \item Using \texttt{RETokenizer}, create a tokenizer \texttt{p3b()}
      which does a better
      job at handling punctuation than the one given in the tutorial, and
      explain your improvements in the inline documentation.
\end{enumerate}

\section{Zipf's Law}

Let $f(w)$ be the frequency of a word $w$ in free text.  Suppose that
all the words of a text are ranked according to their frequency, with
the most frequent word first.
Zipf's law states that the frequency of a word type is inversely
proportional to its rank (i.e. $f$*$r$=$k$, for some constant $k$).
For example, the 50th most common word type should
occur three times as frequently as the 150th most common word type.
(See \emph{Foundations of Statistical Natural Language Processing}
(Manning \& Schutze), pp. 23-24, for more information on Zipf's Law.)

Write a Python function \texttt{p4()}
to process a large text and plot word frequency
against word rank using the \texttt{nltk.draw.plot\_graph} module.
Do you confirm Zipf's law?  (Hint: it helps to
set the axes to log-log.)

\section{The Switchboard Corpus}

Listen to one of the recorded telephone conversations on the
Switchboard CD-ROM (also on
\texttt{unagi:/spd25/cis530/cdrom/switchboard/speech-pc/}).  Identify an
utterance which is not grammatical, standard English.  Transcribe the
utterance and describe the way(s) in which it is ungrammatical or
non-standard.  Put your response in a plain text string, to be returned
by function \texttt{p5()}.

Points will be awarded for well-structured and well-documented code.
Points will be deducted for late submissions.
Assignments must be strictly original work.

Please email your completed assignment to the instructors
\texttt{sb@cis, edloper@gradient.cis}, in the body of the message.
Please use plain ASCII (no zip files or attachments) and ensure
that no lines are longer than 80 characters.

\end{document}

