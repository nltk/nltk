\documentclass{cis530}
\usepackage{url}
\num{4}\due{1:30pm Wednesday 10 October 2001}

\begin{document}
\maketitle

Please read the \emph{Parsing} tutorial.

{\bf Accessing NLTK.}  
To access the centrally installed version of NLTK on unagi you will
need to have\\ \texttt{/spd25/cis530/nltk/src} in your PYTHONPATH (see
Lutz p. 20).  If you are working from your own copy of NLTK, it will
be necessary to download the latest version (0.3); please see the
course website for details.

{\bf Grading.}  
Points will be awarded for well-structured and well-documented code,
and for original and insightful discussion.
Points will be deducted for late submissions.  Assignments must be
strictly original work.

{\bf Submission.}  
Please email your completed assignment as a
\emph{single Python program} to both instructors
\texttt{sb@cis, edloper@gradient.cis}, in the body of the message.
\emph{Please use plain ASCII (no zip files or attachments) and ensure
that no lines are longer than 80 characters.}
\vspace{2ex}

\section{Shift-Reduce Parser}

The tutorial describes a shift-reduce parser, and a template is
provided in the \texttt{SRParser} class in the module \verb|srparser_template|.
Complete the class, and write a function \texttt{p1()} to demonstrate it.

\section{Left-Corner Parser}

A left-corner parser is a top-down parser with bottom-up filtering
(see the tutorial and the reading from Jurafsky \S 10).
Develop a new parser class, inheriting from \texttt{ParserI}, which
implements a left-corner parser, and write a function \texttt{p2()} to
demonstrate it.

\section{Chart Parser}

Familiarize yourself with the \texttt{chartparser} module, and the way
in which chartparsers are initialized and used in the demonstration code.
Use the graphical interface in the \texttt{draw/chart} module to experiment
with different rule invocation strategies.  Come up with your own strategy
which you can execute manually using the graphical interface.  Describe the
steps, and report any efficiency improvements it has (e.g. in terms of the
size of the resulting chart).  Do these improvements depend on the structure
of the grammar?  What do you think of the prospects for significant performance
boosts from cleverer rule invocation strategies?  Please put your answer in a formatted
text string returned by function \texttt{p3()}.

\pagebreak

\section{Grammar Development}

Pick some of the syntactic constructions described in Finegan \S 5 or Manning \S 3
and create a test set consisting of 40-50 sentences.  At least 10 of the sentences
should be ungrammatical.  Develop a grammar which accounts for these sentences,
providing trees for the grammatical ones and rejecting the ungrammatical ones,
refining your test set as needed to test the grammar.  Write
a function \texttt{p4a()} to demonstrate your grammar on three sentences:
(i) a sentence having exactly one parse;
(ii) a sentence having more than one parse;
(iii) a sentence having no parses.

Create a list \texttt{Words} of all the words in your lexicon, and use
\texttt{random.choice(Words)} to generate strings consisting of a random
number of random words.
(Hint: include a special word \texttt{END} in your list which signals
the end of a "sentence".  Include this word more than once if necessary
to bring the average sentence length down to 6-12 words.)
Report any grammatical
sentences which your grammar rejects, and any ungrammatical sentences which
your grammar accepts.  Discuss your observations and put your answer in
a formatted text string returned by function \texttt{p4b()}.

\section{Parser Comparison}

Compare the performance of these three parsers using the grammar and
three sentences from p4a.  Use \texttt{time.time()} to log the amount
of time each parser takes on the same sentence.  (See Lutz p.~240 for
instructions on how to do this.)  Write a function \texttt{p5a()}
which runs all three parsers on all three sentences, and prints
a 3x3 grid of times, as well as row and column totals.
Discuss your findings and put your answer in
a formatted text string returned by function \texttt{p5b()}.

\end{document}