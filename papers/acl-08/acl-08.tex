\documentclass[11pt]{article}
\usepackage{acl08}
\usepackage{times}
\usepackage{latexsym}
\usepackage{epsfig,url}
\setlength\titlebox{6.5cm}    % Expanding the titlebox

\title{Multidisciplinary Instruction with the Natural Language Toolkit}
\author{}

% \author{Joakim Nivre \\
%   School of Mathematics and Systems Engineering \\
%   V\"{a}xj\"{o} University \\
%   SE-35195, V\"{a}xj\"{o}, Sweden \\
%   {\tt nivre@msi.vxu.se} \And
%   Noah A. Smith \\
%   Language Technologies Institute \\
%   Carnegie Mellon University \\
%   Pittsburgh, PA 15213, USA\\
%   {\tt nasmith@cs.cmu.edu}}

\date{}

\begin{document}
\maketitle
\begin{abstract}
` The Natural Language Toolkit (NLTK) is widely used for teaching
  natural language processing to students majoring in linguists or
  computer science.  This paper describes the design of NLTK, and
  reports on how it has been used effectively in classes that involve
  a combination of linguists and computer scientists.
\end{abstract}

\section{Introduction}

Natural Language Processing (NLP) is often taught within the confines
of a single-semester course at advanced undergraduate level or
postgraduate level. Many instructors have found that it is difficult
to cover both the theoretical and practical sides of the subject in
such a short span of time, especially when some of the students have
limited programming experience. Some courses focus on theory to the
exclusion of practical exercises, and deprive students of the
challenge and excitement of writing programs to automatically process
language. Other courses are simply designed to teach programming for
linguists, and do not manage to cover any significant NLP content. The
Natural Language Toolkit (NLTK) was originally developed to address
this problem, making it feasible to cover a substantial amount of
theory and practice within a single-semester course, for an audience
consisting of both linguists and computer scientists.

The Natural Language Toolkit (NLTK) is a suite of Python modules
distributed under the GPL open source license from \url{nltk.org}.
NLTK comes with a large collection of corpora and a free online book
containing hundreds of exercises.  It has been used in over 50
university courses in over 15 countries.\footnote{\url{http://nltk.org/courses.html}}
We believe NLTK is unique in providing a comprehensive framework
for students to learn about NLP in the context of learning to
program. What sets it apart is the tight coupling of the
chapters and exercises with NLTK, giving students -- even those with no
prior programming experience -- a practical introduction to NLP. Once
completing these materials, students are ready to attempt one of
the more advanced textbooks.

Overview of paper: ...

\section{Design Decisions Affecting Teaching}

\subsection{Python}

We chose Python because it has a shallow learning curve, its syntax
and semantics are transparent, and it has good string-handling
functionality.  As an interpreted language, Python facilitates
interactive exploration.  As an object-oriented language, Python
permits data and methods to be encapsulated and re-used easily.  Python comes with an extensive
standard library, including tools for graphical programming and
numerical processing.

help function, access to NLTK documentation on any module.

Python itself fosters an
interactive style of teaching.  For instance, we've found it quite
natural to build up moderately complex programs in front of a class,
with the weaker students transcribing it into a Python session on
their laptop to satisfy themselves it works (but not necessarily
understanding everything they enter first time), while the stronger
students quickly grasp the theoretical concepts and algorithms.  While
both groups can be served by the same presentation, they tend to ask
quite different questions.  However, this is addressed by dividing
them into smaller clusters and having TAs visit them separately to
discuss issues arising from the content.

\subsection{Coding Requirements}

Consistency, extensibility, simplicity, modularity.
Non-requirements: comprehensiveness, efficiency, cleverness.
\cite{BirdLoper02}

Code is readable -- a student who doesn't understand the maths of HMMs,
smoothing, etc can get another angle on it via the code.

\subsection{Corpus Access}

Uniform corpus access.  After importing NLTK, one can access all the corpora
using the \texttt{nltk.corpus} module:

{\small\begin{verbatim}
>>> nltk.corpus.NAME.METHOD(PARAMETERS)
\end{verbatim}}

Here, \texttt{NAME} is any of the 45 corpora distributed with NLTK, including
parsed, POS-tagged, plain text, categorized text, and lexicons.\footnote{\url{http://nltk.org/corpora.html}}
The \texttt{METHOD} can be any of
\texttt{raw}, for the raw contents of the corpus;
\texttt{words}, for a list of tokenized words;
\texttt{sents}, for the same list grouped into sentences;
\texttt{tagged_words}, for a list of (word,tag) pairs;
\texttt{tagged_sents}, for the same list grouped into sentences;
\texttt{parsed_sents}, for a list of parse trees.
The following example shows how to access the Brown Corpus:

{\small\begin{verbatim}
>>> nltk.corpus.brown.tagged_words()
[('The', 'at'), ('Fulton', 'np-tl'),
('County', 'nn-tl'), ('Grand', 'jj-tl'),
('Jury', 'nn-tl'), ('said', 'vbd'), ...]
\end{verbatim}}

Note that not all methods are available for all corpora (e.g. we can
ask for the words from just about any corpus, but only for
parsed_sents from the parsed corpora).

The \texttt{PARAMETERS} are typically used to restrict the amount of material returned,
e.g. to a section of a corpus, or an individual corpus file.

\subsection{Diversity of Programming Experience}

Self-paced learning: tutorials, hundreds of graded exercises (self-evaluation)

NLTK supports assignments of varying difficulty and scope. In the
simplest assignments, students experiment with existing components to
perform a wide variety of NLP tasks. This may involve no programming
at all, in the case of the existing demonstrations, or simply changing
a line or two of program code. As students become more familiar with
the toolkit they can be asked to modify existing components or to
create complete systems out of existing components. NLTK also provides
students with a flexible framework for advanced projects, such as
developing a multi-component system, by integrating and extending NLTK
components, and adding on entirely new components. Here NLTK helps by
providing standard implementations of all the basic data structures
and algorithms, interfaces to standard corpora, substantial corpus
samples, and a flexible and extensible architecture.

\section{Getting Started}

\subsection{The First Lecture}

Motivating and exemplifying NLP to a mixed audience

the holy grail: machines that understand language:
- relates to linguists as 
- relates to CS students: technologies that demonstrate some level of NLU:
  SLDS, QA, Summarization, MT

drivers: large data
- web
- corpora
- no end to corpus formats, incompatible tools (linguists)
  exploratory data analysis (find patterns not supported by existing software)



- algorithms


\subsection{CDROM}

ISO image -- give out CDs




\section{Classroom Interaction}

\subsection{Demonstrations with the Python Interpreter}

try it and see

NLTK book has many examples...

An effective way to deliver the materials is through interactive
presentation of the examples, entering them at the Python prompt,
observing what they do, and modifying them to explore some empirical
or theoretical question.


\subsection{Interactive Demonstrations}

A significant fraction of any NLP syllabus covers fundamental data
structures and algorithms. These are usually taught with the help of
formal notations and complex diagrams. Large trees and charts are
copied onto the board and edited in tedious slow motion, or
laboriously prepared for presentation slides. It is more effective to
use live demonstrations in which those diagrams are generated and
updated automatically. NLTK provides interactive graphical user
interfaces, making it possible to view program state and to study
program execution step-by-step. Most NLTK components have a
demonstration mode, and will perform an interesting task without
requiring any special input from the user. It is even possible to make
minor modifications to programs in response to ``what if'' questions. In
this way, students learn the mechanics of NLP quickly, gain deeper
insights into the data structures and algorithms, and acquire new
problem-solving skills.

\subsection{Small Group Discussion}

animate this with a quiz, presented as a slide or a handout, giving code samples and asking what they do.

\section{Project Work}

Group projects involving a mixture of linguists and CS students:
initial appeal is that a CS student will help the linguist student with programming,
and vice versa.  However, there's a
complex dynamic, unpredictable success, linguist probably won't get to program
in the interests of a good project mark.
Multi-stage project mandating stages that require linguistic and CS content: difficult
to foster continuous collaboration, more likely to get e.g. parser being developed by
a CS team member, then thrown over the wall to a linguist member to develop a grammar.

Instead, believe it is more productive in the context of a single-semester introductory
course to have students work on their own projects.  Devise distinct projects for
students depending on background.  Provide a list and give them the option of proposing
other projects.

\url{http://nltk.org/projects.html}

Peer review (including code review) to improve quality of programming, and
emphasize the communicative dimension of programming.
(Even grade a student on the quality of his/her peer review of another student.)






\bibliography{acl-08}
\end{document}
