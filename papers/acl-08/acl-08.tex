% $Rev%
\documentclass[11pt]{article}
\usepackage{acl08}
\usepackage{times}
\usepackage{latexsym}
\usepackage{epsfig,url}
\usepackage{alltt}

\newcommand{\NLP}{\textsc{nlp}}
\newcommand{\NLTK}{\textsc{nltk}}
\newcommand{\code}[1]{\texttt{\small #1}}

\pretolerance 250
\tolerance 500
\hyphenpenalty 200
\exhyphenpenalty 100
\doublehyphendemerits 7500
\finalhyphendemerits 7500
\brokenpenalty 10000
\lefthyphenmin 3
\righthyphenmin 3
\widowpenalty 10000
\clubpenalty 10000
\displaywidowpenalty 10000
\looseness 1

\setlength\titlebox{6.5cm}    % Expanding the titlebox

\title{Multidisciplinary Instruction with the Natural Language Toolkit}
\author{}

% \author{Joakim Nivre \\
%   School of Mathematics and Systems Engineering \\
%   V\"{a}xj\"{o} University \\
%   SE-35195, V\"{a}xj\"{o}, Sweden \\
%   {\tt nivre@msi.vxu.se} \And
%   Noah A. Smith \\
%   Language Technologies Institute \\
%   Carnegie Mellon University \\
%   Pittsburgh, PA 15213, USA\\
%   {\tt nasmith@cs.cmu.edu}}

\date{}

\begin{document}
\maketitle
\begin{abstract}
  The Natural Language Toolkit (\NLTK) is widely used for teaching
  natural language processing to students majoring in linguists or
  computer science.  This paper describes the design of \NLTK, and
  reports on how it has been used effectively in classes that involve
  different mixes of
  linguistics and computer science students.  We focus
  on three key issues: getting started with a course, fostering
  classroom interaction, and organizing project work.
  In each case, we report on practical experience and make
  recommendations on how to use \NLTK\ to maximum effect.
\end{abstract}

\section{Introduction}

It is relatively easy to teach natural language processing (\NLP) in a
single-disciplinary mode to a uniform cohort of students.  Linguists
can be taught to program, leading to projects where students
manipulate their own linguistic data.  Computer scientists can be
taught methods for automatic text processing, leading to projects on
text mining and chatbots.  Yet these approaches have almost nothing in
common, and it is a stretch to call either of these \NLP: more apt
titles for such courses might be `linguistic data management' and
`text technologies'.

To the extent that \NLP\ is a coherent discipline, it has core
knowledge and skills that must be learned by all students.  The
Natural Language Toolkit (\NLTK)\footnote{\url{http://nltk.org}} was
developed to give a broad range of students access to the core
knowledge and skills of \NLP.  In particular, \NLTK\ makes it feasible
to run a course that covers a substantial amount of theory and
practice with an audience consisting of both linguists and computer
scientists.  \NLTK\ is a suite of Python modules distributed under the
GPL open source license.  \NLTK\ comes with a large collection of
corpora, extensive documentation, and hundreds of exercises, making
\NLTK\ unique in providing a comprehensive framework for students to
develop a computational understanding of language.  \NLTK's code base
of 100,000 lines of Python code includes support for corpus access,
tokenizing, stemming, tagging, chunking, parsing, clustering,
classification, language modelling, semantic interpretation,
unification, and much else
besides.\footnote{\url{http://nltk.org/code.html}} As a measure of its
impact, \NLTK\ has been used in over 60 university courses in 20
countries.\footnote{\url{http://nltk.org/courses.html}}

Since its inception in 2001, \NLTK\ has undergone considerable
evolution, based on the experience gained by teaching courses at
several universities, and based on feedback from many teachers and
students.  Over this period, a series of practical online tutorials
about \NLTK\ has grown up into a comprehensive online book.\footnote{\url{http://nltk.org/book.html}}
The book has been designed to stay in lock-step
with the \NLTK\ library, and is intended to be a major support in
helping students follow the mantra: \textit{learn by doing}.

This paper describes the main features of \NLTK, and reports on how it has
been used effectively in classes that involve a combination of
linguists and computer scientists.  First we discuss aspects of the
design of the toolkit that arose from our need to teach computational
linguistics to a multidisciplinary audience (\S\ref{sec:design}).
The following sections cover three distinct challenges:
getting started with a course (\S\ref{sec:getting-started});
interactive demonstrations (\S\ref{sec:interactive-demonstrations});
and organizing assignments and student projects (\S\ref{sec:projects}).

\section{Design Decisions Affecting Teaching}
\label{sec:design}

\subsection{Python}

We chose Python\footnote{\url{http://www.python.org/}} as the
implementation language for \NLTK\ because it has a shallow learning
curve, its syntax and semantics are transparent, and it has good
string-handling functionality.  As an interpreted language, Python
facilitates interactive exploration.  As an object-oriented language,
Python permits data and methods to be encapsulated and re-used easily.
Python comes with an extensive standard library, including tools for
graphical programming and numerical processing, which means it can be
used for a wide range of non-trivial applications.  Python is ideal in
a context serving newcomers and experienced programmers
\cite{Shannon03}.

We have taken the step of incorporating a detailed introduction to
Python programming in the \NLTK\ book, taking care to motivate
programming constructs with linguistic examples. Extensive feedback
from students has been humbling, and revealed that for students with
no prior programming experience, it is almost impossible to
over-explain. Despite the difficulty of providing a
self-contained introduction to Python for linguists, we nevertheless
have also had very positive feedback, and in combination with the
teaching techniques described below, have managed to bring a
large group of non-programmer students rapidly to a point where they
could carry out interesting and useful exercises in text processing.

In addition to the \NLTK\ book, the code in the \NLTK\ core is richly
documented, using Python docstrings and
\textsc{epydoc}\footnote{\url{http://epydoc.sourceforge.net/}} support
for API documentation.\footnote{\url{http://nltk.org/doc/api/}} Access
to the code documentation is available using the Python \code{help()}
command at the interactive prompt, and this can be especially useful
for checking the input parameters and return type of functions.

Other Python libraries are very useful in the \NLP\ context: NumPy
provides optimized support for linear algebra and sparse
arrays,\footnote{\url{http://numpy.scipy.org/}} and PyLab provides
sophisticated facilities for scientific
visualisation.\footnote{\url{http://matplotlib.sourceforge.net/}}



\subsection{Coding Requirements}

As discussed in Loper \& Bird~\shortcite{LoperBird02}, the priorities for \NLTK\ code
focus on its teaching role. When code is readable, a student who
doesn't understand the maths of HMMs, smoothing, and so on may benefit
from looking at how an algorithm is implemented. Thus consistency,
simplicity, modularity are all vital features of \NLTK\ code. A
similar importance is place on extensibility, since this helps to
ensure that the code grows as a coherent whole, rather than by
unpredictable and haphazard additions.  

By contrast, although efficiency cannot be completely ignored, it has
always taken second place to simplicity and clarity of coding. In a
similar vein, we have tried to eschew `clever' programming tricks,
since these typically hinder intelligiblity of the code.  Finally,
comprehensiveness of coverage has never been an overriding concern of
\NLTK---indeed, some students seem to positively prefer the restricted
scope of \NLTK, since they find it less overwhelming. 


\subsection{Naming}

One issue which has absorbed a considerable amount of attention is the
naming of user-oriented functions in \NLTK. To a large extent, the
system of naming \emph{is} the user interface to the toolkit, and it is
important that users should be able to guess what action might be
performed by a given function. Consequently, naming conventions need
to be consistent and semantically transparent. At the same time, there is a
countervailing pressure for relatively succinct names, since excessive verbosity
can also hinder comprehension and usability. An additional
complication is that adopting an object-oriented style of programming
may be well-motivated for a number of reasons but nevertheless
baffling to the linguist student. For example, although it is
perfectly respectable to invoke an instance method
\code{WordPunctTokenizer().tokenize(text)} (for some input
string \code{text}), from a presentational point of view, 
\code{wordpunct\_tokenize(text)} works a lot better
(both are available).


\subsection{Corpus Access}

The scope of exercises and projects that students can perform is
greatly increased by the inclusion of a large collection of corpora,
along with easy-to-use corpus readers.  This collection, which
currently stands at 45 corpora, includes parsed, POS-tagged, plain
text, categorized text, and lexicons.\footnote{\url{http://nltk.org/corpora.html}}

\begin{figure*}[t]
{\small
\begin{alltt}
\textbf{>>> nltk.corpus.treebank.tagged_words()}
[('Pierre', 'NNP'), ('Vinken', 'NNP'), (',', ','), ...]
\textbf{>>> nltk.corpus.brown.tagged_words()}
[('The', 'AT'), ('Fulton', 'NP-TL'), ...]
\textbf{>>> nltk.corpus.floresta.tagged_words()}
[('Um', '>N+art'), ('revivalismo', 'H+n'), ...]
\textbf{>>> nltk.corpus.cess_esp.tagged_words()}
[('El', 'da0ms0'), ('grupo', 'ncms000'), ...]
\textbf{>>> nltk.corpus.alpino.tagged_words()}
[('De', 'det'), ('verzekeringsmaatschappijen', 'noun'), ...]
\end{alltt}}
\caption{Accessing Different Corpora via a Uniform Interface}
\label{fig:tagged}
\vspace*{1ex}\hrule
\end{figure*}

In designing the corpus readers, we emphasised simplicitly,
consistency, and efficiency.  \emph{Corpus objects}, such as
\code{nltk.corpus.brown} and \code{nltk.corpus.treebank}, define
common methods for reading the corpus contents, abstracting
away from idiosyncratic file formats to provide a uniform interface.
See Figure~\ref{fig:tagged} for an example of accessing POS-tagged
data from different tagged and parsed corpora.

The corpus objects provide methods for reading corpus contents
in various formats.  Common methods include:
%
\code{raw()}, for the raw contents of the corpus;
\code{words()}, for a list of tokenized words;
\code{sents()}, for the same list grouped into sentences;
\code{tagged\_words()}, for a list of (\textit{word}, \textit{tag}) pairs;
\code{tagged\_sents()}, for the same list grouped into sentences;
and
\code{parsed\_sents()}, for a list of parse trees.
%
Optional parameters can be used to restrict what portion of the corpus
is returned, e.g., to a section of the corpus, or an individual corpus
file.

Most corpus reader methods return a \emph{corpus view}, which acts as
a list of text objects, but maintains responsiveness and memory
efficiency by only loading items from the file on an as-needed basis.
Thus, when we print a corpus view we only see the first line, but when
we process this object we get the complete data set:

{\footnotesize
\begin{alltt}
\textbf{>>> nltk.corpus.alpino.words()}
['De', 'verzekeringsmaatschappijen',
'verhelen', ...]
\textbf{>>> len(nltk.corpus.alpino.words())}
139820
\end{alltt}}

\subsection{Accessing Shoebox Files}

\NLTK\ provides functionality for working with Shoebox data
\cite{robinson:etal:2007}. Shoebox is a system used by many
documentary linguists to produce lexicons and interlinear glossed text.  The
ability to work straightforwardly with Shoebox data promises to open
up a new avenue for encouraging linguists to learn about the promise
of computational methods. 

As an example, in the Linguistics Department at the University of
XXX, a course has been offered on Python programming and
working with corpora, but so far uptake from the
target audience of core linguistics students has been low. They usually have very
practical needs and concerns when it comes to computational
linguistics, and many of them are also intimidated by the very idea of
programming. % So, there is a high bar for them to get over to sign up
% for any computational linguistics course. 

We believe that the appeal of this course can be significantly
enhanced by designing a significant component
with the goal of helping documentary linguistics students take control of their
\emph{own} Shoebox data. This will give them  skills that are
useful for their research and also transferable to other activities.
Although the \NLTK\ Shoebox functionality was not
originally designed with instruction in mind, its relevance to
students of documentary linguistics is highly fortuitous and
may be relevant at other similar linguistics departments.

\section{Getting Started}
\label{sec:getting-started}

\NLP\ is usually only available as an elective course, and students
will vote with their feet after attending one or two classes.  This
initial period is important for attracting and retaining students.  In
particular, students need to get a sense of the richness of language
in general, and \NLP\ in particular, while gaining a realistic
impression of what will be accomplished during the course and what
skills they will have by the end.  During this time when rapport needs
to be rapidly established, it is easy for instructors to alienate
students through the use of linguistic or computational concepts and
terminology that are foreign to students, or to bore students by
getting bogged down in defining terms like `noun phrase' or `function'
which are basic to one audience and new for the other.  Thus, we
believe it is crucial for instructors to understand and shape the
student's expectations, and to get off to a good start.  The best
overall strategy that we have found is to use succinct `nuggets' of
\NLTK\ code to stimulate students' interest in both data and
processing techniques.

\subsection{Student Expectations}

Computer science students come to \NLP\ expecting to learn about \NLP\
algorithms and data structures.  They typically have enough
mathematical preparation to be confident in `playing' with abstract
formal systems (including systems of linguistic rules).  Moreover,
they are already proficient in multiple programming languages, and
have little difficulty in learning \NLP\ algorithms by reading and
manipulating the implementations provided with \NLTK. At the same
time, they tend to be unfamiliar with the terminology and concepts
that linguists take for granted, and may struggle to come up with
`reasonable' linguistic analyses of data.

Linguistics students, on the other hand, are interested in
understanding \NLP\ algorithms and data structures only insofar as it helps them
to use computational tools to perform analytic tasks from `core linguistics',
e.g.\ writing a set of CFG productions to parse some sentences, or
plugging together \NLP\ components in order to derive the subcategorization
requirements of verbs in a corpus.
They are usually not interested in reading significant chunks of code;
it isn't what they care about and they
probably lack the self-confidence to poke around in source files.

In a nutshell, the computer science students typically want to analyze
the tools and synthesize new implementations, while the linguists
typically want to use the tools to analyze language and
synthesize new theories.  There is a risk that the former group
never really gets to grips with natural language, while the latter
group never really gets to grips with processing.  Instead,
computer science students need to learn that \NLP\ is not just an
application of techniques from formal language theory and compiler
construction, and linguistics students need to understand that \NLP\ is not
just computer-based housekeeping and a solution to the shortcomings of
office productivity software for managing their linguistic data.

In many courses, linguistics students or computer science students
will dominate the class numerically, simply because the course is only
listed in one department.  In such cases it is usually enough to
provide additional support in the form of some extra readings,
tutorials, and exercises in the opening stages of the course.  In
other cases, e.g.\ courses we have taught at the universities of YYY
and ZZZ, or in summer intensive programs in several countries, there
is more of an even split, and the challenge of serving both cohorts of
students becomes acute.

\subsection{Articulating the Goals}

Despite an instructor's efforts to add a cross-disciplinary angle, students
easily `revert to type'.  The pressure of assessment encourages students to emphasize
what they do well.  Students' desire to understand what is expected of them encourages
instructors to stick to familiar assessment instruments.  As a consequence,
the path of least resistance is for students to remain firmly
`monolingual' in their own discipline, while
acquiring a smattering of words from a foreign language, at a level we might
call `survival linguistics' or `survival computer science'.

Asking computer science students to write their first essay in years, or asking
linguistics students to write their first ever program, leads to
stressed students who complain that they don't know what is expected
of them.

Students need to confront the challenge of becoming bilingual, of
working hard to learn the basics of another discipline.  In parallel,
instructors need to confront the challenge of synthesizing material
from linguistics and computer science into a coherent whole, and
devising effective methods for teaching, learning, and assessment.

\subsection{The First Lecture}

It is important that the first lecture is effective at motivating and
exemplifying \NLP\ to an audience of computer science and linguistics
students.  They need to get an accurate sense of the interesting
conceptual and technical challenges awaiting them.  Fortunately, the
task is made easier by the simple fact that language technologies, and
language itself, are intrinsically interesting and appealing to a wide audience.
Several opening topics appear to work particularly well.

\textbf{The holy grail:}
A long term challenge, mythologized in science fiction movies, is to
build machines that understand human language.  Current technologies
that exhibit some basic level of natural language understanding include
spoken dialogue systems, question answering systems, summarization
systems, and machine translation systems.  These can be demonstrated
in class without too much difficulty.  The Turing test is a linguistic
test, easily understood by all students, and which helps the computer science
students to see \NLP\ in relation to the field of Artificial Intelligence.
The evolution of programming languages has brought them closer to natural language,
helping students see the essentially linguistic purpose of this central development
in computer science.  The corresponding holy grail in linguistics is full
understanding of the human language faculty; writing programs and building machines
surely informs this quest too.

\textbf{The riches of language:}
It is easy to find examples of the creative richness of language in
its myriad uses.  However, linguists will understand that language
contains hidden riches that can only be uncovered by careful analysis
of large quantities of linguistically annotated data, work that
benefits from suitable computational tools.  Moreover, the
computational needs for exploratory linguistic research often goes
beyond the capabilities of the current tools.  Computer scientists
will appreciate the cognate problem of extracting information from the
web, and the economic riches associated with state-of-the-art text
mining capabilities.

\textbf{Formal approaches to language:}
Computer science and linguistics have a shared history in the area of
philosophical logic and formal language theory.  Whether the language
is natural or artificial, computer scientists and linguists use
similar logical formalisms for investigating the formal semantics of
languages, similar grammar formalisms for modelling the syntax of
languages, and similar finite-state methods for manipulating text.
Both rely on the recursive, compositional nature of natural and
artificial languages.

\subsection{First Assignment}

The first coursework assignment can be a significant step forwards in
helping students get to grips with the material, and is best given out
early, perhaps even in week 1.  We have found it advisable for this
assignment to include both programming and linguistics content. One
example is to ask students to carry out NP chunking of some data
(e.g.\ a section of the Brown Corpus). The \code{nltk.RegexpParser}
class is initialized with a set of chunking rules expressed in a
simple, regular expression-oriented syntax, and the resulting chunk
parser can be run over POS-tagged input text. Given a Gold Standard
test set like the CoNLL-2000
data,\footnote{\url{http://www.cnts.ua.ac.be/conll2000/chunking/}}
precision and recall of the chunk grammar can be easily determined.
Thus, if students are given an existing, incomplete set of rules as
their starting point, they just have to modify and test their rules.

There are distinctive outcomes for each set of students: linguistics students
learn to write grammar fragments that respect the literal-minded
needs of the computer, and also come to appreciate the noisiness of
typical \NLP\ corpora (including automatically annotated corpora like
CoNLL-2000).
Computer science students become more familiar with parts of speech
and with typical syntactic structures in English. Both groups learn
the importance of formal evaluation using precision and recall.

\section{Interactive Demonstrations}
\label{sec:interactive-demonstrations}

\subsection{Demonstrations with the Python Interpreter}

Python fosters a highly interactive style of teaching.  It is quite
natural to build up moderately complex programs in front of a class,
with the weaker students transcribing it into a Python session on
their laptop to satisfy themselves it works (but not necessarily
understanding everything they enter first time), while the stronger
students quickly grasp the theoretical concepts and algorithms.  While
both groups can be served by the same presentation, they tend to ask
quite different questions.  However, this is addressed by dividing
them into smaller clusters and having teaching assistants visit them
separately to discuss issues arising from the content.

The \NLTK\ book contains a large number of examples, and the instructor
can present an interactive lecture that includes running these examples,
and experimenting with them in response to student questions.  In early
classes, the focus will probably be on learning the language.  In later classes,
the driver for such interactive lessons can be an externally motivated
empirical or theoretical question.

As a practical matter, it is important to consider some low-level issues
that may get in the way of students' ability to capture the material
covered in interactive Python sessions.  These include choice of
appropriate font size for screen display, avoiding the problem of output scrolling the
command out of view, and distributing a log of the instructor's interactive session
for students to study in their own time.


\subsection{NLTK Demonstrations}

A significant fraction of any \NLP\ syllabus covers fundamental data
structures and algorithms. These are usually taught with the help of
formal notations and complex diagrams. Large trees and charts are
copied onto the board and edited in tedious slow motion, or
laboriously prepared for presentation slides. It is more effective to
use live demonstrations in which those diagrams are generated and
updated automatically. \NLTK\ provides interactive graphical user
interfaces, making it possible to view program state and to study
program execution step-by-step. Most NLTK components have a
demonstration mode, and will perform an interesting task without
requiring any special input from the user. It is even possible to make
minor modifications to programs in response to ``what if'' questions. In
this way, students learn the mechanics of \NLP\ quickly, gain deeper
insights into the data structures and algorithms, and acquire new
problem-solving skills.

An example of a particularly effective set of demos are those for
shift-reduce and recursive descent parsing. These make the difference
between the algorithms glaringly obvious. More importantly, students
get a very concrete sense of many issues that affect the design of
algorithms for tasks like parsing. The partial analysis constructed by
the recursive descent parser bobs up and down as it steps forward and
backtracks, and students often go wide-eyed as the parser retraces its
steps and does ``dumb'' things like expanding N to {\it man} when it
has already tried the rule unsuccessfully (but is now trying to match
a bare NP rather than an NP with a PP modifier). Linguistics students
who are extremely knowledgeable about context-free grammars and thus
understand the representations gain a new appreciation for just how
naive an algorithm can be. This gives them a very concrete appreciate
for the need for techniques like dynamic programming and motivates
them to learn how they work and can be used to solve such problems
much more efficiently.

Another highly useful aspect of \NLTK\ is the ability to define a
context-free grammar using a very simple format and to display tree
structures graphically. This can be used to teach context-free
grammars interactively, where the instructor and the students develop
a grammar from scratch and check its coverage against a testbed of
sentences (including grammatical and ungrammatical ones). Because it
is so easy to modify the grammar and check its behavior, students
readily participate and suggest various solutions. When the grammar
produces an analysis for an ungrammatical sentence in the testbed, the
tree structure can be displayed graphically and inspected to see what
went wrong. Conversely, textual representations of the CKY parse chart
can be used to see where the grammar failed on grammatical sentences.

\NLTK's easy access to many corpora also greatly facilitates classroom
instruction. It is straightforward to pull in different sections of
corpora and build programs in class for many different tasks, from
simple things like doing word counts to building rule-based
part-of-speech taggers. This not only makes it easier to experiment
with ideas on the fly, but also allows students to replicate the
exercises easily outside of class. Graphical displays that show the
dispersion of terms throughout a text also give students excellent
examples of how a few simple statistics collected from a corpus can
provide useful and interesting views on a text---including seeing the
frequency with which various characters appear in a novel. This can in
turn be related to other resources like Google Trends, which shows the
frequency with which a term has been referenced in news reports or
been used in search terms over several years.

% \subsection{Small Group Interaction}
% 
% Even the most engaging interactive Python demonstration may only
% amount to a demonstration of the instructor's Python prowess ('charming python').
% Student learning is enhanced when they are encouraged to engage actively
% with the material, responding to a quiz, discussing ...

% animate this with a quiz, presented as a slide or a handout, giving code samples and asking what they do.
% could be a competition
% class exercise, e.g. counting uses of ``must'' in a spoken vs written language corpus
% (deontic vs epistemic uses).
% chatroom for online discussion: useful during intensive summer program; otherwise couldn't be staffed adequately

\section{Exercises, Assignments and Projects}
\label{sec:projects}

\subsection{Exercises}

Copious exercises, graded for difficulty, are provided with the \NLTK\ documentation. These
exercises have the tremendous advantage of building on the \NLTK\
infrastructure---both code and documentation. The exercises are
intended to be suitable both for self-based learning and also
required coursework.

A mixed class of linguistics and computer science students will have a
diverse range of programming experience, and students with no
programming experience will typically have different aptitudes for
programming \cite{Caspersen07}.  A course which forces all students to
progress at the same rate will be too difficult for some, and too dull
for others, and will risk alienating many students.  Thus, course
materials need to accommodate self-paced learning.  An effective way
to do this is to provide students with contexts in which they can test
and extend their knowledge at their own rate.

One such context is provided by lecture or laboratory sessions in
which students have a machine in front of them (or one between two),
and where there is time to work through a series of exercises to
consolidate what has just been taught from the front, or read from a
tutorial. When this can be done at regular intervals, it is easier for
students to know which part of the materials to re-read.  It also
encourages them to get into the habit of checking their own
understanding of a concept by writing some program code.

When exercises are graded for difficulty, it is easier for students to
understand how much effort is expected, and whether they even have
time to attempt an exercise.  Graded exercises are also good for
supporting self-evaluation.  If a student takes 20 minutes to write a
solution, they also need to have some idea of whether this was an
appropriate amount of time.

% well-motivated questions: needs to be transparently obvious why this
% is a linguistically meaningful thing to do (esp for inexperienced
% programmers)

% include open-ended questions for the experienced programmers

% For students who are
% learning to program as part of a computational linguistics course, the
% parallels between the examples in the documentation and the
% requirements of the assignments is very helpful. As a result, they
% are able use \NLTK\ to carry out far more complex tasks than they could
% otherwise have hoped to do.
% The availability of online examples that
% they could try out in interactive Python were a huge help for them.


\begin{figure}
{\footnotesize
\begin{verbatim}
>>> nltk.FreqDist(nltk.corpus.brown.words())
\end{verbatim}
\medskip

\begin{verbatim}
>>> fd = nltk.FreqDist()
>>> for filename in corpus_files:
...     raw_text = open(filename).read()
...     for word in nltk.tokenize(raw_text):
...         fd.inc(word)
\end{verbatim}
\medskip

\begin{verbatim}
>>> counts = {}
>>> for word in nltk.corpus.brown.words():
...     if word not in counts:
...         counts[word] = 0
...     counts += 1
\end{verbatim}
}
\caption{Three Ways to Build up a Frequency Distribution of Words in the Brown Corpus}
\label{fig:freqdist}
\vspace*{1ex}
\hrule
\end{figure}

The exercises are also highly adaptable. It is common for instructors
to take them as a starting point in building homework assignments that
are tailored to their own students.  Some instructors prefer to
include exercises that do not allow students to take advantage of
built-in \NLTK\ functionality, e.g.\  using a Python dictionary to
count word frequencies in the Brown corpus rather than \NLTK 's
\code{FreqDist} (see Figure~\ref{fig:freqdist}).  This is an important
part of building facility with general text processing in Python,
since eventually students will have to play outside of the \NLTK\
sandbox, particularly when they start looking at larger
corpora. Nonetheless, students often use \NLTK\ functionality as part
of their solutions, e.g., for managing frequencies and
distributions. Again, this flexibility is a good thing: students learn
to work with resources they know how to use, and can branch out to new
exercises from that basis. When course content includes discussion of
Unix command line utilities for text processing, students can
furthermore gain a better appreciation of the pros and cons of writing
their own scripts versus using an appropriate Unix pipe.

\subsection{Assignments}
\NLTK\ supports assignments of varying difficulty and scope. In the
simplest assignments, students experiment with existing components to
perform a wide variety of \NLP\ tasks. This may involve no programming
at all, in the case of the existing demonstrations, or simply changing
a line or two of program code. As students become more familiar with
the toolkit they can be asked to modify existing components or to
create complete systems out of existing components. \NLTK\ also provides
students with a flexible framework for advanced projects, such as
developing a multi-component system by integrating and extending \NLTK\
components, and adding on entirely new components. Here \NLTK\ helps by
providing standard implementations of all the basic data structures
and algorithms, interfaces to standard corpora, substantial corpus
samples, and a flexible and extensible architecture.

\subsection{Projects}

Group projects involving a mixture of linguists and computer science
students have an initial appeal, assuming that each kind of student
can learn from the other.  However, there's a complex social dynamic
in such groups, one effect of which is that the linguistics students
may opt out of the programming aspects of the task, perhaps with view
that their contribution would only hurt the chances of achieving a
good overall project mark. It is difficult to mandate significant
collaboration across disciplinary boundaries, with the more likely
outcome being, for example, that a parser is developed by a computer
science team member, then thrown over the wall so that a linguist
member can develop an appropriate grammar.

Instead, we believe that it is generally more productive in the
context of a single-semester introductory course to have students work individually
on their own projects.  Distinct projects can be devised for students
depending on their background, or students can be given a list of
project topics,\footnote{\url{http://nltk.org/projects.html}} and offered option of
self-proposing other projects.

% Peer review (including code review) to improve quality of programming, and
% emphasize the communicative dimension of programming.
% (Even grade a student on the quality of his/her peer review of another student.)

\section{Conclusion}

We have argued that the distinctive features of \NLTK\ make it an apt
vehicle for teaching \NLP\ to mixed audiences of linguistic and
computer science students. On the one hand, complete novices can
quickly gain confidence in their ability to do fun and useful things
with language processing, while the transparency and consistency of
the implementation also makes it easy for experienced programmers to
learn about natural language and to explore more challenging
programming tasks. The success of this recipe is borne out by the wide
uptake of the toolkit, not only within tertiary education but more
broadly by users who just want try their hand at \NLP. We also have
encouraging results in presenting \NLTK\ in classrooms at the
secondary level, thereby trying to inspire the computational linguists of the
future!

Finally, we believe that \NLTK\ has gained much by participating 
in the Open Source software movement, specifically from
the infrastructure provided by
Sourceforge\footnote{\url{http://sourceforge.net/}} and from the
invaluable contributions of a wide range of people, including many
students.

\bibliographystyle{acl}
\bibliography{acl-08}

\end{document}
