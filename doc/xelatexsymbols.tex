%&program=xelatex
%&encoding=UTF-8 Unicode

\newcommand{\as}[1]{{\fontspec{Apple Symbols}#1}}
\newcommand{\asb}[1]{{\fontspec[Scale=1.1]{Apple Symbols}#1}}
\newcommand{\ls}[1]{{\fontspec[Scale=0.9]{Lucida Grande}#1}}
\def\reflect#1{{\setbox0=\hbox{#1}\rlap{\kern0.5\wd0
  \special{x:gsave}\special{x:scale -1 1}}\box0 \special{x:grestore}}}
\def\XeLaTeX{\leavevmode
  \setbox0=\hbox{X\lower.5ex\hbox{\kern-.15em\reflect{E}}\kern-.0833em \LaTeX}%
  \dp0=0pt\ht0=0pt\box0 }

\documentclass[11pt]{article}
\title{Math Symbols in \XeLaTeX}
\author{Ewan Klein}
\date{\today}
\usepackage{fontspec}
\setromanfont{Palatino}

\begin{document}
\maketitle
\section{Introduction}

This file tests direct insertion of unicode characters using cut and paste
from the Mac OS X font book application. As far as I can tell, some of
the characters are only available in the Apple Symbols font, and some
which you might have expected to be in Apple Symbols aren't, but can
be found instead in Lucida Grande, for example.

I still haven't found a way of producing angle brackets.


\begin{itemize}
\item $P$ \as{≐} $R$ \as{∧} $Q$
\item Z ≠ X \as{⊆} Y and X \as{↦} Y
\item X \as{⊑} Y
\item a ≥ b ≤ c < d > e
\item p \ls{→} \as{⊥}
\item 3 \as{╳} 0.2 = (0.3)\as{⋅}2
\item 3 — \as{〈}4, 5 \as{〉}
\item A \as{⊓} B = A if \as{≡} \ls{¬}
\item \asb{∃}x\asb{∀}y.R(x) \as{→} R(y)
\item $x$ \as{∈} $Y$
\end{itemize}



\end{document}
