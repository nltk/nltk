%%%%%%%%%%%%%%%%%%%%%%%%%%%%%%%%%%%%%%%%
% Revised: Time-stamp: <2006-02-13 13:05:54 ewan>
% 
%
%%%%%%%%%%%%%%%%%%%%%%%%%%%%%%%%%%%%%%%%

% NB
% Add some diagrams of FSAs!!!
% add something on sub
% add regexp tokenizer
% add another example of using REs?
% 
% Named Groups example -- not a good place to introduce string formatting!

\title{Regular Expressions}
\author{Ewan Klein \newline \mbox{ }ewan@inf.ed.ac.uk\mbox{ }}
\date{ICL --- 10 October 2005}



%%%%%%%%%%%%%%%%%%%%%%%%%%%%%%%%%%%%%%%%%%%%%%%%%%%%%%%%%%%%
\usepackage{fancybox}
\usepackage{color}
\usepackage{amsmath}
\usepackage{graphicx}
\usepackage{alltt}
\usepackage{url}


% Shading
\definecolor{light}{gray}{.80}
\newcommand{\Hilite}[1]{\colorbox{yellow}{#1}}
\newcommand{\Shade}[1]{\colorbox{light}{#1}}
%\newcommand{\Hilite}[1]{\colorbox{white}{#1}}
%\newcommand{\Shade}[1]{\colorbox{white}{#1}}


\newcommand{\Comment}[1]{\textcolor{blue}{#1}}
\newcommand{\Em}[1]{\textcolor{red}{#1}}
\newcommand{\Dim}[1]{\textcolor{gray}{#1}}

\setlength{\parskip}{0in}
\setlength{\parindent}{0in}


\begin{document}

\frame{\titlepage}



\mode<article>{\section[Outline]{ICL/Regular Expressions/2005-10-10}}
\mode<presentation>{
  \section[Outline]{}
}

\frame{\tableofcontents}


\section{Overview of REs}

\subsection{Introduction}

\begin{frame}[fragile]

Goals: 
\begin{itemize}
  \item a basic idea of the formal background for REs
  \item an ability to write small Python programs that do 
useful things with REs
\end{itemize}

\end{frame}

\begin{frame}[fragile]
  \frametitle{Motivation}

  \begin{description}[<+->]
  \item [Task:] To search for strings using (partially
      specified) \Em{patterns}
  \item [Why:]\hfill\\ 
    \begin{itemize}
    \item validate data fields (dates, email addresses, URLs)
    \item filter text (spam, disallowed web sites)
    \item identify particular strings in a text (token boundaries for tokenization)
    \item convert the output of one processing component into the format
          required for a second component  (\verb!rabbit_NN!
          $\rightarrow$ \verb!<word pos=''NN''>rabbit</word>!) 
    \end{itemize}

  \end{description}
\end{frame}


\begin{frame}[fragile]
  \frametitle{The Basic Idea}

  \begin{itemize}[<+->]
    \item \Em{Regular expressions} form a language for expressing patterns.
    \item The language can be stated as a formal algebra.
    \item Recognizers for RE can be efficiently implemented.
    \item `Regular expression' also a term for a pattern that is
      constructed using the language.
    \item Every pattern specifies a \Em{set of strings}.
    \item Text string: a sequence of letters, numerals, spaces, tabs,
      punctuation, \ldots
  \end{itemize}
\end{frame}

\begin{frame}[fragile]
  \frametitle{Initital Examples}

  \begin{tabular}{lll} \hline
              & \Em{Pattern}   & \Em{Matches} \\ \hline
concatenation & \textbf{abc} & abc \\ \hline
disjunction   & \textbf{a} $\mid$ \textbf{b} & a, b \\
              & (\textbf{a} $\mid$ \textbf{bb}) \textbf{d} & ad, bbd\\ \hline
closure       & \textbf{a}* & $\epsilon$, a, aa, aaa, aaaa, \ldots\\ 
              & \textbf{c}(\textbf{a} $\mid$ \textbf{bb})* & c, ca, cbb,
                                                             cabb, caa, cbbbb,
                                                             \ldots\\ \hline
             

  \end{tabular}
\end{frame}

\begin{frame}[fragile]
\frametitle{Two Types of RE}

\begin{description}[<+->]
  \item [Literals] Every normal text character is an RE, and denotes
    itself.
  \item [Metacharacters] Special characters which allow you to specify
    various sets of strings.
\end{description}

\onslide<3->

Example---Kleene star (*)
\begin{itemize}
  \item \textbf{a} denotes \textit{a}
  \item \textbf{a*} denotes $\epsilon$ (empty string), \textit{a},
    \textit{aa}, \textit{aaa}, \ldots
  
\end{itemize}
\end{frame}

\subsection{Formal Background to REs}

\begin{frame}[fragile]
  \frametitle{Preliminaries: Operations on Sets of Strings}

    Let $\Sigma$ be a finite set of symbols and let $\Sigma^*$ be the set
    of all strings (including the empty string) over $\Sigma$. Suppose
    $L, L_{1}, L_{2}$ are subsets of $\Sigma^*$.

\pause
  \begin{itemize}[<+->]
  
    \item The \textit{union} of $L_{1}, L_{2}$,
      denoted $L_{1}\cup L_{2}$,  is the set of
      strings $x$ such that
      $x \in L_{1}$ or $x \in L_{2}$.
   \item The \textit{concatenation} of $L_{1}, L_{2}$,
      denoted $L_{1}L_{2}$,  is the set of
      strings $xy$ such that
      $x \in L_{1}$ and $y \in L_{2}$.
    \item The \textit{Kleene closure} of $L$, denoted $L^*$, is the set of
      strings constructed by concatenating any number of strings from
      $L$. $L^*$ contains $\epsilon$, the empty string.
    \item The \textit{positive closure} of $L$, denoted $L^+$, is the same as
      $L^*$ but without $\epsilon$.

  \end{itemize}
\end{frame}

\begin{frame}[fragile]
  \frametitle{Examples}

Let $L_1 = \{a, b\} \text{ and } L_2 = \{c\}$. Then
\pause

\begin{itemize}[<+->]
  \item $L_{1}\cup L_{2} = \{a, b, c\}$
  \item $L_{1}L_{2} = \{ac, bc\}$
  \item $\{a, b\}^* = \{\epsilon, a, b, aa, bb, ab, ba, \ldots\}$
  \item $\{a, b\}^+ = \{a, b, aa, bb, ab, ba, \ldots\}$
\end{itemize}
\end{frame}

\begin{frame}[fragile]
  \frametitle{Formal Definition of Regular Expressions}

Regular expressions over a finite alphabet $\Sigma$:
\pause

\begin{enumerate}[<+->]
%\item $\emptyset$  is a regular expression and denotes the empty set.
\item $\epsilon$ is a regular expression and denotes the set
  $\{\epsilon\}$.
\item For each $a$ in $\Sigma$, \boldmath{a} is a regular expression and
  denotes the set $\{a\}$.
\item If $r$ and $s$ are regular expressions denoting the sets $R$ and
  $S$ respectively, then 
  \begin{itemize}
    \item $(r \mid s)$ is a regular expression denoting $R \cup S$.
    \item $(rs)$ is a regular expression denoting $RS$.
    \item $(r^*)$ is a regular expression denoting $R^*$.
  \end{itemize}

\end{enumerate}
\end{frame}

\begin{frame}[fragile]
  \frametitle{Recognizers}

  \begin{itemize}[<+->]
  \item A \Em{recognizer} for a language is a program that takes as input a
string $x$ and answers ``yes'' if $x$ is a sentence of the language and
``no'' otherwise.
\item We can think of this program as a machine which only emits two possible
responses to its input.
  \end{itemize}

\end{frame}

\begin{frame}[fragile]
 \frametitle{Finite State Automata\hfill \includegraphics[scale=.15]{../images/skleene}}

\begin{itemize}[<+->]
\item A Finite State Automaton (FSA) is an \Em{abstract finite machine}.

\item Regular expressions can be viewed as a way to describe a Finite
  State Automaton (FSA)
  
\item Kleene's theorem (1956): FSA and RE describe the same languages:
  \begin{itemize}
    \item Any regular expression can be implemented as an FSA.
    \item Any FSA can be described by a regular expression.
  \end{itemize}
  
\item Regular languages are those that can be \Em{recognized} by FSAs (or
  characterized by a regular expression).
  \end{itemize}

\end{frame}

\subsection{Extensions of Basic REs}

\begin{frame}[fragile]
  \frametitle{Metacharacters}

NB. Different sets of metacharacters and notations used by different `host languages' (e.g., Unix
grep, GNU emacs, Perl, Java, Python,  etc.). Cf. Jurafsky \& Martin, Appendix A)

\begin{description}[<+->]
  \item [Disjunction:] \textbf{$\mid$}
  \item [Wild card:] \textbf{}.
  \item [Optionality:] \textbf{}?
  \item [Quantification:] \textbf{*} and \textbf{+}
  \item [Choice:] \textbf{[Mm]} \textbf{[0123456789]}
  \item [Ranges:] \textbf{[a-z]} \textbf{[0-9]}
  \item [Negation:] \textbf{[$^\wedge$Mm]} (only when `$^\wedge$' occurs immediately
    after `[')

\end{description}
\end{frame}
\begin{frame}[fragile]
  \frametitle{Special Backslash Sequences}

  \begin{itemize}[<+->]
  \item Standard escape sequences\\
  \begin{tabular}{l}
\verb!\t!: tab \\
\verb!\n!: newline \\
  \end{tabular}

  \item Abbreviatory forms\\
  \begin{tabular}{ll}
\verb!\d!: digit (i.e., numeral) & \verb!\D!: non-digit \\
\verb!\s!: `whitespace' ([ \verb!\t\n!]) & \verb!\S!: non-whitespace \\
\verb!\w!: `alphanumeric' ([a-zA-Z0-9]) & \verb!\W!: non-alphanumeric \\
  \end{tabular}

   \item \verb!\! is a general escape character; e.g., \verb!\.! is
         not a wildcard, but
         matches a period, \verb!.!
   \item If you want to use \verb!\! in a string, it has to be
         escaped: \verb!\\!

  \end{itemize}





\end{frame}

\begin{frame}[fragile]
  \frametitle{Anchors}

(Also: zero-width characters)

\begin{itemize}[<+->]
  \item Anchors don't match strings in the text, instead
  \item they match \Em{positions} in the text.\\

 \begin{tabular}{lp{3in}}
\verb!^!: & matches beginning of line (or text)\\
\verb!$!: & matches end of line (or text) \\%$
\verb!\b!: &  matches word boundary (i.e., a location with \verb!\w! on one side
but not the other)  \\
  \end{tabular}
\end{itemize}
 
\end{frame}

\section{REs in Python}

\subsection{Examples with \texttt{re\_show}}

\begin{frame}[fragile]
\frametitle{Wildcard}

\begin{verbatim}
>>> from nltk_lite.utilities import re_show
>>> s = '''BP has agreed to sell
... it's petrochemicals unit for $5.1bn.'''
>>> re_show('...', s)
{BP }{has}{ ag}{ree}{d t}{o s}{ell}
{it'}{s p}{etr}{och}{emi}{cal}{s u}{nit}{ fo}{r $}{5.1}{bn.}
\end{verbatim}

\onslide<2->
\begin{verbatim}
>>> re_show('.a..', s)
BP {has }agreed to sell
it's petrochemi{cals} unit for $5.1bn.
\end{verbatim}

\end{frame}



\begin{frame}[fragile]
\frametitle{Wildcards with Quantifiers}


\begin{verbatim}
>>> re_show('s.*l', s)
BP ha{s agreed to sell}
it'{s petrochemical}s unit for $5.1bn.
\end{verbatim}

\onslide<2->
\begin{verbatim}
>>> re_show('B.*P', s)
{BP} has agreed to sell
it's petrochemicals unit for $5.1bn.
\end{verbatim}

\onslide<3->
\begin{verbatim}
>>> re_show('B.+P', s)
BP has agreed to sell
it's petrochemicals unit for $5.1bn.
\end{verbatim}
\end{frame}

\begin{frame}[fragile]
\frametitle{Disjunction}


\begin{verbatim}
>>> re_show('has|it', s)
BP {has} agreed to sell
{it}'s petrochemicals un{it} for $5.1bn.
\end{verbatim}

\onslide<2->
\begin{verbatim}
>>> re_show('has | it', s)
BP {has }agreed to sell
it's petrochemicals unit for $5.1bn.
\end{verbatim}

\onslide<3->
\begin{verbatim}
>>> re_show('(e|l)+', s)
BP has agr{ee}d to s{ell}
it's p{e}troch{e}mica{l}s unit for $5.1bn.
\end{verbatim}
\end{frame}


\begin{frame}[fragile]
\frametitle{Zero Width Characters}


\begin{verbatim}
>>> re_show('l', s)
BP has agreed to se{l}{l}
it's petrochemica{l}s unit for $5.1bn.
\end{verbatim}

\onslide<2->
\begin{verbatim}
>>> re_show('l$', s)
BP has agreed to sel{l}
it's petrochemicals unit for $5.1bn.
\end{verbatim}

\onslide<3->
\begin{verbatim}
>>> re_show('i', s)
BP has agreed to sell
{i}t's petrochem{i}cals un{i}t for $5.1bn.
\end{verbatim}

\onslide<4->
\begin{verbatim}
>>> re_show('^i', s)
BP has agreed to sell
{i}t's petrochemicals unit for $5.1bn.
\end{verbatim}

\end{frame}


\begin{frame}[fragile]
\frametitle{Escaping Special Characters}


\begin{verbatim}
>>> re_show('.', s)
{B}{P}{ }{h}{a}{s}{ }{a}{g}{r}{e}{e}{d}...
\end{verbatim}

\onslide<2->
\begin{verbatim}
>>> re_show('\.', s)
BP has agreed to sell
it's petrochemicals unit for $5{.}1bn{.}
\end{verbatim}

\onslide<3->
\begin{verbatim}
>>> re_show('$', s)
BP has agreed to sell{}
it's petrochemicals unit for $5.1bn.{}
\end{verbatim}

\onslide<4->
\begin{verbatim}
>>> re_show('\$', s)
BP has agreed to sell
it's petrochemicals unit for {$}5.1bn.
\end{verbatim}
\end{frame}

\begin{frame}[fragile]
\frametitle{Metacharacters and Negated Ranges}


\begin{verbatim}
>>> re_show('\w',s)
{B}{P} {h}{a}{s} {a}{g}{r}{e}{e}{d} ...
\end{verbatim}

\onslide<2->
\begin{verbatim}
>>> re_show('\d',s)
BP has agreed to sell
it's petrochemicals unit for ${5}.{1}bn.
\end{verbatim}

\onslide<3->
\begin{verbatim}
>>> re_show('[^a-z\s]',s)
{B}{P} has agreed to sell
it{'}s petrochemicals unit for {$}{5}{.}{1}bn{.}
\end{verbatim}

\onslide<4->
\begin{verbatim}
>>> re_show('[^\w]',s)
BP{ }has{ }agreed{ }to{ }sell{
}it{'}s{ }petrochemicals{ }unit{ }for{ }{$}5{.}1bn{.}
\end{verbatim}
\end{frame}

\subsection{Match objects in Python}

\begin{frame}[fragile]
\frametitle{Using REs in Python, 1}

\begin{itemize}[<+->]
\item Usually best to compile the RE into a PatternObject; more
  efficient, and it can be re-used.
\begin{verbatim}
>>> import re
>>> str = 'do you say hello or hullo?'
>>> helloRE = re.compile('h[eu]llo')
\end{verbatim}
\item The resulting PatternObject has a number of methods:
\end{itemize}

\onslide<3->
\begin{description}[<+->]
  \item [findall(s):] returns a list of \Em{all} matches of pattern in string \texttt{s}
  \item [search(s):] searches for \Em{leftmost} occurrence of pattern in string \texttt{s}
  \item [match(s):] tries to match pattern at the \Em{beginning} of string \texttt{s}
\end{description}

\end{frame}

\begin{frame}[fragile]
\frametitle{Using REs in Python, 2}

\begin{itemize}[<+->]
\item The PatternObject method\texttt{findall} returns a \Em{list}:
\begin{verbatim}
>>> helloRE.findall(str)
['hello', 'hullo']
\end{verbatim}
\item  The PatternObject method \texttt{search} (and \texttt{match)} returns a MatchObject or None.
\item A MatchObject has a variety of methods, but is not a string.
\begin{verbatim}
>>> m = helloRE.search(str)
>>> m
<_sre.SRE_Match object at 0x47b138>
>>> m.group() # return matched substring (sort of!)
'hello'
>>> m.end() # index of end of target
16
\end{verbatim}

\end{itemize}
\end{frame}


\begin{frame}[fragile]
\frametitle{Groups}

\begin{itemize}[<+->]
\item Groups in regular expressions are captured using parentheses.
\begin{verbatim}
>>> import re
>>> str = 'do you say hello or hullo?'
>>> reGRP = re.compile('(d.)(.*)(e..)')
>>> m = reGRP.search(str)
>>> m
<_sre.SRE_Match object at 0x64390>
>>> m.groups()
('do', ' you say h', 'ell')
\end{verbatim}
\end{itemize}

\end{frame}

\begin{frame}[fragile]
\frametitle{Named Groups}

\begin{itemize}[<+->]
\item Name groups captured using \verb!(?P<name>)!:
\end{itemize}
{\small
\begin{verbatim}
FROM = re.compile("""
     ^From:          # Anchor to start of line
     \s*             # maybe some spaces
     (?P<user>\w+)   # 'user': group of word characters 
     @               
     (?P<domain>     # the 'domain':
     \S+)            # some non-space characters
     \s              # finally, a space character
     """,re.VERBOSE)
\end{verbatim}
}
\end{frame}

\begin{frame}[fragile]
\frametitle{Named Groups (cont.)}
{\small
\begin{verbatim}
from nltk_lite.corpus import twenty_newsgroups

for item in twenty_newsgroups.items('misc.forsale'):
    text = twenty_newsgroups.read(item)
    m = FROM.search(text)
    if m:
      print '%s is at %s' % \
      (m.group('user'), m.group('domain'))
\end{verbatim}
}

\onslide<2->
\begin{verbatim}
kedz is at bigwpi.WPI.EDU
myoakam is at cis.ohio-state.edu
gt1706a is at prism.gatech.EDU
jvinson is at xsoft.xerox.com
hungjenc is at usc.edu
thouchin is at cs.umr.edu
kssimon is at silver.ucs.indiana.edu
\end{verbatim}
\end{frame}

\section{Reading}



\begin{frame}[fragile]
\frametitle{Reading}



\begin{itemize}
  \item {\large Jurafsky \& Martin, Chap 2}
  \item {\large NLTK Lite Tutorial: Regular Expressions}
available from 
\url{http://nltk.sourceforge.net/lite/doc/en/regexps.html}
\end{itemize}


\end{frame}




%%%%%%%%%%%%%%%%%%%%%%%%%%%%%%%%%%%%%%%%%%%%%%%%%%%%%%%%%%%%%%%%%%%%%%%%%
\end{document}
%%%%%%%%%%%%%%%%%%%%%%%%%%%%%%%%%%%%%%%%%%%%%%%%%%%%%%%%%%%%%%%%%%%%%%%%%



%%% Local Variables: 
%%% mode: latex
%%% TeX-master: "regexp-lec"
%%% End: 
