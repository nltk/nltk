\title{Introduction to Programming in Python (3)}
\author{Steve Renals \newline \mbox{ }s.renals@ed.ac.uk\mbox{ }}
\date{ICL --- 3 October 2005}

\begin{document}

\frame{\titlepage}



\mode<article>{\section[Outline]{ICL/Introduction to Python 3/2005-10-03}}
\mode<presentation>{
  \section[Outline]{}
}

\frame{\tableofcontents}

\section{NLTK}

\begin{frame}[fragile]
  \frametitle{NLTK: Python Natural Language ToolKit}

  \begin{itemize}
  \item<1-> NLTK is a set of Python modules which you can \texttt{import}
    into your programs, eg:
\begin{verbatim}
from nltk_lite.utilities import re_show
\end{verbatim}
  \item<2-> NLTK is distributed with several \emph{corpora} (singular:
    \emph{corpus}).  A corpus is  a body of text (or other language
    data, eg speech).
  \item<2-> Example corpora with NLTK:  \texttt{gutenberg}
    (works of literature from project Gutenberg), \texttt{treebank}
    (parsed text from the (part of) the Penn treebank), \texttt{brown}
    (the first  million word, PoS-tagged corpus, in 1961)
  \item<3-> Load a corpus (eg  \texttt{gutenberg}) using:
\begin{verbatim}
>>> from nltk_lite.corpora import gutenberg
>>> print gutenberg.items
['austen-emma', 'austen-persuasion', 'austen-sense', 'bible-kjv', 'blake-poems', 
\end{verbatim}
  \end{itemize}
\end{frame}

\begin{frame}[fragile]
  \frametitle{Simple corpus operations}

  \begin{itemize}
  \item<1-> Simple processing of a corpus includes \emph{tokenization}
    (splitting the text into word tokens), text normalization (eg by
    case), then many possible operations such as obtaining word
    statistics, \emph{tagging} and \emph{parsing} 
  \item<2-> Count the number of words in ``Macbeth''
{\small
\begin{verbatim}
from nltk_lite.corpora import gutenberg
nwords = 0

#iterate over all word tokens in Macbeth
for word in gutenberg.raw('shakespeare-macbeth'): 
    nwords += 1
print nwords                                     # 154873
\end{verbatim}}
  \item <3-> \texttt{gutenberg.raw(<textname>)} is an
    \emph{iterator}, which behaves like a sequence (eg a list) except
    it returns elements one at a time as requested
  \end{itemize}
\end{frame}

\begin{frame}[fragile]
  \frametitle{Richer corpora}

  \begin{itemize}
  \item<1-> The Gutenberg corpus is tokenized as a sequence of words, with
    no further structure.  
  \item<2-> The Brown corpus has sentences marked, and is stored as a list
    of sentences, where a sentence is a list of word tokens.  We can
    use the \texttt{extract} function to obtain individual sentences
{\small
\begin{verbatim}
from nltk_lite.corpora import brown
from nltk_lite.corpora import extract

firstSentence =  extract(0, brown.raw('a'))

# ['The', 'Fulton', 'County', 'Grand', 'Jury', 'said', 'Friday', ...]
\end{verbatim}}
  \item<3-> Part-of-speech tagged text can also be extracted:
{\small
\begin{verbatim}
taggedFirstSentence =  extract(0, brown.tagged('a'))

# [('The', 'at'), ('Fulton', 'np-tl'), ('County', 'nn-tl'), ('Grand', 'jj-tl'), ...]
\end{verbatim}}
  \end{itemize}
\end{frame}

\begin{frame}[fragile]
  \frametitle{Parsed text}

  Parsed text from the Penn treebank can also be accessed:
{\footnotesize
\begin{verbatim}
>>> from nltk_lite.corpora import treebank
>>> parsedSent = extract(0, treebank.parsed())
>>> print parsedSent
>>> print parsedSent
(S:
  (NP-SBJ:
    (NP: (NNP: 'Pierre') (NNP: 'Vinken'))
    (,: ',')
    (ADJP: (NP: (CD: '61') (NNS: 'years')) (JJ: 'old'))
    (,: ','))
  (VP:
    (MD: 'will')
    (VP:
      (VB: 'join')
      (NP: (DT: 'the') (NN: 'board'))
      (PP-CLR:
        (IN: 'as')
        (NP: (DT: 'a') (JJ: 'nonexecutive') (NN: 'director')))
      (NP-TMP: (NNP: 'Nov.') (CD: '29'))))
  (.: '.'))
\end{verbatim}}
\end{frame}


\begin{frame}[fragile]
  \frametitle{Count the frequency of each word in \emph{Macbeth}}
{\small
\begin{verbatim}
from nltk_lite.corpora import gutenberg

count = {}                      # initialize dictionary

for word in gutenberg.raw('shakespeare-macbeth'):    
    word = word.lower()            # normalize case
    if word not in count:          # previously unseen word?
        count[word] = 0            #   if so set count to 0
    count[word] += 1               # increment word count
\end{verbatim}}
\pause
We can inspect the dictionary:
{\small
\begin{verbatim}
print count['scotland']            # 12
print count['thane']               # 25
print count['blood']               # 24
print count['duncan']              # 10
\end{verbatim}}
\end{frame}

\begin{frame}[fragile]
  \frametitle{Sorting by frequency}

  We would like to sort the dictionary by frequency, but:


\begin{itemize}
  \item<1-> If we just sort the values (\texttt{count.values()}) we lose
    the link to the keys 
  \item<2-> \texttt{count.items()} returns a list of the pairs, but
    naively sorting that list doesn't do what we want:
\begin{verbatim}
wordfreq = count.items()   
wordfreq.sort()         # WRONG! - sorted by word!
\end{verbatim}
  \end{itemize}

  \uncover<3->{
  We will show five different ways of doing this right in Python...}
  
\end{frame}

\begin{frame}[fragile]
  \frametitle{Sorting by word frequency (1)}

  One way to do it:
\begin{verbatim}
wordfreq = count.items()
res = []
for wf in wordfreq:
    res.append((wf[1], wf[0]))

res.sort()
res.reverse()
print res[:10]
\end{verbatim}
  
\end{frame}

\begin{frame}[fragile]
  \frametitle{Sorting by word frequency (2)}

  slightly less clunky
\begin{verbatim}
wordfreq = count.items()
res = []

#for wf in wordfreq:
#    res.append((wf[1], wf[0]))

# explicitly assign to elements of a tuple
for (w, f)  in wordfreq:        
    res.append((f, w))

res.sort()
res.reverse()
\end{verbatim}
\end{frame}

\begin{frame}[fragile]
  \frametitle{Sorting by word frequency (3)}

  Could even use a list comprehension:
\begin{verbatim}
wordfreq = count.items()

#res = []
#for (w, f)  in wordfreq:        
#    res.append((f, w))

# use a list comprehension instead
res = [(f, w) for (w, f) in wordfreq]

res.sort()
res.reverse()
\end{verbatim}
\end{frame}

\begin{frame}[fragile]
  \frametitle{Sorting by word frequency (4)}

  \begin{itemize}
  \item<1->   The \texttt{sort} function uses a comparator function
    \texttt{cmp(x,y)} which returns negative if \verb+x<y+, zero if
    \verb+x==y+, positive if \verb+x>y+.
  \item<2-> You can define your own comparator which sorts on the secoond element of
  each item, and sorts with biggest first:
\begin{verbatim}
def mycmp(s1, s2):
   return cmp(s2[1], s1[1])
\end{verbatim}
\item<3-> Making sorting by word frequency straightforward:
\begin{verbatim}
wordfreq = count.items()
wordfreq.sort(mycmp)
print wordfreq[:10]
\end{verbatim}
\end{itemize}
\end{frame}

\begin{frame}[fragile]
  \frametitle{Sorting by word frequency (5)}

  Finally, can even use an anonymous comparator function:
\begin{verbatim}
wordfreq = count.items()
wordfreq.sort(lambda s1, s2: cmp(s2[1], s1[1]))
print wordfreq[:10]
\end{verbatim}
\end{frame}

\section{Regular Expressions}

\begin{frame}[fragile]
  \frametitle{Regular expressions in Python}

  \begin{itemize}
  \item<1->  The Python regular expression module \texttt{re} can be used
    for matching, substituting and searching within strings:
{\small
\begin{verbatim}
>>> import re
>>> from nltk_lite.utilities import re_show
>>> s = "introduction to computational linguistics"
>>> re_show ('i', s)
{i}ntroduct{i}on to computat{i}onal l{i}ngu{i}st{i}cs
>>> re_show('tion', s)
introduc{tion} to computa{tion}al linguistics
\end{verbatim}}
  \item<2-> We can perform a substitution with \texttt{re.sub}
{\small
\begin{verbatim}
t = re.sub('tion', 'XX', s)
# t = 'introducXX to computaXXal linguistics'
\end{verbatim}}
    Remember strings are immutable; \texttt{re.sub} returns a new string
  \end{itemize}
\end{frame}

\begin{frame}[fragile]
  \frametitle{Disjunction in regular expressions}

  We can look for one string or another
{\small
\begin{verbatim}
u = re.findall('(ti|c)', s)
# u= ['c', 'ti', 'c', 'ti', 'ti', 'c']
\end{verbatim}}

  \pause
  Or we can disjoin characters, eg \texttt{[aeiou]} matches any of
  \texttt{a}, \texttt{e}, \texttt{i}, \texttt{o} or \texttt{u}
  (vowels), and \verb+[^aeiou]+ matches anything that is not a
  vowel.  So we can match sequences finding non-vowels followed by
  vowels:
{\small
\begin{verbatim}
v  = re.findall('[aeiou][^aeiou]', s)
# v =['in', 'od', 'uc', 'on', 'o ', 'om', 'ut', 'at', 'on', 'al', 'in', 'is', 'ic']
\end{verbatim}}

  More on regular expressions next week.
\end{frame}

\section{Classes}

\begin{frame}[fragile]
  \frametitle{(A brief introduction to) Classes in Python}
  
  \begin{itemize}
  \item<1-> Classes are general data structures containing
    \begin{itemize}
    \item \emph{Attributes} --- Data: components, parts, properties, etc. 
    \item \emph{Methods} --- operations that can be performed on the data
    \end{itemize}
  \item<2-> The Python \texttt{class} system handles this
    \begin{itemize}
    \item Define a class using statement \texttt{class}
    \item Attributes are defined when first assigned - no need to
      declare in advanced
    \item All methods have a first parameter that corresponds to the
      object, named (by convention) \texttt{self}
    \item Special methods: \verb+__init__+ is called when the object
      is created;  \verb+__str__+ is called when a \texttt{print}
      statement is called on the object
    \end{itemize}
  \end{itemize}
\end{frame}

\begin{frame}[fragile]
  \frametitle{Example Person class}
{\small
\begin{verbatim}
class Person:
    def __init__(self, givenName, familyName):
        self.givenName = givenName
        self.familyName = familyName

    def fullName(self):
        return self.givenName + ' ' + self.familyName

    def __str__(self):
        return '<Person: givenName=' + self.givenName + 
        ' familyName=' + self.familyName
\end{verbatim}}
  
\end{frame}

\begin{frame}[fragile]
  \frametitle{Creating a Person object}
{\small
\begin{verbatim}
>>> from Person import *
>>> p = Person('Steve', 'Renals')
>>> print p
<Person: givenName=Steve familyName=Renals>
>>> p.fullName()
'Steve Renals'
\end{verbatim}
  \pause
\begin{verbatim}
>>> p.address=('Buccleuch Place')>
>> print p
<Person: givenName=Steve familyName=Renals>
>>> p.address  
'Buccleuch Place'
\end{verbatim}
  \pause
\begin{verbatim}
>>> p.age
Traceback (most recent call last):
  File "<stdin>", line 1, in ?
AttributeError: Person instance has no attribute 'age'
\end{verbatim}
}
\end{frame}

\begin{frame}[fragile]
  \frametitle{Inheritance}

  Classes can inherit from a \emph{superclass}:

\begin{verbatim}
class Student(Person):
    def __init__(self, givenName, familyName, num):
        Person.__init__(self, givenName, familyName)
        self.matricNumber = num

    def __str__(self):
        return '<Student:' + Person.__str__(self) + 
        ' matricNumber=' + str(self.matricNumber) + '>'
\end{verbatim}
  
\end{frame}

\begin{frame}[fragile]
  \frametitle{Creating a Student object}
\begin{verbatim}
>>> from Person import *
>>> s = Student('Steve', 'Renals', 123456)
>>> print s
<Student:<Person: givenName=Steve familyName=Renals> matricNumber=123456>
>>> s.matricNumber
123456
\end{verbatim}
\end{frame}

\section{Summary}

\begin{frame}
  \frametitle{Summary}

  \begin{itemize}
  \item The NLTK toolkit
  \item Accessing (raw / tagged / parsed) corpora from NLTK
  \item Five ways to sort words by corpus frequency
  \item Brief introduction to classes in Python
  \end{itemize}
\end{frame}


\end{document}
