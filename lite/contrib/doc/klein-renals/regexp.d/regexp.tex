%%%%%%%%%%%%%%%%%%%%%%%%%%%%%%%%%%%%%%%%
% Revised: Time-stamp: <2005-10-10 09:43:30 ewan>
% 
% $Log$
% Revision 1.1  2006/02/08 15:36:53  ehk
% First import of beamer.sty slides. Some files still need renaming.
%
% Revision 1.8  2005/10/10 14:29:09  ewan
% *** empty log message ***
%
% Revision 1.7  2005/10/10 12:43:33  ewan
% *** empty log message ***
%
% Revision 1.6  2005/10/10 11:37:45  ewan
% *** empty log message ***
%
% Revision 1.5  2005/10/10 11:31:34  ewan
% Added  more re_show examples
%
% Revision 1.4  2005/10/10 10:59:54  ewan
% *** empty log message ***
%
% Revision 1.3  2005/10/10 08:46:15  ewan
% Reorganized sequence of slides, added some sectioning, minor edits.
%
% Revision 1.2  2005/10/09 13:39:14  ewan
% *** empty log message ***
%
% Revision 1.1  2005/10/09 10:19:04  ewan
% First import
%
%
%
%%%%%%%%%%%%%%%%%%%%%%%%%%%%%%%%%%%%%%%%

% NB
% Add some diagrams of FSAs!!!
% add something on sub
% add regexp tokenizer
% add another example of using REs?
% 
% Named Groups example -- not a good place to introduce string formatting!

\title{Regular Expressions}
\author{Ewan Klein \newline \mbox{ }ewan@inf.ed.ac.uk\mbox{ }}
\date{ICL --- 10 October 2005}



%%%%%%%%%%%%%%%%%%%%%%%%%%%%%%%%%%%%%%%%%%%%%%%%%%%%%%%%%%%%
%%%%%%%%%%%%%%%%%%%%%%%%%%%%%%%%%%%%%%%%%
% Title:  defs
%
% Doc: /home/ewan/Teach/Intro/src/defs.tex
% Original Author:     Ewan Klein (ewan@zangfu)
% Created:             Tue Oct  8 1996
% Revised: Time-stamp: <2005-10-28 16:38:35 ewan>
% $Log$
% Revision 1.1  2006/03/09 09:13:21  ehk
% *** empty log message ***
%
% Revision 1.1  2005/11/03 10:42:49  ewan
% *** empty log message ***
%
% Revision 1.1  2004/09/28 07:28:46  ewan
% *** empty log message ***
%
% Revision 1.2  1996/10/11  16:01:58  ewan
% Installed in /projects/ltg/users/colin/Teach/Intro
%
% Revision 1.2  1996/10/09  03:41:25  ewan
% *** empty log message ***
%
%%%%%%%%%%%%%%%%%%%%%%%%%%%%%%%%%%%%%%%%
%\newcommand{\iupp}{\^\i}
%\newcommand{\nasa}{\diatop[\~|a]}
%\newcommand{\nase}{\diatop[\~|\niepsilon]}
%\newcommand{\naso}{\diatop[\~|\openo]}
%\newcommand{\slt}{\diatop[/|t]}
%\newcommand{\hand}{\ding{43}}
%\newcommand{\hand}{$\Rightarrow$}
\newcommand{\bad}{\sqz{*}}
\newcommand{\cat}[1]{\textit{#1\/}} %linguistic categories
\newcommand{\con}[1]{\textbf{#1\/}} %constituents etc in examples
\newcommand{\dash}{$\underline{\hspace{1em}}$}
\newcommand{\Easy}{\ling{easy}-Constructions}
\newcommand{\Feat}[1]{\textsc{#1}}
\newcommand{\feat}[1]{\textsc{#1}}
\newcommand{\featval}[2]{\textsc{#1}\ \textsc{#2}}
\newcommand{\fun}[1]{\mbox{\textsl{#1}}}
\newcommand{\mygloss}[1]{`{#1}'}
\newcommand{\group}[1]{$\underbrace{\mbox{#1}}$}
\newcommand{\hpsg}{\textsc{hpsg}}
\newcommand{\latin}[1]{\textit{#1\/}} 
%\newcommand{\lextab}{\begin{tabular}{l l @{:\extracolsep{.5in}} l}}
\newcommand{\lextab}{\begin{tabular}{l l l}}
\newcommand{\lingloss}[2]{\ling{#1} \gloss{#2}}
\newcommand{\ling}[1]{\mbox{\textit{#1}}}
\newcommand{\quot}[1]{\textit{#1\/}}
\newcommand{\scare}[1]{`#1'}
\newcommand{\str}[1]{$ #1 $}
\newcommand{\sxref}[2]{(\ref{ex:#1}\ref{ex:#1#2})} %(3a) type example references
\newcommand{\term}[1]{\textsc{#1}}
\newcommand{\type}[1]{\textit{#1\/}}
\newcommand{\val}[1]{\textsc{#1}}
\newcommand{\whIs}{\ling{wh}-Interrogatives}
\newcommand{\whs}{\ling{wh} expressions}
\newcommand{\Whs}{\ling{Wh} expressions}
\newcommand{\xref}[1]{(\ref{ex:#1})} %(3) type example references
%
%\newcommand{\AP}{\textsc{ap\xspace}}
%\newcommand{\NP}{\textsc{np\xspace}}
%\newcommand{\PP}{\textsc{pp\xspace}}
%\newcommand{\VP}{\textsc{vp\xspace}}
%%%%
% changed \textsc to \textsf, 2002-12-02
%%%
\newcommand{\AP}{\textsf{AP}}
\newcommand{\NP}{\textsf{NP}}
\newcommand{\PP}{\textsf{PP}}
\newcommand{\VP}{\textsf{VP}}
\newcommand{\Adj}{\textsf{Adj}}
\newcommand{\Adv}{\textsf{Adv}}
\newcommand{\Aux}{\textsf{Aux}}
\newcommand{\Nom}{\textsf{Nom}}
\newcommand{\Conj}{\textsf{Conj}}
\newcommand{\Pro}{\textsf{Pro}}
\newcommand{\Card}{\textsf{Card}}
\newcommand{\Quant}{\textsf{Quant}}
\newcommand{\Ord}{\textsf{Ord}}
\newcommand{\AdvP}{\textsf{AdvP}}
\newcommand{\A}{\textsf{A}}
\newcommand{\Det}{\textsf{Det}}
\newcommand{\N}{\textsf{N}}
\newcommand{\PropN}{\textsf{PropN}}
\newcommand{\Prep}{\textsf{P}}
\newcommand{\V}{\textsf{V}}
\newcommand{\Se}{\textsf{S}}
\newcommand{\Ntbar}{$\overline{\overline{\overline{\mbox{N}}}}$}
\newcommand{\Ndbar}{$\overline{\overline{\mbox{N}}}$}
\newcommand{\Nbar}{$\overline{\mbox{N}}$}
\newcommand{\Atbar}{$\overline{\overline{\overline{\mbox{A}}}}$}
\newcommand{\Adbar}{$\overline{\overline{\mbox{A}}}$}
\newcommand{\Abar}{$\overline{\mbox{A}}$}
\newcommand{\Ptbar}{$\overline{\overline{\overline{\mbox{P}}}}$}
\newcommand{\Pdbar}{$\overline{\overline{\mbox{P}}}$}
\newcommand{\Pbar}{$\overline{\mbox{P}}$}
\newcommand{\Vtbar}{$\overline{\overline{\overline{\mbox{V}}}}$}
\newcommand{\Vdbar}{$\overline{\overline{\mbox{V}}}$}
\newcommand{\Vbar}{$\overline{\mbox{V}}$}
\newcommand{\Xtbar}{$\overline{\overline{\overline{\mbox{X}}}}$}
\newcommand{\Xdbar}{$\overline{\overline{\mbox{X}}}$}
\newcommand{\Xbar}{$\overline{\mbox{X}}$}
\newcommand{\PS}{\textsc{ps}}
\newcommand{\PSG}{\textsc{psg}}
\newcommand{\Vintrans}{V$_{intrans}$}
\newcommand{\Vtrans}{V$_{trans}$}
\newcommand{\Vsent}{V$_{sent}$}
\newcommand{\XP}{\textsf{XP}}
%
%\newcommand{\W}[1]{\makebox[.75in]{#1}}
%
\newcommand{\cf}{\textsl{c\,f\xspace}}
\newcommand{\ie}{\textsl{i\,e\xspace}}
\newcommand{\eg}{\textsl{e\,g\xspace}}
%
\newcommand{\Book}[1]{\textit{#1\/}}
%
\newenvironment{gram}{%
    \begin{verse}}%
    {\end{verse}}

% Lists
\newenvironment{shortlist}{%
    \begin{itemize}%
        \setlength{\parsep}{0pt}
        \setlength{\itemsep}{0pt}}{%
    \end{itemize}}

\newenvironment{enum}{%
    \begin{enumerate}%
        \setlength{\parsep}{0pt}
        \setlength{\itemsep}{0pt}}{%
    \end{enumerate}}

\newenvironment{desc}{%
    \begin{description}%
        \setlength{\parsep}{0pt}
        \setlength{\itemsep}{0pt}}{%
    \end{description}}

\usepackage{fancybox}
\usepackage{color}
\usepackage{amsmath}
\usepackage{graphicx}
\usepackage{alltt}
\usepackage{url}


% Shading
\definecolor{light}{gray}{.80}
\newcommand{\Hilite}[1]{\colorbox{yellow}{#1}}
\newcommand{\Shade}[1]{\colorbox{light}{#1}}
%\newcommand{\Hilite}[1]{\colorbox{white}{#1}}
%\newcommand{\Shade}[1]{\colorbox{white}{#1}}


\newcommand{\Comment}[1]{\textcolor{blue}{#1}}
\newcommand{\Em}[1]{\textcolor{red}{#1}}
\newcommand{\Dim}[1]{\textcolor{gray}{#1}}

\setlength{\parskip}{0in}
\setlength{\parindent}{0in}

% Headers and Footers
% \newpagestyle{mystyle}%
% {} %header
% {\sl Ewan Klein, ICL Week 3, Lecture 1 \hfil \thepage \hfil October 4, 2004} %footer
%{\hfil File \jobname.tex; Printed \today\hfil}

\begin{document}

\frame{\titlepage}



\mode<article>{\section[Outline]{ICL/Regular Expressions/2005-10-10}}
\mode<presentation>{
  \section[Outline]{}
}

\frame{\tableofcontents}

% \begin{frame}


%  \begin{center}
%     {\Huge \textbf{Introduction to Computational Linguistics}

% Regular Expressions

% }
% \bigskip
% Week 3, Lecture 1\\
% \today
% \end{center}

% \end{frame}

\section{Overview of REs}

\subsection{Introduction}

\begin{frame}[fragile]
%   \frametitle{Today}

%   \begin{itemize}
%     \item Overview of Regular Expressions (REs)
%     \item Examples of Using REs
%   \end{itemize}

Goals: 
\begin{itemize}
  \item a basic idea of the formal background for REs
  \item an ability to write small Python programs that do 
useful things with REs
\end{itemize}

\end{frame}

\begin{frame}[fragile]
  \frametitle{Motivation}

  \begin{description}
  \item [Task:] To search for strings using (partially
      specified) \Em{patterns}
  \item [Why:]\hfill\\ 
    \begin{itemize}
    \item validate data fields (dates, email addresses, URLs)
    \item filter text (spam, disallowed web sites)
    \item identify particular strings in a text (token boundaries for tokenization)
    \item convert the output of one processing component into the format
          required for a second component  (\verb!rabbit_NN!
          $\rightarrow$ \verb!<word pos=''NN''>rabbit</word>!) 
    \end{itemize}

  \end{description}
\end{frame}


\begin{frame}[fragile]
  \frametitle{The Basic Idea}

  \begin{itemize}
    \item \Em{Regular expressions} form a language for expressing patterns.
    \item The language can be stated as a formal algebra.
    \item Recognizers for RE can be efficiently implemented.
    \item `Regular expression' also a term for a pattern that is
      constructed using the language.
    \item Every pattern specifies a \Em{set of strings}.
    \item Text string: a sequence of letters, numerals, spaces, tabs,
      punctuation, \ldots
  \end{itemize}
\end{frame}

\begin{frame}[fragile]
  \frametitle{Initital Examples}

  \begin{tabular}{lll} \hline
              & \Em{Pattern}   & \Em{Matches} \\ \hline
concatenation & \textbf{abc} & abc \\ \hline
disjunction   & \textbf{a} $\mid$ \textbf{b} & a, b \\
              & (\textbf{a} $\mid$ \textbf{bb}) \textbf{d} & ad, bbd\\ \hline
closure       & \textbf{a}* & $\epsilon$, a, aa, aaa, aaaa, \ldots\\ 
              & \textbf{c}(\textbf{a} $\mid$ \textbf{bb})* & c, ca, cbb,
                                                             cabb, caa, cbbbb,
                                                             \ldots\\ \hline
             

  \end{tabular}
\end{frame}

\begin{frame}[fragile]
\frametitle{Two Types of RE}

\begin{description}
  \item [Literals] Every normal text character is an RE, and denotes
    itself.
  \item [Metacharacters] Special characters which allow you to specify
    various sets of strings.
\end{description}
Example---Kleene star (*)
\begin{itemize}
  \item \textbf{a} denotes \textit{a}
  \item \textbf{a*} denotes $\epsilon$ (empty string), \textit{a},
    \textit{aa}, \textit{aaa}, \ldots
  
\end{itemize}
\end{frame}

\subsection{Formal Background to REs}

\begin{frame}[fragile]
  \frametitle{Preliminaries: Operations on Sets of Strings}

    Let $\Sigma$ be a finite set of symbols and let $\Sigma^*$ be the set
    of all strings (including the empty string) over $\Sigma$. Suppose
    $L, L_{1}, L_{2}$ are subsets of $\Sigma^*$.

  \begin{itemize}
  
    \item The \textit{union} of $L_{1}, L_{2}$,
      denoted $L_{1}\cup L_{2}$,  is the set of
      strings $x$ such that
      $x \in L_{1}$ or $x \in L_{2}$.
   \item The \textit{concatenation} of $L_{1}, L_{2}$,
      denoted $L_{1}L_{2}$,  is the set of
      strings $xy$ such that
      $x \in L_{1}$ and $y \in L_{2}$.
    \item The \textit{Kleene closure} of $L$, denoted $L^*$, is the set of
      strings constructed by concatenating any number of strings from
      $L$. $L^*$ contains $\epsilon$, the empty string.
    \item The \textit{positive closure} of $L$, denoted $L^+$, is the same as
      $L^*$ but without $\epsilon$.

  \end{itemize}
\end{frame}

\begin{frame}[fragile]
  \frametitle{Examples}

Let $L_1 = \{a, b\} \text{ and } L_2 = \{c\}$. Then
\begin{itemize}
  \item $L_{1}\cup L_{2} = \{a, b, c\}$
  \item $L_{1}L_{2} = \{ac, bc\}$
  \item $\{a, b\}^* = \{\epsilon, a, b, aa, bb, ab, ba, \ldots\}$
  \item $\{a, b\}^+ = \{a, b, aa, bb, ab, ba, \ldots\}$
\end{itemize}
\end{frame}

\begin{frame}[fragile]
  \frametitle{Formal Definition of Regular Expressions}

Regular expressions over a finite alphabet $\Sigma$:
\begin{enumerate}
%\item $\emptyset$  is a regular expression and denotes the empty set.
\item $\epsilon$ is a regular expression and denotes the set
  $\{\epsilon\}$.
\item For each $a$ in $\Sigma$, \boldmath{a} is a regular expression and
  denotes the set $\{a\}$.
\item If $r$ and $s$ are regular expressions denoting the sets $R$ and
  $S$ respectively, then 
  \begin{itemize}
    \item $(r \mid s)$ is a regular expression denoting $R \cup S$.
    \item $(rs)$ is a regular expression denoting $RS$.
    \item $(r^*)$ is a regular expression denoting $R^*$.
  \end{itemize}

\end{enumerate}
\end{frame}

\begin{frame}[fragile]
  \frametitle{Recognizers}

  \begin{itemize}
  \item A \Em{recognizer} for a language is a program that takes as input a
string $x$ and answers ``yes'' if $x$ is a sentence of the language and
``no'' otherwise.
\item We can think of this program as a machine which only emits two possible
responses to its input.
  \end{itemize}



\end{frame}
\begin{frame}[fragile]
 \frametitle{Finite State Automata\hfill \includegraphics[scale=.15]{../images/skleene}}

\begin{itemize}
\item A Finite State Automaton (FSA) is an \Em{abstract finite machine}.

\item Regular expressions can be viewed as a way to describe a Finite
  State Automaton (FSA)
  
\item Kleene's theorem (1956): FSA and RE describe the same languages:
  \begin{itemize}
    \item Any regular expression can be implemented as an FSA.
    \item Any FSA can be described by a regular expression.
  \end{itemize}
  
\item Regular languages are those that can be \Em{recognized} by FSAs (or
  characterized by a regular expression).
  \end{itemize}

\end{frame}

\subsection{Extensions of Basic REs}

\begin{frame}[fragile]
  \frametitle{Metacharacters}

NB. Different sets of metacharacters and notations used by different `host languages' (e.g., Unix
grep, GNU emacs, Perl, Java, Python,  etc.). Cf. Jurafsky \& Martin, Appendix A)

\begin{description}
  \item [Disjunction:] \textbf{$\mid$}
  \item [Wild card:] \textbf{}.
  \item [Optionality:] \textbf{}?
  \item [Quantification:] \textbf{*} and \textbf{+}
  \item [Choice:] \textbf{[Mm]} \textbf{[0123456789]}
  \item [Ranges:] \textbf{[a-z]} \textbf{[0-9]}
  \item [Negation:] \textbf{[$^\wedge$Mm]} (only when `$^\wedge$' occurs immediately
    after `[')

\end{description}
\end{frame}
\begin{frame}[fragile]
  \frametitle{Special Backslash Sequences}

  \begin{itemize}
  \item Standard escape sequences\\
  \begin{tabular}{l}
\verb!\t!: tab \\
\verb!\n!: newline \\
  \end{tabular}

  \item Abbreviatory forms\\
  \begin{tabular}{ll}
\verb!\d!: digit (i.e., numeral) & \verb!\D!: non-digit \\
\verb!\s!: `whitespace' ([ \verb!\t\n!]) & \verb!\S!: non-whitespace \\
\verb!\w!: `alphanumeric' ([a-zA-Z0-9]) & \verb!\W!: non-alphanumeric \\
  \end{tabular}

   \item \verb!\! is a general escape character; e.g., \verb!\.! is
         not a wildcard, but
         matches a period, \verb!.!
   \item If you want to use \verb!\! in a string, it has to be
         escaped: \verb!\\!

  \end{itemize}





\end{frame}

\begin{frame}[fragile]
  \frametitle{Anchors}

(Also: zero-width characters)

\begin{itemize}
  \item Anchors don't match strings in the text, instead
  \item they match \Em{positions} in the text.\\

 \begin{tabular}{lp{3in}}
\verb!^!: & matches beginning of line (or text)\\
\verb!$!: & matches end of line (or text) \\%$
\verb!\b!: &  matches word boundary (i.e., a location with \verb!\w! on one side
but not the other)  \\
  \end{tabular}
\end{itemize}
 
\end{frame}

\section{REs in Python}

\subsection{Examples with \texttt{re\_show}}

\begin{frame}[fragile]
\frametitle{Wildcard}

\begin{verbatim}
>>> from nltk_lite.utilities import re_show
>>> s = '''BP has agreed to sell
... it's petrochemicals unit for $5.1bn.'''
>>> re_show('...', s)
{BP }{has}{ ag}{ree}{d t}{o s}{ell}
{it'}{s p}{etr}{och}{emi}{cal}{s u}{nit}{ fo}{r $}{5.1}{bn.}
\end{verbatim}
\begin{verbatim}
>>> re_show('.a..', s)
BP {has }agreed to sell
it's petrochemi{cals} unit for $5.1bn.
\end{verbatim}

\end{frame}



\begin{frame}[fragile]
\frametitle{Wildcards with Quantifiers}


\begin{verbatim}
>>> re_show('s.*l', s)
BP ha{s agreed to sell}
it'{s petrochemical}s unit for $5.1bn.
\end{verbatim}

\begin{verbatim}
>>> re_show('B.*P', s)
{BP} has agreed to sell
it's petrochemicals unit for $5.1bn.
\end{verbatim}

\begin{verbatim}
>>> re_show('B.+P', s)
BP has agreed to sell
it's petrochemicals unit for $5.1bn.
\end{verbatim}
\end{frame}

\begin{frame}[fragile]
\frametitle{Disjunction}


\begin{verbatim}
>>> re_show('has|it', s)
BP {has} agreed to sell
{it}'s petrochemicals un{it} for $5.1bn.
\end{verbatim}

\begin{verbatim}
>>> re_show('has | it', s)
BP {has }agreed to sell
it's petrochemicals unit for $5.1bn.
\end{verbatim}

\begin{verbatim}
>>> re_show('(e|l)+', s)
BP has agr{ee}d to s{ell}
it's p{e}troch{e}mica{l}s unit for $5.1bn.
\end{verbatim}
\end{frame}


\begin{frame}[fragile]
\frametitle{Zero Width Characters}


\begin{verbatim}
>>> re_show('l', s)
BP has agreed to se{l}{l}
it's petrochemica{l}s unit for $5.1bn.
\end{verbatim}

\begin{verbatim}
>>> re_show('l$', s)
BP has agreed to sel{l}
it's petrochemicals unit for $5.1bn.
\end{verbatim}

\begin{verbatim}
>>> re_show('i', s)
BP has agreed to sell
{i}t's petrochem{i}cals un{i}t for $5.1bn.
\end{verbatim}

\begin{verbatim}
>>> re_show('^i', s)
BP has agreed to sell
{i}t's petrochemicals unit for $5.1bn.
\end{verbatim}

\end{frame}


\begin{frame}[fragile]
\frametitle{Escaping Special Characters}


\begin{verbatim}
>>> re_show('.', s)
{B}{P}{ }{h}{a}{s}{ }{a}{g}{r}{e}{e}{d}...
\end{verbatim}

\begin{verbatim}
>>> re_show('\.', s)
BP has agreed to sell
it's petrochemicals unit for $5{.}1bn{.}
\end{verbatim}

\begin{verbatim}
>>> re_show('$', s)
BP has agreed to sell{}
it's petrochemicals unit for $5.1bn.{}
\end{verbatim}

\begin{verbatim}
>>> re_show('\$', s)
BP has agreed to sell
it's petrochemicals unit for {$}5.1bn.
\end{verbatim}
\end{frame}

\begin{frame}[fragile]
\frametitle{Metacharacters and Negated Ranges}


\begin{verbatim}
>>> re_show('\w',s)
{B}{P} {h}{a}{s} {a}{g}{r}{e}{e}{d} ...
\end{verbatim}

\begin{verbatim}
>>> re_show('\d',s)
BP has agreed to sell
it's petrochemicals unit for ${5}.{1}bn.
\end{verbatim}

\begin{verbatim}
>>> re_show('[^a-z\s]',s)
{B}{P} has agreed to sell
it{'}s petrochemicals unit for {$}{5}{.}{1}bn{.}
\end{verbatim}

\begin{verbatim}
>>> re_show('[^\w]',s)
BP{ }has{ }agreed{ }to{ }sell{
}it{'}s{ }petrochemicals{ }unit{ }for{ }{$}5{.}1bn{.}
\end{verbatim}
\end{frame}

\subsection{Match objects in Python}

\begin{frame}[fragile]
\frametitle{Using REs in Python, 1}

\begin{itemize}
\item Usually best to compile the RE into a PatternObject; more
  efficient, and it can be re-used.
\begin{verbatim}
>>> import re
>>> str = 'do you say hello or hullo?'
>>> helloRE = re.compile('h[eu]llo')
\end{verbatim}
\item The resulting PatternObject has a number of methods:
\end{itemize}
\begin{description}
  \item [findall(s):] returns a list of \Em{all} matches of pattern in string \texttt{s}
  \item [search(s):] searches for \Em{leftmost} occurrence of pattern in string \texttt{s}
  \item [match(s):] tries to match pattern at the \Em{beginning} of string \texttt{s}
\end{description}

\end{frame}

\begin{frame}[fragile]
\frametitle{Using REs in Python, 2}

\begin{itemize}
\item The PatternObject method\texttt{findall} returns a \Em{list}:
\begin{verbatim}
>>> helloRE.findall(str)
['hello', 'hullo']
\end{verbatim}
\item  The PatternObject method \texttt{search} (and \texttt{match)} returns a MatchObject or None.
\item A MatchObject has a variety of methods, but is not a string.
\begin{verbatim}
>>> m = helloRE.search(str)
>>> m
<_sre.SRE_Match object at 0x47b138>
>>> m.group() # return matched substring (sort of!)
'hello'
>>> m.end() # index of end of target
16
\end{verbatim}

\end{itemize}
\end{frame}

%NB Say something about what LOOK.search() means

% \begin{frame}[fragile]
%   \frametitle{I am looking for\ldots}

% \begin{verbatim}
% #!/usr/bin/python2.2

% import re
% from nltk.corpus import *

% LOOK  = re.compile('I am looking for')

% for item in twenty_newsgroups.items('misc.forsale'):
%     for line in twenty_newsgroups.readlines(item):
%         if LOOK.search(line):
%             print line
% \end{verbatim}
% \end{frame}


% \begin{frame}[fragile]
%   \frametitle{}
% \begin{verbatim}
%         I am looking for a round trip ...

%    I am looking for a good used window ...

%  I am looking for a large futon and frame.

%   I am looking for the Coleco Tablehockey ...

% I am looking for a math coprocessor for a 286-16mhz.

% As it says in the subject, I am looking for ...
% \end{verbatim}
% \end{frame}

% % where does lstrip() come from? I.e atring methods....
% \begin{frame}[fragile]
%   \frametitle{Prettifying the output}
% \begin{alltt}

% for item in twenty_newsgroups.items('misc.forsale'):
%     for line in twenty_newsgroups.readlines(item):
%         if LOOK.search(line):
%             print line\Hilite{.lstrip(),}
% \end{alltt}
% \begin{verbatim}
% I am looking for a round trip ...
% I am looking for a good used window ...
% I am looking for a large futon and frame.
% I am looking for the Coleco Tablehockey ...
% I am looking for a math coprocessor for a 286-16mhz.
% As it says in the subject, I am looking for ...
% \end{verbatim}
% \end{frame}

% \begin{frame}[fragile]
%   \frametitle{Disjunction}

% \begin{alltt}
% LOOK  = re.compile(``I\Hilite{('| a)}m looking for'')   
% \end{alltt}
% \begin{verbatim}
% I'm looking for $500 or best offer, but act fast ...
% I am looking for a round trip Madison/Chicago ...
% I'm looking for a Sharp 6220 or TI Travelmate 2000 ...
% I am looking for a good used window air conditioner ...
% I am looking for a large futon and frame.
% I'm looking for a used/inexpensive audio ...
% I'm looking for about 25-30 dollars but
% \end{verbatim}
% \end{frame}

% \begin{frame}[fragile]
%   \frametitle{Multiple Tests}

% \begin{alltt}
% MONEY  = re.compile('\verb!\$\d+|!dollars')

% for item in twenty_newsgroups.items('misc.forsale'):
%   for line in twenty_newsgroups.readlines(item):
%     if LOOK.search(line) \Hilite{and not MONEY.search(line)}:
%       print line.lstrip(), 
% \end{alltt}
% \end{frame}

% \begin{frame}[fragile]
%   \frametitle{Optionality}

% \begin{alltt}
% MON = re.compile('\Hilite{S?VGA}')

% for item in twenty_newsgroups.items('misc.forsale'):
%     for line in twenty_newsgroups.readlines(item):
%         if MON.search(line):
%             print line.lstrip(),
% \end{alltt}
% \begin{verbatim}
% SVGA monitor that syncs from 15-38khz
% Subject: Diamond Stealth 24 24bit SVGA for sale
% An AAMAZING  1024x768 .28 dot pitch SVGA monitor
% Local Bus 1MB SVGA Video Card,
% 14" SVGA Monitor (.28dpi)
% VGA Monochrome 64 Grey Scale
% BUILT-IN Paradise SVGA controller with 1 meg of RAM 
% \end{verbatim}
% \end{frame}

% \begin{frame}[fragile]
%   \frametitle{Repetition}
% \begin{alltt}
% \Hilite{MON = re.compile('.?VGA.*[Mm]onitors?')}    

% SVGA monitor that syncs from 15-38khz
% An AAMAZING  1024x768 .28 dot pitch SVGA monitor
% 14" SVGA Monitor (.28dpi)
% SVGA card/w color Tatung VGA Monitor
% 386-40 with VGA \Shade{Color} Monitor, dual floppy
% For sale KFC SVGA Monitor 1024X768 .28DP Non-interlaced
% Subject: EGA/VGA Monitor&Card wanted
% I am looking for a decent EGA or VGA monitor/card
% Samsung VGA monitor
% multisynching or straight VGA \Shade{(cheap)} monitor.
% KFC SVGA Monitor 1024X768 .28DP Non-interlaced 14" 
% \end{alltt}

% \end{frame}

% % give example of embedded From:
% \begin{frame}[fragile]
%   \frametitle{Anchors}

% \begin{alltt}
% FROM = re.compile('\Hilite{^}From:.*@\verb!\w!*com')

% for item in twenty_newsgroups.items('misc.forsale'):
%     for line in twenty_newsgroups.readlines(item):
%         if FROM.search(line):
%             print line,    
% \end{alltt}
% \begin{alltt}
% From: creol@net\Shade{com}.com
% From: bigjoe@netcom.com (g perry)
% From: jonathan@\Shade{com}p.lancs.ac.uk (Mr J J Trevor)
% From: marc@comp.lancs.ac.uk (Marc Goldman)
% From: jonathan@comp.lancs.ac.uk (Mr J J Trevor)
% From: gregg@netcom.com (gregg weber)
% From: afhetzel@netcom.com (A.F. Hetzel)
% \end{alltt}
% \end{frame}

% \begin{frame}[fragile]
%   \frametitle{Scoping}

% \begin{alltt}
% FROM = re.compile('^From:.*@(\verb!\!w+\verb!\!.)+com')
% \end{alltt}
% one or more word characters followed by a period --- occurring one or more
% times
% \begin{alltt}
% From: jvinson@xsoft.xerox.\Shade{com} (Jeffrey A Vinson)
% From: pat@wrs.\Shade{com} (Patrick Boylan)
% From: creol@netcom.\Shade{com}
% From: gregh@hprnd.rose.hp.\Shade{com} (Greg Holdren)
% From: srfergu@rufus.erenj.com (Scott Ferguson)
% From: pfc@jungle.genrad.com (Paul F. Cappucci)
% From: pat@wrs.com (Patrick Boylan)
% From: srscnslt@telesciences.com (SRS Consultant)
% From: isifisher@aol.com
% \end{alltt}
% \end{frame}

% \begin{frame}[fragile]
%   \frametitle{Greedy Matching}

% \begin{alltt}
% FROM = re.compile('^From:.*@.*com\verb!\!s')
% \end{alltt}
% Match as many characters as possible between `@' and the string `com\verb!\!s'

% Equivalent to previous version for this data.
% \end{frame}

% give example of numeric groups

% do the string modulus stuff in a separate example?
\begin{frame}[fragile]
\frametitle{Groups}

\begin{itemize}
\item Groups in regular expressions are captured using parentheses.
\begin{verbatim}
>>> import re
>>> str = 'do you say hello or hullo?'
>>> reGRP = re.compile('(d.)(.*)(e..)')
>>> m = reGRP.search(str)
>>> m
<_sre.SRE_Match object at 0x64390>
>>> m.groups()
('do', ' you say h', 'ell')
\end{verbatim}
\end{itemize}

\end{frame}

\begin{frame}[fragile]
\frametitle{Named Groups}

\begin{itemize}
\item Name groups captured using \verb!(?P<name>)!:
\end{itemize}
{\small
\begin{verbatim}
FROM = re.compile("""
     ^From:          # Anchor to start of line
     \s*             # maybe some spaces
     (?P<user>\w+)   # 'user': group of word characters 
     @               
     (?P<domain>     # the 'domain':
     \S+)            # some non-space characters
     \s              # finally, a space character
     """,re.VERBOSE)
\end{verbatim}
}
\end{frame}

\begin{frame}[fragile]
\frametitle{Named Groups (cont.)}
{\small
\begin{verbatim}
from nltk_lite.corpus import twenty_newsgroups

for item in twenty_newsgroups.items('misc.forsale'):
    text = twenty_newsgroups.read(item)
    m = FROM.search(text)
    if m:
      print '%s is at %s' % \
      (m.group('user'), m.group('domain'))
\end{verbatim}
}
\begin{verbatim}
kedz is at bigwpi.WPI.EDU
myoakam is at cis.ohio-state.edu
gt1706a is at prism.gatech.EDU
jvinson is at xsoft.xerox.com
hungjenc is at usc.edu
thouchin is at cs.umr.edu
kssimon is at silver.ucs.indiana.edu
\end{verbatim}
\end{frame}

\section{Admin}



\begin{frame}[fragile]
\frametitle{Reading}



\begin{itemize}
  \item {\large Jurafsky \& Martin, Chap 2}
  \item {\large NLTK Lite Tutorial: Regular Expressions}
available from 
\url{http://nltk.sourceforge.net/lite/doc/en/regexps.html}
\end{itemize}

%NB. You're not expected to know about details of Finite State Automata for exams

\end{frame}

\begin{frame}[fragile]
\frametitle{Admin Stuff}



\begin{itemize}
\item The main exam for this course will be in May, but there will be
  an additional exam for visiting students at the end of this
  semester.
  \item ICL Web Page is at: \\
\url{http://www.inf.ed.ac.uk/teaching/courses/icl/}
\end{itemize}

%NB. You're not expected to know about details of Finite State Automata for exams

\end{frame}





%%%%%%%%%%%%%%%%%%%%%%%%%%%%%%%%%%%%%%%%%%%%%%%%%%%%%%%%%%%%%%%%%%%%%%%%%
\end{document}
%%%%%%%%%%%%%%%%%%%%%%%%%%%%%%%%%%%%%%%%%%%%%%%%%%%%%%%%%%%%%%%%%%%%%%%%%



%%% Local Variables: 
%%% mode: latex
%%% TeX-master: "regexp-lec"
%%% End: 
