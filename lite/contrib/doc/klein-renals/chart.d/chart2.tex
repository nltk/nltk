%%%%%%%%%%%%%%%%%%%%%%%%%%%%%%%%%%%%%%%%
% Revised: Time-stamp: <2005-11-10 16:30:21 ewan>
% 
% $Log$
% Revision 1.1  2006/02/08 15:36:53  ehk
% First import of beamer.sty slides. Some files still need renaming.
%
% Revision 1.4  2005/11/10 17:02:02  ewan
% *** empty log message ***
%
% Revision 1.3  2005/11/10 13:49:59  ewan
% *** empty log message ***
%
% Revision 1.2  2005/11/10 13:45:40  ewan
% *** empty log message ***
%
% Revision 1.1  2005/11/10 13:17:27  ewan
% *** empty log message ***
%
%
%%%%%%%%%%%%%%%%%%%%%%%%%%%%%%%%%%%%%%%%
\title{Chart Parsing, Part 2}
\author{Ewan Klein \newline \mbox{ }ewan@inf.ed.ac.uk\mbox{ }}
\date{ICL --- 10 November 2005}



\newcommand{\Rule}{\rule{\textwidth}{1pt}}
\newcommand{\SqB}[1]{[#1]}

\newcommand{\bigdot}{\mbox{\begin{scriptsize}{\ensuremath{\bullet}}\end{scriptsize}}}

\newcommand{\chdot}{\bigdot}

%%%%%%%%%%%%%%%%%%%%%%%%%%%%%%%%%%%%%%%%
% Title:  defs
%
% Doc: /home/ewan/Teach/Intro/src/defs.tex
% Original Author:     Ewan Klein (ewan@zangfu)
% Created:             Tue Oct  8 1996
% Revised: Time-stamp: <2005-10-28 16:38:35 ewan>
% $Log$
% Revision 1.1  2006/03/09 09:13:21  ehk
% *** empty log message ***
%
% Revision 1.1  2005/11/03 10:42:49  ewan
% *** empty log message ***
%
% Revision 1.1  2004/09/28 07:28:46  ewan
% *** empty log message ***
%
% Revision 1.2  1996/10/11  16:01:58  ewan
% Installed in /projects/ltg/users/colin/Teach/Intro
%
% Revision 1.2  1996/10/09  03:41:25  ewan
% *** empty log message ***
%
%%%%%%%%%%%%%%%%%%%%%%%%%%%%%%%%%%%%%%%%
%\newcommand{\iupp}{\^\i}
%\newcommand{\nasa}{\diatop[\~|a]}
%\newcommand{\nase}{\diatop[\~|\niepsilon]}
%\newcommand{\naso}{\diatop[\~|\openo]}
%\newcommand{\slt}{\diatop[/|t]}
%\newcommand{\hand}{\ding{43}}
%\newcommand{\hand}{$\Rightarrow$}
\newcommand{\bad}{\sqz{*}}
\newcommand{\cat}[1]{\textit{#1\/}} %linguistic categories
\newcommand{\con}[1]{\textbf{#1\/}} %constituents etc in examples
\newcommand{\dash}{$\underline{\hspace{1em}}$}
\newcommand{\Easy}{\ling{easy}-Constructions}
\newcommand{\Feat}[1]{\textsc{#1}}
\newcommand{\feat}[1]{\textsc{#1}}
\newcommand{\featval}[2]{\textsc{#1}\ \textsc{#2}}
\newcommand{\fun}[1]{\mbox{\textsl{#1}}}
\newcommand{\mygloss}[1]{`{#1}'}
\newcommand{\group}[1]{$\underbrace{\mbox{#1}}$}
\newcommand{\hpsg}{\textsc{hpsg}}
\newcommand{\latin}[1]{\textit{#1\/}} 
%\newcommand{\lextab}{\begin{tabular}{l l @{:\extracolsep{.5in}} l}}
\newcommand{\lextab}{\begin{tabular}{l l l}}
\newcommand{\lingloss}[2]{\ling{#1} \gloss{#2}}
\newcommand{\ling}[1]{\mbox{\textit{#1}}}
\newcommand{\quot}[1]{\textit{#1\/}}
\newcommand{\scare}[1]{`#1'}
\newcommand{\str}[1]{$ #1 $}
\newcommand{\sxref}[2]{(\ref{ex:#1}\ref{ex:#1#2})} %(3a) type example references
\newcommand{\term}[1]{\textsc{#1}}
\newcommand{\type}[1]{\textit{#1\/}}
\newcommand{\val}[1]{\textsc{#1}}
\newcommand{\whIs}{\ling{wh}-Interrogatives}
\newcommand{\whs}{\ling{wh} expressions}
\newcommand{\Whs}{\ling{Wh} expressions}
\newcommand{\xref}[1]{(\ref{ex:#1})} %(3) type example references
%
%\newcommand{\AP}{\textsc{ap\xspace}}
%\newcommand{\NP}{\textsc{np\xspace}}
%\newcommand{\PP}{\textsc{pp\xspace}}
%\newcommand{\VP}{\textsc{vp\xspace}}
%%%%
% changed \textsc to \textsf, 2002-12-02
%%%
\newcommand{\AP}{\textsf{AP}}
\newcommand{\NP}{\textsf{NP}}
\newcommand{\PP}{\textsf{PP}}
\newcommand{\VP}{\textsf{VP}}
\newcommand{\Adj}{\textsf{Adj}}
\newcommand{\Adv}{\textsf{Adv}}
\newcommand{\Aux}{\textsf{Aux}}
\newcommand{\Nom}{\textsf{Nom}}
\newcommand{\Conj}{\textsf{Conj}}
\newcommand{\Pro}{\textsf{Pro}}
\newcommand{\Card}{\textsf{Card}}
\newcommand{\Quant}{\textsf{Quant}}
\newcommand{\Ord}{\textsf{Ord}}
\newcommand{\AdvP}{\textsf{AdvP}}
\newcommand{\A}{\textsf{A}}
\newcommand{\Det}{\textsf{Det}}
\newcommand{\N}{\textsf{N}}
\newcommand{\PropN}{\textsf{PropN}}
\newcommand{\Prep}{\textsf{P}}
\newcommand{\V}{\textsf{V}}
\newcommand{\Se}{\textsf{S}}
\newcommand{\Ntbar}{$\overline{\overline{\overline{\mbox{N}}}}$}
\newcommand{\Ndbar}{$\overline{\overline{\mbox{N}}}$}
\newcommand{\Nbar}{$\overline{\mbox{N}}$}
\newcommand{\Atbar}{$\overline{\overline{\overline{\mbox{A}}}}$}
\newcommand{\Adbar}{$\overline{\overline{\mbox{A}}}$}
\newcommand{\Abar}{$\overline{\mbox{A}}$}
\newcommand{\Ptbar}{$\overline{\overline{\overline{\mbox{P}}}}$}
\newcommand{\Pdbar}{$\overline{\overline{\mbox{P}}}$}
\newcommand{\Pbar}{$\overline{\mbox{P}}$}
\newcommand{\Vtbar}{$\overline{\overline{\overline{\mbox{V}}}}$}
\newcommand{\Vdbar}{$\overline{\overline{\mbox{V}}}$}
\newcommand{\Vbar}{$\overline{\mbox{V}}$}
\newcommand{\Xtbar}{$\overline{\overline{\overline{\mbox{X}}}}$}
\newcommand{\Xdbar}{$\overline{\overline{\mbox{X}}}$}
\newcommand{\Xbar}{$\overline{\mbox{X}}$}
\newcommand{\PS}{\textsc{ps}}
\newcommand{\PSG}{\textsc{psg}}
\newcommand{\Vintrans}{V$_{intrans}$}
\newcommand{\Vtrans}{V$_{trans}$}
\newcommand{\Vsent}{V$_{sent}$}
\newcommand{\XP}{\textsf{XP}}
%
%\newcommand{\W}[1]{\makebox[.75in]{#1}}
%
\newcommand{\cf}{\textsl{c\,f\xspace}}
\newcommand{\ie}{\textsl{i\,e\xspace}}
\newcommand{\eg}{\textsl{e\,g\xspace}}
%
\newcommand{\Book}[1]{\textit{#1\/}}
%
\newenvironment{gram}{%
    \begin{verse}}%
    {\end{verse}}

% Lists
\newenvironment{shortlist}{%
    \begin{itemize}%
        \setlength{\parsep}{0pt}
        \setlength{\itemsep}{0pt}}{%
    \end{itemize}}

\newenvironment{enum}{%
    \begin{enumerate}%
        \setlength{\parsep}{0pt}
        \setlength{\itemsep}{0pt}}{%
    \end{enumerate}}

\newenvironment{desc}{%
    \begin{description}%
        \setlength{\parsep}{0pt}
        \setlength{\itemsep}{0pt}}{%
    \end{description}}



%%%%%%%%%%%%%%%%%%%%%%%%%%%%%%%%%%%%%%%%%%%%%%%%%%%%%%%%%%%%

\usepackage{color}
\usepackage{amsmath}
\usepackage{graphicx}
\usepackage[plain]{algorithm}
%\renewcommand{\algorithmicelsif}{\textbf{elsif}}
% Seminar style settings
%%%%%%%%%%%%%%%%%%%%%%%%
% 

% Shading
\definecolor{light}{gray}{.80}
\newcommand{\Hilite}[1]{\colorbox{yellow}{#1}}
\newcommand{\Shade}[1]{\colorbox{light}{#1}}

\newcommand{\Em}[1]{\textcolor{red}{#1}}
\newcommand{\Dim}[1]{\textcolor{gray}{#1}}

\setlength{\parskip}{0in}
\setlength{\parindent}{0in}

%%%%%%%%%%%%%%%%%%%%%%%%%%%%%%%%%%%%%%%%%%%%%%%%%%%%%%%%%%
%% JM abbreviations
\def\tab{\hbox{\kern0.4in}}    %% Insert space (even at start of line)
\def\al{\\\tab}                %% Next line is indented
\def\nl{\\\tab\tab}            %% Next line is double-indented
\def\nnl{\\\tab\tab\tab}       %% Next line is triple-indented
\def\blob{$\diamondsuit\ \ $}  %% Itemization without all the spacing
\def\mysum{\begin{Huge}\mbox{$\Sigma$}\end{Huge}}
\def\myprod{\begin{Huge}\mbox{$\Pi$}\end{Huge}}
\def\rarrow{$\rightarrow\ \ $}  %% 

\newcommand{\elt}{{\,{\in}\,}}  %%%cuts down on spacing
\newcommand{\eq}{{\,{=}\,}}  %%%cuts down on spacing
\def\stimes{{\,\times\,}}       %%%cuts down on spacing

%% BNF grammars %pnorvig Mar 14 1994:
\newcommand{\bl}{\;}                   %A blank between BNF constituents
\newcommand{\bnf}[1]{{\it #1}}          %A BNF constituent
\newcommand{\bnfeq}{\ensuremath{\rightarrow}} % BNF arrow (separates LHS from RHS) 
\newcommand{\bnfreveq}{\ensuremath{\leftarrow}}      % Reverse BNF arrow
\newcommand{\bnft}[1]{\ensuremath{\boldmath \( #1 \)}} % A BNF terminal 
\newcommand{\bnfv}[1]{\bnf{#1}}         % A BNF terminal variable constituent
\newcommand{\bnfnull}{\bnft{\epsilon}}  % An empty BNF right-hand-side
\newcommand{\bnfor}{\ensuremath{\mid\bl}}            % A BNF disjunction
\newcommand{\dcg}{\Leftarrow_{dcg}}     % Obsolete!

\newcommand{\fhs}[3]{\frac{#1}{#2} #3}  % Filler, Hole, Semantics

\newlength{\maxfigwidth}
\setlength{\maxfigwidth}{\textwidth}
\addtolength{\maxfigwidth}{-8\fboxsep}
\addtolength{\maxfigwidth}{-8\fboxrule}

%%%%%%%%%%%% elements of programs %%%%%%%%%%%%%%%%%%%%%%%%%%%%%%%%%


\newcommand{\defprog}[1]{\mbox{{\sc #1}}} 
\newcommand{\prog}[1]{\mbox{{\sc #1}}}   %%% same as defprog
\newcommand{\noprog}[1]{\mbox{{\sc #1}}} %%% same as defprog
\newcommand{\system}[1]{\mbox{{\sc #1}}} %%% same as defprog
\newcommand{\nosystem}[1]{\mbox{{\sc #1}}} %%% same as defprog
\newcommand{\act}[1]{{\it #1}}
\newcommand{\key}[1]{{\bf #1}}
\def\k{\key}
\newcommand{\var}[1]{{\it #1}}
\def\v{\var}
\def\p{\prg}
\newcommand{\setq}[2]{#1\hbox{$\,\leftarrow\,$}#2}
\newcommand{\setqmath}[2]{#1 \leftarrow #2}

%%pnorvig Mar 14 1994:
\newcommand{\labl}[1]{\prog{#1}}        % Statement Label in code
\newcommand{\cmt}[1]{{\tt /*} {\it #1} {\tt */}}        % Comment in code
\newcommand{\rcmt}[1]{\hfill {\tt /*} {\it #1} {\tt */}}
\newcommand{\Endfn}[1]{}                % End of a function in code
\newcommand{\fnsep}{\hspace*{0in}\raisebox{3pt}{\rule{\codewidth}{0.2pt}}\hfill}        % Between functions in code

\def\code{\noindent\nobreak\begin{Large}
        \begingroup\obeylines\obeyspaces\docode}
\def\docode#1{\noindent\framebox[\columnwidth][l]%
{~~\begin{minipage}{\codewidth}~
#1
~\end{minipage}}\endgroup%
\nobreak\end{Large}\noindent\ignorespaces}
\def\asis{\obeylines\obeyspaces}
\def\ttp{.\skip -0.5em\ }
{\obeyspaces\global\def {\ }}


%% Fine tuning for perfect formatting

\newcommand{\ac}{,\hspace{0.15em}}
\newcommand{\ts}{\,}
\newcommand{\bodysep}{\vspace{0.1in}}
\newcommand{\subfnsep}{\vspace{0.06in}}
\newcommand{\prebnfsep}{\vspace{-0.25in}}
\newcommand{\postbnfsep}{\vspace{0.1in}}
\newcommand{\fnvar}[1]{\prog{#1}}
\newcommand{\func}[3]{\key{function} \defprog{#1}(#2) \key{returns} #3}
\newcommand{\nofunc}[3]{\key{function} \noprog{#1}(#2) \key{returns} #3}
\newcommand{\proc}[2]{\key{procedure} \defprog{#1}(#2)}
\newcommand{\noproc}[2]{\key{procedure} \noprog{#1}(#2)}
\newcommand{\firstinputs}[2]{\key{inputs}: \var{#1}, #2}
\newcommand{\inputs}[2]{\phantom{\key{inputs}: }\var{#1}, #2}
\newcommand{\firstlocal}[2]{\key{local variables}: \var{#1}, #2}
\newcommand{\local}[2]{\phantom{\key{local variables}: }\var{#1}, #2}
\newcommand{\firststatic}[2]{\key{static}: \var{#1}, #2}
\newcommand{\static}[2]{\phantom{\key{static}: }\var{#1}, #2}

\newlength{\codewidth}
\setlength{\codewidth}{\textwidth}
\addtolength{\codewidth}{-0.5in}

%%%%%%%%%%%%%%%%%%%%%%%%%%%%%%%%%%%%%%%%%%%%%%%%%%%%%%%%%%%%%%%%%%%%%%%%%%%%%
\usepackage{synttree}
\branchheight{5ex}


\begin{document}
\frame{\titlepage}

\mode<article>{\section[Outline]{ICL/Chart Parsing, Part 2/2005-11-10}}
\mode<presentation>{
  \section[Outline]{}
}

\frame{\tableofcontents}

\section{Review Chart Parsing}

\begin{frame}[fragile]
 \frametitle{Charts and Edges}


 \begin{itemize}
 \item A \Em{chart} records hypotheses about possible constituents;
 \item it contains a set of \Em{edges}.
 \item Each edge has information about
   \begin{itemize}
   \item start index of the constituent,
   \item end index,
   \item the hypothesized type of the constituent and its sub-constituents
   \end{itemize}
 \item NLTK: \verb![A -> B C]@[i:j]!
 \item The content of an edge takes the form of a dotted rule:
$A \rightarrow \alpha \, \bigdot \, \beta$:
\begin{itemize}
\item $\alpha$ is what we've \Em{found} so far,
\item $\beta$ is what we still  \Em{need} in order to complete an $A$.
\end{itemize}
 \end{itemize}


\end{frame}

\begin{frame}[fragile]
 \frametitle{Chart Parser Rules}


 \begin{itemize}
 \item A chart parser \Em{rule} (or function) adds new edges to the chart.
 \item A chart parsing \Em{strategy} defines a set of rules.
 \item \Em{Top Down}
   \begin{itemize}
   \item top down initialization rule
   \item top down predictor rule
   \item fundamental rule (completer)
   \end{itemize}
 \item \Em{Bottom Up}
   \begin{itemize}
   \item bottom up predictor rule
   \item fundamental rule (completer)
   \end{itemize}
 
 \end{itemize}


\end{frame}

\begin{frame}[fragile]
 \frametitle{Fundamental Rule}

 \begin{itemize}
 \item Fundamental Rule is used by both strategies.
 \item If the chart contains
   \begin{itemize}
   \item $A \rightarrow \alpha \, \bigdot B \gamma, [i,j]$
 and
\item $C \rightarrow \beta [j,k]$, then add  

\item $A \rightarrow \alpha  B \, \bigdot \gamma, [i,k]$
   \end{itemize}
 

 \end{itemize}
\end{frame}

\begin{frame}[fragile]
 \frametitle{Top Down Predictor Rule}

 \begin{itemize}
 \item If the chart contains
   \begin{itemize}
   \item $A \rightarrow \alpha \, \bigdot  \, C \beta, [i,j]$
     then
      \begin{itemize}
      \item for each production $C \rightarrow \gamma$,  add  
      \item $C \rightarrow  \bigdot \gamma, [j,j]$
      \end{itemize}
   \end{itemize}
 \end{itemize}

\end{frame}

\begin{frame}[fragile]
 \frametitle{Bottom Up Predictor Rule}

 \begin{itemize}
 \item If the chart contains
   \begin{itemize}
   \item $A \rightarrow \alpha \, \bigdot, [i,j]$
     then
      \begin{itemize}
      \item for each production $B \rightarrow A \beta$,  add  
      \item $B \rightarrow  \bigdot A  \beta, [i,j]$
      \end{itemize}
   \end{itemize}
 \end{itemize}

\end{frame}


%%%%%%%%%%%%%%%%%%%%%%%%%%%%%%%%%%%%%%%%%%%%%%%%%%
% Slide 25
%%%%%%%%%%%%%%%%%%%%%%%%%%%%%%%%%%%%%%%%%%%%%%%%%%

%\begin{frame}[fragile]
%  \frametitle{Examples: Local Ambiguity}


%Let's assume that we're parsing the \VP \ling{gave mary a book} using the 
%following  rules:


%\begin{quote}
%  \Se\ $\rightarrow$ \VP

%  \VP\ $\rightarrow$ \V
  
%  \VP\ $\rightarrow$ \V\ \NP
  
%  \VP\ $\rightarrow$ \V\  \NP  \PP
  
%  \VP\ $\rightarrow$ \V\  \NP  \NP

%\end{quote}
%\end{frame}

%%%%%%%%%%%%%%%%%%%%%%%%%%%%%%%%%%%%%%%%%%%%%%%%%%%
%% Slide 26
%%%%%%%%%%%%%%%%%%%%%%%%%%%%%%%%%%%%%%%%%%%%%%%%%%%

%\begin{frame}[fragile]
%  \frametitle{Examples: Local Ambiguity}






\section{Left Recursion and Ambiguity}

%\includegraphics[scale=.7]{moore/figures/mary.ps}

%\end{frame}

%%%%%%%%%%%%%%%%%%%%%%%%%%%%%%%%%%%%%%%%%%%%%%%%%%
% Slide 27
%%%%%%%%%%%%%%%%%%%%%%%%%%%%%%%%%%%%%%%%%%%%%%%%%%

\begin{frame}[fragile]
  \frametitle{Examples: Left Recursion}


Assume we are parsing \NP \ling{a flight from Denver to Boston}
with the following rules:
\begin{quote}
  \NP\ $\rightarrow$ \NP\ \PP

  \NP\ $\rightarrow$ \Det\ \Nom

  \NP\ $\rightarrow$ Proper-Noun

\end{quote}

\begin{itemize}
  \item We construct the state (\NP\ $\rightarrow$ \bigdot\ \NP\ \PP,
    [0,0]) and add it to \textit{chart}[0]
  \item The Predictor rule  requires us to find a rule
    which expands the (non-lexical) category immediately to the right of
    the dot. 
  \item Pick the first rule above, and add the
    state (\NP\ $\rightarrow$ \bigdot\ \NP\ \PP, [0,0]).
  \item But this is already in the chart, so we don't add it again. 
\end{itemize}


%\includegraphics[scale=.7]{moore/figures/denver.ps}

\end{frame}

%%%%%%%%%%%%%%%%%%%%%%%%%%%%%%%%%%%%%%%%%%%%%%%%%%%
%% Slide 28
%%%%%%%%%%%%%%%%%%%%%%%%%%%%%%%%%%%%%%%%%%%%%%%%%%%

\begin{frame}[fragile]
 \frametitle{Examples: Global Ambiguity}

\includegraphics[scale=.6]{../images/oldmen.pdf}


\end{frame}

\begin{frame}[fragile]
 \frametitle{Examples: Global Ambiguity}

\includegraphics[scale=.6]{../images/oldmen2.pdf}


\end{frame}


%\begin{frame}[fragile]
%  \frametitle{Error handling}


%  \begin{itemize}
    

%  \item What happens when we look at the contents of the last table
%    column and don't find a \Se\ $\rightarrow \alpha \, \bigdot$ rule?
%  \item Is  there some way to proceed?
%  \item Yes. The chart contains every constituent and combination of
%    constituents that could have been found for that input given the
%    grammar.
%  \item Useful for partial (or shallow) parsing, also called
%    \ling{chunking}, which is frequently used in information extraction
%    and question answering.
%\end{itemize}

%\end{frame}



%\end{frame}

\begin{frame}[fragile]
  \frametitle{Chart Parser Demo}

\begin{verbatim}
>>> from nltk_lite.draw.chart import demo
>>> demo()
\end{verbatim}


\end{frame}



\section{Reading}

%%%%%%%%%%%%%%%%%%%%%%%%%%%%%%%%%%%%%%%%%%%%%%%%%%
% Slide 33
%%%%%%%%%%%%%%%%%%%%%%%%%%%%%%%%%%%%%%%%%%%%%%%%%%

\begin{frame}[fragile]
  \frametitle{Reading}

{\large 
  \begin{itemize}
    
  \item Read section 10.4 of J\&M

  \item Read the NLTK-Lite Tutorial on Chart Parsing


  \end{itemize}

}


\end{frame}




\end{document}


%%% Local Variables: 
%%% mode: latex
%%% TeX-master: "earley-pt2-lec"
%%% End: 
