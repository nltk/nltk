%%%%%%%%%%%%%%%%%%%%%%%%%%%%%%%%%%%%%%%%
% Revised: Time-stamp: <2006-03-09 09:17:28 ewan>
% 
% $Log$
% Revision 1.2  2006/03/09 09:23:39  ehk
% Made file xrefs consistent after renaming
%
% Revision 1.1  2006/02/08 15:36:53  ehk
% First import of beamer.sty slides. Some files still need renaming.
%
% Revision 1.4  2005/11/08 15:34:52  ewan
% *** empty log message ***
%
% Revision 1.3  2005/11/07 12:54:07  ewan
% Fixed images
%
% Revision 1.2  2005/11/07 09:53:00  ewan
% First cut
%
% Revision 1.1  2005/10/28 18:35:56  ewan
% *** empty log message ***
%
% Revision 1.1  2004/11/17 13:54:43  ewan
% *** empty log message ***
%
% Revision 1.1  2004/08/16 16:24:28  ewan
% *** empty log message ***
%
% Revision 1.1  2004/08/09 21:56:09  neilb
% move from teaching/modules/icl to teaching/courses/icl. No corrections done
%
% Revision 1.9  2003/11/17 12:10:32  ewan
% Edited includegraphics to look in images for chart_n.eps files
%
% Revision 1.8  2003/11/17 12:02:54  ewan
% First import into CVS
%
% Revision 1.1  2003/10/14 15:23:23  ewan
% *** empty log message ***
%
% Revision 1.2  2002/12/06 14:06:20  ewan
% Added acknowledgement slide
%
% Revision 1.1  2002/11/25 10:08:31  ewan
% First import into CVS
%
%
%
%
%%%%%%%%%%%%%%%%%%%%%%%%%%%%%%%%%%%%%%%%
\title{Chart Parsing}
\author{Ewan Klein \newline \mbox{ }ewan@inf.ed.ac.uk\mbox{ }}
\date{ICL --- 7 November 2005}



\newcommand{\Rule}{\rule{\textwidth}{1pt}}
\newcommand{\SqB}[1]{[#1]}

\newcommand{\bigdot}{\mbox{\begin{scriptsize}{\ensuremath{\bullet}}\end{scriptsize}}}

\newcommand{\chdot}{\bigdot}

%%%%%%%%%%%%%%%%%%%%%%%%%%%%%%%%%%%%%%%%
% Title:  defs
%
% Doc: /home/ewan/Teach/Intro/src/defs.tex
% Original Author:     Ewan Klein (ewan@zangfu)
% Created:             Tue Oct  8 1996
% Revised: Time-stamp: <2005-10-28 16:38:35 ewan>
% $Log$
% Revision 1.1  2006/03/09 09:13:21  ehk
% *** empty log message ***
%
% Revision 1.1  2005/11/03 10:42:49  ewan
% *** empty log message ***
%
% Revision 1.1  2004/09/28 07:28:46  ewan
% *** empty log message ***
%
% Revision 1.2  1996/10/11  16:01:58  ewan
% Installed in /projects/ltg/users/colin/Teach/Intro
%
% Revision 1.2  1996/10/09  03:41:25  ewan
% *** empty log message ***
%
%%%%%%%%%%%%%%%%%%%%%%%%%%%%%%%%%%%%%%%%
%\newcommand{\iupp}{\^\i}
%\newcommand{\nasa}{\diatop[\~|a]}
%\newcommand{\nase}{\diatop[\~|\niepsilon]}
%\newcommand{\naso}{\diatop[\~|\openo]}
%\newcommand{\slt}{\diatop[/|t]}
%\newcommand{\hand}{\ding{43}}
%\newcommand{\hand}{$\Rightarrow$}
\newcommand{\bad}{\sqz{*}}
\newcommand{\cat}[1]{\textit{#1\/}} %linguistic categories
\newcommand{\con}[1]{\textbf{#1\/}} %constituents etc in examples
\newcommand{\dash}{$\underline{\hspace{1em}}$}
\newcommand{\Easy}{\ling{easy}-Constructions}
\newcommand{\Feat}[1]{\textsc{#1}}
\newcommand{\feat}[1]{\textsc{#1}}
\newcommand{\featval}[2]{\textsc{#1}\ \textsc{#2}}
\newcommand{\fun}[1]{\mbox{\textsl{#1}}}
\newcommand{\mygloss}[1]{`{#1}'}
\newcommand{\group}[1]{$\underbrace{\mbox{#1}}$}
\newcommand{\hpsg}{\textsc{hpsg}}
\newcommand{\latin}[1]{\textit{#1\/}} 
%\newcommand{\lextab}{\begin{tabular}{l l @{:\extracolsep{.5in}} l}}
\newcommand{\lextab}{\begin{tabular}{l l l}}
\newcommand{\lingloss}[2]{\ling{#1} \gloss{#2}}
\newcommand{\ling}[1]{\mbox{\textit{#1}}}
\newcommand{\quot}[1]{\textit{#1\/}}
\newcommand{\scare}[1]{`#1'}
\newcommand{\str}[1]{$ #1 $}
\newcommand{\sxref}[2]{(\ref{ex:#1}\ref{ex:#1#2})} %(3a) type example references
\newcommand{\term}[1]{\textsc{#1}}
\newcommand{\type}[1]{\textit{#1\/}}
\newcommand{\val}[1]{\textsc{#1}}
\newcommand{\whIs}{\ling{wh}-Interrogatives}
\newcommand{\whs}{\ling{wh} expressions}
\newcommand{\Whs}{\ling{Wh} expressions}
\newcommand{\xref}[1]{(\ref{ex:#1})} %(3) type example references
%
%\newcommand{\AP}{\textsc{ap\xspace}}
%\newcommand{\NP}{\textsc{np\xspace}}
%\newcommand{\PP}{\textsc{pp\xspace}}
%\newcommand{\VP}{\textsc{vp\xspace}}
%%%%
% changed \textsc to \textsf, 2002-12-02
%%%
\newcommand{\AP}{\textsf{AP}}
\newcommand{\NP}{\textsf{NP}}
\newcommand{\PP}{\textsf{PP}}
\newcommand{\VP}{\textsf{VP}}
\newcommand{\Adj}{\textsf{Adj}}
\newcommand{\Adv}{\textsf{Adv}}
\newcommand{\Aux}{\textsf{Aux}}
\newcommand{\Nom}{\textsf{Nom}}
\newcommand{\Conj}{\textsf{Conj}}
\newcommand{\Pro}{\textsf{Pro}}
\newcommand{\Card}{\textsf{Card}}
\newcommand{\Quant}{\textsf{Quant}}
\newcommand{\Ord}{\textsf{Ord}}
\newcommand{\AdvP}{\textsf{AdvP}}
\newcommand{\A}{\textsf{A}}
\newcommand{\Det}{\textsf{Det}}
\newcommand{\N}{\textsf{N}}
\newcommand{\PropN}{\textsf{PropN}}
\newcommand{\Prep}{\textsf{P}}
\newcommand{\V}{\textsf{V}}
\newcommand{\Se}{\textsf{S}}
\newcommand{\Ntbar}{$\overline{\overline{\overline{\mbox{N}}}}$}
\newcommand{\Ndbar}{$\overline{\overline{\mbox{N}}}$}
\newcommand{\Nbar}{$\overline{\mbox{N}}$}
\newcommand{\Atbar}{$\overline{\overline{\overline{\mbox{A}}}}$}
\newcommand{\Adbar}{$\overline{\overline{\mbox{A}}}$}
\newcommand{\Abar}{$\overline{\mbox{A}}$}
\newcommand{\Ptbar}{$\overline{\overline{\overline{\mbox{P}}}}$}
\newcommand{\Pdbar}{$\overline{\overline{\mbox{P}}}$}
\newcommand{\Pbar}{$\overline{\mbox{P}}$}
\newcommand{\Vtbar}{$\overline{\overline{\overline{\mbox{V}}}}$}
\newcommand{\Vdbar}{$\overline{\overline{\mbox{V}}}$}
\newcommand{\Vbar}{$\overline{\mbox{V}}$}
\newcommand{\Xtbar}{$\overline{\overline{\overline{\mbox{X}}}}$}
\newcommand{\Xdbar}{$\overline{\overline{\mbox{X}}}$}
\newcommand{\Xbar}{$\overline{\mbox{X}}$}
\newcommand{\PS}{\textsc{ps}}
\newcommand{\PSG}{\textsc{psg}}
\newcommand{\Vintrans}{V$_{intrans}$}
\newcommand{\Vtrans}{V$_{trans}$}
\newcommand{\Vsent}{V$_{sent}$}
\newcommand{\XP}{\textsf{XP}}
%
%\newcommand{\W}[1]{\makebox[.75in]{#1}}
%
\newcommand{\cf}{\textsl{c\,f\xspace}}
\newcommand{\ie}{\textsl{i\,e\xspace}}
\newcommand{\eg}{\textsl{e\,g\xspace}}
%
\newcommand{\Book}[1]{\textit{#1\/}}
%
\newenvironment{gram}{%
    \begin{verse}}%
    {\end{verse}}

% Lists
\newenvironment{shortlist}{%
    \begin{itemize}%
        \setlength{\parsep}{0pt}
        \setlength{\itemsep}{0pt}}{%
    \end{itemize}}

\newenvironment{enum}{%
    \begin{enumerate}%
        \setlength{\parsep}{0pt}
        \setlength{\itemsep}{0pt}}{%
    \end{enumerate}}

\newenvironment{desc}{%
    \begin{description}%
        \setlength{\parsep}{0pt}
        \setlength{\itemsep}{0pt}}{%
    \end{description}}



%%%%%%%%%%%%%%%%%%%%%%%%%%%%%%%%%%%%%%%%%%%%%%%%%%%%%%%%%%%%

\usepackage{color}
\usepackage{amsmath}
\usepackage{graphicx}
\usepackage[plain]{algorithm}
%\renewcommand{\algorithmicelsif}{\textbf{elsif}}
% Seminar style settings
%%%%%%%%%%%%%%%%%%%%%%%%
% 

% Shading
\definecolor{light}{gray}{.80}
\newcommand{\Hilite}[1]{\colorbox{yellow}{#1}}
\newcommand{\Shade}[1]{\colorbox{light}{#1}}

\newcommand{\Em}[1]{\textcolor{red}{#1}}
\newcommand{\Dim}[1]{\textcolor{gray}{#1}}

\setlength{\parskip}{0in}
\setlength{\parindent}{0in}

%%%%%%%%%%%%%%%%%%%%%%%%%%%%%%%%%%%%%%%%%%%%%%%%%%%%%%%%%%
%% JM abbreviations
\def\tab{\hbox{\kern0.4in}}    %% Insert space (even at start of line)
\def\al{\\\tab}                %% Next line is indented
\def\nl{\\\tab\tab}            %% Next line is double-indented
\def\nnl{\\\tab\tab\tab}       %% Next line is triple-indented
\def\blob{$\diamondsuit\ \ $}  %% Itemization without all the spacing
\def\mysum{\begin{Huge}\mbox{$\Sigma$}\end{Huge}}
\def\myprod{\begin{Huge}\mbox{$\Pi$}\end{Huge}}
\def\rarrow{$\rightarrow\ \ $}  %% 

\newcommand{\elt}{{\,{\in}\,}}  %%%cuts down on spacing
\newcommand{\eq}{{\,{=}\,}}  %%%cuts down on spacing
\def\stimes{{\,\times\,}}       %%%cuts down on spacing

%% BNF grammars %pnorvig Mar 14 1994:
\newcommand{\bl}{\;}                   %A blank between BNF constituents
\newcommand{\bnf}[1]{{\it #1}}          %A BNF constituent
\newcommand{\bnfeq}{\ensuremath{\rightarrow}} % BNF arrow (separates LHS from RHS) 
\newcommand{\bnfreveq}{\ensuremath{\leftarrow}}      % Reverse BNF arrow
\newcommand{\bnft}[1]{\ensuremath{\boldmath \( #1 \)}} % A BNF terminal 
\newcommand{\bnfv}[1]{\bnf{#1}}         % A BNF terminal variable constituent
\newcommand{\bnfnull}{\bnft{\epsilon}}  % An empty BNF right-hand-side
\newcommand{\bnfor}{\ensuremath{\mid\bl}}            % A BNF disjunction
\newcommand{\dcg}{\Leftarrow_{dcg}}     % Obsolete!

\newcommand{\fhs}[3]{\frac{#1}{#2} #3}  % Filler, Hole, Semantics

\newlength{\maxfigwidth}
\setlength{\maxfigwidth}{\textwidth}
\addtolength{\maxfigwidth}{-8\fboxsep}
\addtolength{\maxfigwidth}{-8\fboxrule}

%%%%%%%%%%%% elements of programs %%%%%%%%%%%%%%%%%%%%%%%%%%%%%%%%%


\newcommand{\defprog}[1]{\mbox{{\sc #1}}} 
\newcommand{\prog}[1]{\mbox{{\sc #1}}}   %%% same as defprog
\newcommand{\noprog}[1]{\mbox{{\sc #1}}} %%% same as defprog
\newcommand{\system}[1]{\mbox{{\sc #1}}} %%% same as defprog
\newcommand{\nosystem}[1]{\mbox{{\sc #1}}} %%% same as defprog
\newcommand{\act}[1]{{\it #1}}
\newcommand{\key}[1]{{\bf #1}}
\def\k{\key}
\newcommand{\var}[1]{{\it #1}}
\def\v{\var}
\def\p{\prg}
\newcommand{\setq}[2]{#1\hbox{$\,\leftarrow\,$}#2}
\newcommand{\setqmath}[2]{#1 \leftarrow #2}

%%pnorvig Mar 14 1994:
\newcommand{\labl}[1]{\prog{#1}}        % Statement Label in code
\newcommand{\cmt}[1]{{\tt /*} {\it #1} {\tt */}}        % Comment in code
\newcommand{\rcmt}[1]{\hfill {\tt /*} {\it #1} {\tt */}}
\newcommand{\Endfn}[1]{}                % End of a function in code
\newcommand{\fnsep}{\hspace*{0in}\raisebox{3pt}{\rule{\codewidth}{0.2pt}}\hfill}        % Between functions in code

\def\code{\noindent\nobreak\begin{Large}
        \begingroup\obeylines\obeyspaces\docode}
\def\docode#1{\noindent\framebox[\columnwidth][l]%
{~~\begin{minipage}{\codewidth}~
#1
~\end{minipage}}\endgroup%
\nobreak\end{Large}\noindent\ignorespaces}
\def\asis{\obeylines\obeyspaces}
\def\ttp{.\skip -0.5em\ }
{\obeyspaces\global\def {\ }}


%% Fine tuning for perfect formatting

\newcommand{\ac}{,\hspace{0.15em}}
\newcommand{\ts}{\,}
\newcommand{\bodysep}{\vspace{0.1in}}
\newcommand{\subfnsep}{\vspace{0.06in}}
\newcommand{\prebnfsep}{\vspace{-0.25in}}
\newcommand{\postbnfsep}{\vspace{0.1in}}
\newcommand{\fnvar}[1]{\prog{#1}}
\newcommand{\func}[3]{\key{function} \defprog{#1}(#2) \key{returns} #3}
\newcommand{\nofunc}[3]{\key{function} \noprog{#1}(#2) \key{returns} #3}
\newcommand{\proc}[2]{\key{procedure} \defprog{#1}(#2)}
\newcommand{\noproc}[2]{\key{procedure} \noprog{#1}(#2)}
\newcommand{\firstinputs}[2]{\key{inputs}: \var{#1}, #2}
\newcommand{\inputs}[2]{\phantom{\key{inputs}: }\var{#1}, #2}
\newcommand{\firstlocal}[2]{\key{local variables}: \var{#1}, #2}
\newcommand{\local}[2]{\phantom{\key{local variables}: }\var{#1}, #2}
\newcommand{\firststatic}[2]{\key{static}: \var{#1}, #2}
\newcommand{\static}[2]{\phantom{\key{static}: }\var{#1}, #2}

\newlength{\codewidth}
\setlength{\codewidth}{\textwidth}
\addtolength{\codewidth}{-0.5in}

%%%%%%%%%%%%%%%%%%%%%%%%%%%%%%%%%%%%%%%%%%%%%%%%%%%%%%%%%%%%%%%%%%%%%%%%%%%%%
\usepackage{synttree}
\branchheight{5ex}


\begin{document}
\frame{\titlepage}

\mode<article>{\section[Outline]{ICL/Chart Parsing CFGs/2005-11-07}}
\mode<presentation>{
  \section[Outline]{}
}

\frame{\tableofcontents}

\section{Review Top-down Parsing}

%%%%%%%%%%%%%%%%%%%%%%%%%%%%%%%%%%%%%%%%%%%%%%%%%%
% Slide 1
%%%%%%%%%%%%%%%%%%%%%%%%%%%%%%%%%%%%%%%%%%%%%%%%%%



\begin{frame}[fragile]
  \frametitle{Parsing}


Parsing with a CFG is the task of assigning a correct tree (or
derivation) to a string given some grammar.

A correct tree is:
\begin{itemize}
\item consistent with the grammar, and
\item the leaves of the tree cover all and only the words in the input.
\end{itemize}

There may be a very large number of correct trees for any given input \ldots


\end{frame}


%%%%%%%%%%%%%%%%%%%%%%%%%%%%%%%%%%%%%%%%%%%%%%%%%%
% Slide 3
%%%%%%%%%%%%%%%%%%%%%%%%%%%%%%%%%%%%%%%%%%%%%%%%%%

\begin{frame}[fragile]
  \frametitle{Problems that arise}

  \begin{itemize}
    
  \item Left Recursion
    
  \item Ambiguity
    
  \item Inefficiencies due to backtracking
\end{itemize}

\end{frame}

%%%%%%%%%%%%%%%%%%%%%%%%%%%%%%%%%%%%%%%%%%%%%%%%%%
% Slide 28
%%%%%%%%%%%%%%%%%%%%%%%%%%%%%%%%%%%%%%%%%%%%%%%%%%

\begin{frame}[fragile]
  \frametitle{Common substructures}


  \begin{itemize}
  \item Despite ambiguity and backtracking there are commons
    substructures to be taken advantage of.


  \item Consider parsing the following \NP:

    \begin{quote}
      \ling{a flight from Indianapolis to Houston on TWA}
    \end{quote}

    with the following rules:

    \begin{quote}
      \NP\ $\rightarrow$ \Det \,\, \Nom

      \Nom\ $\rightarrow$ \Nom \,\, \PP
  
%      \NP\ $\rightarrow$ \PropN
    \end{quote}

  \item What happens with a top-down parser?
  \end{itemize}

\end{frame}

%%%%%%%%%%%%%%%%%%%%%%%%%%%%%%%%%%%%%%%%%%%%%%%%%%
% Slide 29
%%%%%%%%%%%%%%%%%%%%%%%%%%%%%%%%%%%%%%%%%%%%%%%%%%

\begin{frame}[fragile]
    \frametitle{Backtracking}



\synttree[NP
            [Det [\ling{a}]]
            [Nom [.t \ling{flight}]]
] \underline{\ling{from Indianopolis to Houston on TWA}}

\medskip

\synttree[NP
            [Det [\ling{a}]]
            [Nom 
                 [Nom [.t \ling{flight}]]
                 [PP [.t \ling{from Indianapolis}]]]
] \underline{\ling{to Houston on TWA}}
  

\end{frame}

\begin{frame}[fragile]
    \frametitle{Backtracking, cont}


\synttree[NP
            [Det [\ling{a}]]
            [Nom 
                 [Nom 
                     [Nom [.t \ling{flight}]]
                     [PP [.t \ling{from Indianapolis}]]]
                 [PP [.t \ling{to Houston}]]]
] \underline{\ling{on TWA}}
  

\end{frame}

\begin{frame}[fragile]
    \frametitle{Backtracking, cont}


\synttree[NP
            [Det [\ling{a}]]
            [Nom 
                [Nom 
                     [Nom 
                         [Nom [.t \ling{flight}]]
                         [PP [.t \ling{from Indianapolis}]]]
                     [PP [.t \ling{to Houston}]]]
                [PP  [.t \ling{on TWA}]]]
]
  

\end{frame}




\section{Chart Parsing}

\subsection{Overview}
%%%%%%%%%%%%%%%%%%%%%%%%%%%%%%%%%%%%%%%%%%%%%%%%%%
% Slide 6
%%%%%%%%%%%%%%%%%%%%%%%%%%%%%%%%%%%%%%%%%%%%%%%%%%

\begin{frame}[fragile]
  \frametitle{Dynamic Programming}


Our current algorithm builds valid trees, discards them during
backtracking, then rebuilds them.
\begin{itemize}
\item the subtree for \ling{a flight} was derived 4 times.
\end{itemize}

\Em{Dynamic programming} is one answer to problems that have
sub-problems that get solved again and again

We'll consider an algorithm that fills a table with solutions to
subproblems that:


\begin{itemize}
\item does a parallel top-down search with bottom-up filtering
\item does not do repeated work
\item solves the left-recursion problem
%\item solves an exponential problem in $O(N^3)$ time
\end{itemize}

\end{frame}



%%%%%%%%%%%%%%%%%%%%%%%%%%%%%%%%%%%%%%%%%%%%%%%%%%
% Slide 7
%%%%%%%%%%%%%%%%%%%%%%%%%%%%%%%%%%%%%%%%%%%%%%%%%%

\begin{frame}[fragile]
  \frametitle{Dynamic Programming and Parsing}


  \begin{itemize}
    
  \item Systematically fill in tables of solutions to subproblems.
    
    
  \item When complete, the table contains all possible solutions to all of
    the subproblems needed to solve the whole problem
%    compactly encodes all possible solutions to the whole problem.
    
    
  \item For parsing:

    \begin{itemize}
    \item the table stores subtrees for constituents.
    
    
    \item Solves reparsing inefficiencies, because subtrees are not
      reparsed but looked up.
    
    
    \item Solves ambiguity explosions, because the table
      \emph{implicitly} stores all parses.
 
    \item Each subtree is represented only once and shared by all parses
      that need it.
  \end{itemize}
\end{itemize}

\end{frame}







\begin{frame}[fragile]
\frametitle{Lexical Edges}


\begin{center}
\includegraphics[scale=1]{../images/nchart0.pdf}  
\end{center}
\end{frame}

\subsection{Charts as Graphs}

\begin{frame}[fragile]
\frametitle{Nonterminal Edges: NP}

\begin{center}
\includegraphics[scale=1]{../images/nchart1.pdf}  
\end{center}

\end{frame}

\begin{frame}[fragile]
\frametitle{Nonterminal Edges: VP}

\begin{center}
\includegraphics[scale=1]{../images/nchart2.pdf}  
\end{center}

\end{frame}

\begin{frame}[fragile]
\frametitle{Nonterminal Edges: Rules}

\begin{itemize}
\item Useful to label arcs with \Em{rules} rather than categories:
\end{itemize}

\begin{center}
\includegraphics[scale=.8]{../images/nchart3.pdf}  
\end{center}

\end{frame}

\begin{frame}[fragile]
\frametitle{Incomplete Edges}

\begin{itemize}
\item Recall that the parser can make \Em{predictions} on the basis of rules:
  \begin{itemize}
  \item I'm trying to expand an \VP, and I've found a \V\ so I'll start looking for an \NP.
  \end{itemize}
\item Record incomplete constituents with \Em{dotted rules}:
  \begin{itemize}
  \item Dot on the RHS of a rule shows what we\'ve found already:\\
 \VP\ $\rightarrow$ V \,\,\bigdot\,\, \NP
\item We can use this as a label for an incomplete constituent.
  \end{itemize}
\end{itemize}

\end{frame}

\begin{frame}[fragile]
\frametitle{Incomplete Edges, cont.}

\begin{center}
\includegraphics[scale=1]{../images/nchart4.pdf}  
\end{center}

\end{frame}

% \begin{frame}[fragile]
% \frametitle{Nonterminal Edges: S}

% \begin{center}
% \includegraphics[scale=.85]{../images/chart3}  
% \end{center}

% \end{frame}

\subsection{The Basic Idea}

%%%%%%%%%%%%%%%%%%%%%%%%%%%%%%%%%%%%%%%%%%%%%%%%%%
% Slide 8
%%%%%%%%%%%%%%%%%%%%%%%%%%%%%%%%%%%%%%%%%%%%%%%%%%

\begin{frame}[fragile]
  \frametitle{Dynamic Programming and Parsing}


The Earley algorithm:

\begin{itemize}
  
\item fills a table (the \Em{chart}) in a single left-to-right pass
  over the input.
\item The chart will be size $N+1$, where $N$ is the number of words in
  the input.
\item Chart entries are associated with the gaps between the
  words --- like slice indexing in Python.

\item For each word position in the sentence, the chart contains a set
  of edges representing the partial parse trees generated so far.

\item So Chart[0] is the set of edges representing the parse before looking at a word.


\end{itemize}

\end{frame}

%%%%%%%%%%%%%%%%%%%%%%%%%%%%%%%%%%%%%%%%%%%%%%%%%%
% Slide 9
%%%%%%%%%%%%%%%%%%%%%%%%%%%%%%%%%%%%%%%%%%%%%%%%%%

\begin{frame}[fragile]
  \frametitle{States}

  \begin{itemize}
  \item J\&M call the chart entries \Em{states}.
 
\item The chart entries represent three distinct kinds of things:
  \begin{itemize}
  \item completed constituents;
  \item in-progress constituents; and
  \item predicted constituents
  \end{itemize}

  \item The three kinds of states correspond to different dot locations in
    dotted rules:
    \begin{description}
    \item[Completed:] \VP\ $\rightarrow$ V \,\, \NP \,\,\bigdot
  
    \item[In-progress:] \NP\ $\rightarrow$ \Det \,\,\bigdot \,\, \Nom\
  
    \item[Predicted:] \Se\ $\rightarrow$ \bigdot \,\, \VP
    \end{description}

  \end{itemize}
\end{frame}



\begin{frame}[fragile]
\frametitle{Incomplete NP Edge: Self-loop}

%\NP\ $\rightarrow$ \bigdot \,\, \Det \,\, \Nom\
\begin{center}
\includegraphics[scale=1]{../images/nchart5.pdf}  
\end{center}

\end{frame}



% \begin{frame}[fragile]
% \frametitle{Incomplete NP Edge: Det recognised}

% %\NP\ $\rightarrow$ \Det \,\,\bigdot \,\, \Nom\
% \begin{center}
% %\includegraphics[scale=1.2]{../images/chart0-2}  
% \end{center}

% \end{frame}

%%%%%%%%%%%%%%%%%%%%%%%%%%%%%%%%%%%%%%%%%%%%%%%%%%
% Slide 10
%%%%%%%%%%%%%%%%%%%%%%%%%%%%%%%%%%%%%%%%%%%%%%%%%%

\begin{frame}[fragile]
  \frametitle{States, cont.}


  \begin{itemize}
    
  \item Given dotted rules like those we've just seen, we need to record:

    \begin{itemize}
    \item where the represented constituent is in the input, and
    \item what its parts are.
    \end{itemize}

  \item So we add a pair of coordinates $[x,y]$ to the state:

\begin{itemize}
  
  \item $A \rightarrow \alpha , \; [x,y] $
  \item $x$ indicates the position in input where the state begins
  \item $y$ indicates position of dot
  \end{itemize}
\end{itemize}

\end{frame}



\begin{frame}[fragile]
\frametitle{Example with coordinates}


\begin{center}
\includegraphics[scale=1]{../images/nchart6}  
\end{center}
\end{frame}

\subsection{Example States}

%%%%%%%%%%%%%%%%%%%%%%%%%%%%%%%%%%%%%%%%%%%%%%%%%%
% Slide 11
%%%%%%%%%%%%%%%%%%%%%%%%%%%%%%%%%%%%%%%%%%%%%%%%%%

\begin{frame}[fragile]
  \frametitle{States, cont.}

Example states in parsing \ling{Book that flight}:
 
\begin{enum}
  
  \item \Se\ $\rightarrow$\; \bigdot\ \VP, [0,0]

    \begin{itemize}
    \item First 0 indicates that the constituent begins at the start of the
      input.
    \item Second 0 indicates that the dot also begins at start of input, and
      thus indicates a top-down prediction.
    \end{itemize} 

  \item \NP\ $\rightarrow$ \Det\ \bigdot\ \Nom, [1,2]

    \begin{itemize}
    \item the \NP\ begins at position 1
    \item the dot is at position 2
    \item \Det\ has been successfully \ling{completed}
    \item \Nom\ is \Em{predicted} next
    \end{itemize}
  \end{enum}
\end{frame}



%%%%%%%%%%%%%%%%%%%%%%%%%%%%%%%%%%%%%%%%%%%%%%%%%%
% Slide 11-a
%%%%%%%%%%%%%%%%%%%%%%%%%%%%%%%%%%%%%%%%%%%%%%%%%%

\begin{frame}[fragile]
  \frametitle{States, cont.}

  \begin{enum}

    \item \VP\ $\rightarrow$ \V\ \NP\ \bigdot, [0,3]

      \begin{itemize}
      \item \VP\ is \Em{completed}
      \item no further \Em{predictions} from this rule
      \item a successful parse that spans the entire input
      \end{itemize}
    \end{enum}

\end{frame}



%%%%%%%%%%%%%%%%%%%%%%%%%%%%%%%%%%%%%%%%%%%%%%%%%%
% Slide 12
%%%%%%%%%%%%%%%%%%%%%%%%%%%%%%%%%%%%%%%%%%%%%%%%%%

%\begin{frame}[fragile]
%  \frametitle{Graphical States}

%%\includegraphics[trim=72 0 72 0]{moore/figures/fig10.15.ps}
%\includegraphics[scale=.85]{moore/figures/fig10.15.ps}

%\end{frame}

%%%%%%%%%%%%%%%%%%%%%%%%%%%%%%%%%%%%%%%%%%%%%%%%%%
% Slide 13
%%%%%%%%%%%%%%%%%%%%%%%%%%%%%%%%%%%%%%%%%%%%%%%%%%

\begin{frame}[fragile]
  \frametitle{Success}


%As with the Viterbi and Forward algorithms, 
The final answer is found
by looking at the last column of the table. In particular, for an input
of $N$ words, if we
find the following kind of state in the chart then we've succeeded: 
\begin{quote}
  
  \Se\ $\rightarrow \alpha$  \bigdot, [0,N]
\end{quote}

\end{frame}

\section{The Earley Algorithm}

%%%%%%%%%%%%%%%%%%%%%%%%%%%%%%%%%%%%%%%%%%%%%%%%%%
% Slide 14
%%%%%%%%%%%%%%%%%%%%%%%%%%%%%%%%%%%%%%%%%%%%%%%%%%

\begin{frame}[fragile]
  \frametitle{Parsing}


Parsing is sweeping through the chart creating the three
kinds of states as we go. States are never removed, and
we never backtrack.

\begin{itemize}
  
\item New \emph{predicted} states are based on existing table entries
  (predicted, or in-progress) that predict a certain constituent at that
  spot.
  
\item New \emph{in-progress} states are created by updating older states
  to reflect the fact that previously expected completed constituents
  have been located.
  
\item New \emph{completed} states are created when the dot in an
  in-progress state moves to the end.
\end{itemize}
\end{frame}


%%%%%%%%%%%%%%%%%%%%%%%%%%%%%%%%%%%%%%%%%%%%%%%%%%
% Slide 15
%%%%%%%%%%%%%%%%%%%%%%%%%%%%%%%%%%%%%%%%%%%%%%%%%%

\begin{frame}[fragile]
  \frametitle{More Specifically}


  \begin{enum}
  \item Predict all the states you can.
    
  \item Read an input.
    \begin{itemize}
    \item See what predictions you can match.
    \item Extend matched states, add new predictions.
    \item Go to next state (goto 2)
  \end{itemize}

  \item At the end, see if \textit{state\/}$[N+1]$ contains a complete \Se.
\end{enum}

\end{frame}

%\end{document}

%%%%%%%%%%%%%%%%%%%%%%%%%%%%%%%%%%%%%%%%%%%%%%%%%%
% Slide 16
%%%%%%%%%%%%%%%%%%%%%%%%%%%%%%%%%%%%%%%%%%%%%%%%%%

\begin{frame}[fragile]
  \frametitle{Earley Algorithm}


The Earley algorithm has three main functions that do all the work:
\begin{description}
\item [Predictor:] Adds predictions into the chart
\item [Completer:] Moves the dot to the right when new constituents are
  found
\item [Scanner:] Reads the input words and enters states representing
  those words into the chart
\end{description}


\end{frame}

%%%%%%%%%%%%%%%%%%%%%%%%%%%%%%%%%%%%%%%%%%%%%%%%%%
% Slide 17
%%%%%%%%%%%%%%%%%%%%%%%%%%%%%%%%%%%%%%%%%%%%%%%%%%

\begin{frame}[fragile]
  \frametitle{Predictor}


\code{\small
  \proc{Predictor}{($A \bnfeq \alpha \bigdot B \beta, [i,j]$)}
       \key{for each} $(B \bnfeq \gamma)$ \key{in} \prog{Grammar-Rules-For}($B${\ac}\var{grammar}) \key{do}
              \prog{Enqueue}(($B \bnfeq \bigdot \gamma, [j,j]$){\ac}\var{chart}[\var{j}])
       \key{end}
}

\begin{itemize}
\item Intuition: new states represent top-down expectations.
\item Applied when a state has a non-terminal to the right of a dot that
  is not a part-of-speech.
\item Generates one new state for each alternative expansion of the
  non-terminal in the grammar.
\item Adds states to the same chart entry as generating state.
\end{itemize}

\end{frame}

%%%%%%%%%%%%%%%%%%%%%%%%%%%%%%%%%%%%%%%%%%%%%%%%%%
% Slide 18
%%%%%%%%%%%%%%%%%%%%%%%%%%%%%%%%%%%%%%%%%%%%%%%%%%

\begin{frame}[fragile]
  \frametitle{Completer}


\code{\small
  \proc{Completer}{($B \,\, \bnfeq \,\, \gamma \,\, \bigdot, [j,k]$)}
       \key{for each} ($A \bnfeq \alpha \bigdot B \beta, [i,j]$) \key{in} \var{chart}[\var{j}] \key{do}
              \prog{Enqueue}(($A \bnfeq \alpha B \bigdot \beta, [i,k]$){\ac}\var{chart[k]})
       \key{end}

}

\begin{itemize}
\item Intuition: parser has discovered a constituent, so must find and
  advance states that were looking for this grammatical category at this
  position in input.
\item Applied when dot has reached right end of rule.
\item New states are generated by copying old state and advancing dot
  over expected category.
\item Adds new states to same chart entry as generating state.
\end{itemize}

\end{frame}

%%%%%%%%%%%%%%%%%%%%%%%%%%%%%%%%%%%%%%%%%%%%%%%%%%
% Slide 19
%%%%%%%%%%%%%%%%%%%%%%%%%%%%%%%%%%%%%%%%%%%%%%%%%%

\begin{frame}[fragile]
  \frametitle{Scanner}


\code{\small
  \proc{Scanner}{($A \bnfeq \alpha \bigdot B \beta, [i,j]$)}
       \key{if} B $\in$ \prog{Parts-of-Speech}(\var{word[j]}) \key{then}
            \prog{Enqueue}(($B \bnfeq word[j] \bigdot, [j,j+1]$){\ac}\var{chart[j+1]})
}

\begin{itemize}
  
\item New states for predicted part-of-speech.
\item Applicable when part-of-speech is to the right of a dot.
\item Adds states to next chart entry.
\end{itemize}

Note:  Earley parser uses top-down predictions to help disambiguate
part-of-speech ambiguities.  Only those parts-of-speech of a word
that are predicted by some state will find their way into the chart.

\end{frame}


%\begin{frame}[fragile]
%\renewcommand{\baselinestretch}{.9}
%\vspace{-2ex}

%  \code{\small
%\func{Earley-Parse}{\var{words{\ac}grammar}}{\var{chart}}
%\bodysep
%   \prog{Enqueue}(($\gamma \bnfeq \chdot S, [0,0]$){\ac}\var{chart[0]})

%   \key{for} \setq{\var{i}}{\key{from} 0 \key{to} \prog{Length}(\var{words})} \key{do}
%     \key{for} \key{each} \var{state} \key{in} \var{chart}[\var{i}] \key{do}
%        \key{if} \prog{Incomplete?}(\var{state}) \key{and} 
%                      \prog{Next-Cat}(\var{state}) is not a part of speech \key{then}
%             \prog{Predictor}(\var{state})
%        \key{elseif} \prog{Incomplete?}(\var{state}) \key{and} 
%                      \prog{Next-Cat}(\var{state}) is a part of speech \key{then}
%              \prog{Scanner}(\var{state})
%        \key{else}
%             \prog{Completer}(\var{state})
%     \key{end}
%   \key{end}
%   \key{return}(\var{chart})

%\subfnsep
%  \proc{Enqueue}{\var{state{\ac}chart-entry}}
%       \key{if} \var{state} is not already in \var{chart-entry} \key{then}
%             Add \var{state} at end of \var{chart-entry}
%       \key{end}
%}




%\end{frame}
%%%%%%%%%%%%%%%%%%%%%%%%%%%%%%%%%%%%%%%%%%%%%%%%%%
% Slide 20
%%%%%%%%%%%%%%%%%%%%%%%%%%%%%%%%%%%%%%%%%%%%%%%%%%

\subsection{Parsing Example}

\renewcommand{\baselinestretch}{1}
\begin{frame}[fragile]
  \frametitle{Mini grammar and lexicon}

\bigskip

\begin{tabular}{|p{15em}||l|}\hline
\Se\ $\rightarrow$ \NP\ \VP\ $\mid$ \Aux\ \NP\ \VP\ $\mid$  \VP  & 
\Det\ $\rightarrow$ \ling{that} $\mid$ \ling{this} $\mid$ \ling{a} \\
\NP\ $\rightarrow$ \Det\ \Nom\  $\mid$ \PropN & 
\N\ $\rightarrow$ \ling{book} $\mid$ \ling{flight} $\mid$ \ling{meal}\\
\Nom\ $\rightarrow$ \Nom\ \PP\  $\mid$ \N\ \Nom & 
\V\ $\rightarrow$ \ling{book} $\mid$ \ling{include} $\mid$ \ling{prefer}\\
\PP\ $\rightarrow$ \Prep\ \NP & 
\Aux\ $\rightarrow$ \ling{does} \\
\VP\ $\rightarrow$ \V\ $\mid$  \V\ \NP & 
\Prep\ $\rightarrow$ \ling{from} $\mid$ \ling{to} $\mid$ \ling{on}\\
\Nom\ $\rightarrow$ \N\ \PP\ $\mid$  \N\ \Nom  & 
\PropN\ $\rightarrow$ \ling{Houston} $\mid$ \ling{TWA}\\
\hline
\end{tabular}



\end{frame}


\end{frame}

%%%%%%%%%%%%%%%%%%%%%%%%%%%%%%%%%%%%%%%%%%%%%%%%%%%
%% Slide 21
%%%%%%%%%%%%%%%%%%%%%%%%%%%%%%%%%%%%%%%%%%%%%%%%%%%

%\begin{frame}[fragile]
%  \frametitle{Example: Chart[0]}


%{
%\input{moore/martin/earley0}}

%\end{frame}

%%%%%%%%%%%%%%%%%%%%%%%%%%%%%%%%%%%%%%%%%%%%%%%%%%
% Slide 22
%%%%%%%%%%%%%%%%%%%%%%%%%%%%%%%%%%%%%%%%%%%%%%%%%%

\begin{frame}[fragile]
  \frametitle{Example: Chart[0] and Chart[1]}


{
\input{earley1}}

\end{frame}

%%%%%%%%%%%%%%%%%%%%%%%%%%%%%%%%%%%%%%%%%%%%%%%%%%
% Slide 23
%%%%%%%%%%%%%%%%%%%%%%%%%%%%%%%%%%%%%%%%%%%%%%%%%%

\begin{frame}[fragile]
  \frametitle{Example: Chart[1] and Chart[2]}



\input{earley2}

\end{frame}

%%%%%%%%%%%%%%%%%%%%%%%%%%%%%%%%%%%%%%%%%%%%%%%%%%
% Slide 24
%%%%%%%%%%%%%%%%%%%%%%%%%%%%%%%%%%%%%%%%%%%%%%%%%%

\begin{frame}[fragile]
  \frametitle{Example: Chart[3]}



\input{earley3}

\end{frame}

%%%%%%%%%%%%%%%%%%%%%%%%%%%%%%%%%%%%%%%%%%%%%%%%%%
% Slide 25
%%%%%%%%%%%%%%%%%%%%%%%%%%%%%%%%%%%%%%%%%%%%%%%%%%

%\begin{frame}[fragile]
%  \frametitle{Examples: Local Ambiguity}


%Let's assume that we're parsing the \VP \ling{gave mary a book} using the 
%following  rules:


%\begin{quote}
%  \Se\ $\rightarrow$ \VP

%  \VP\ $\rightarrow$ \V
  
%  \VP\ $\rightarrow$ \V\ \NP
  
%  \VP\ $\rightarrow$ \V\  \NP  \PP
  
%  \VP\ $\rightarrow$ \V\  \NP  \NP

%\end{quote}
%\end{frame}

%%%%%%%%%%%%%%%%%%%%%%%%%%%%%%%%%%%%%%%%%%%%%%%%%%%
%% Slide 26
%%%%%%%%%%%%%%%%%%%%%%%%%%%%%%%%%%%%%%%%%%%%%%%%%%%

%\begin{frame}[fragile]
%  \frametitle{Examples: Local Ambiguity}






\subsection{Left Recursion}

%\includegraphics[scale=.7]{moore/figures/mary.ps}

%\end{frame}

%%%%%%%%%%%%%%%%%%%%%%%%%%%%%%%%%%%%%%%%%%%%%%%%%%
% Slide 27
%%%%%%%%%%%%%%%%%%%%%%%%%%%%%%%%%%%%%%%%%%%%%%%%%%

\begin{frame}[fragile]
  \frametitle{Examples: Left Recursion}


What about parsing the \NP \ling{a flight from Denver to Boston}
with the following rules:
\begin{quote}
  \NP\ $\rightarrow$ \NP\ \PP

  \NP\ $\rightarrow$ \Det\ \Nom

  \NP\ $\rightarrow$ Proper-Noun

\end{quote}

\begin{itemize}
  \item We construct the state (\NP\ $\rightarrow$ \bigdot\ \NP\ \PP,
    [0,0]) and add it to \textit{chart}[0]
  \item The \textsc{Predictor} function then requires us to find a rule
    which expands the (non-lexical) category immediately to the right of
    the dot. 
  \item So let's pick the first rule above, and \textsc{Enqueue} the
    state (\NP\ $\rightarrow$ \bigdot\ \NP\ \PP, [0,0]).
  \item But this is already in the state, so we don't add it again. 
\end{itemize}


%\includegraphics[scale=.7]{moore/figures/denver.ps}

\end{frame}

%%%%%%%%%%%%%%%%%%%%%%%%%%%%%%%%%%%%%%%%%%%%%%%%%%%
%% Slide 28
%%%%%%%%%%%%%%%%%%%%%%%%%%%%%%%%%%%%%%%%%%%%%%%%%%%

%\begin{frame}[fragile]
%  \frametitle{Examples: Global Ambiguity}


%The key ambiguity arises from the \VP\ $\rightarrow$ \VP\ \PP\ rule and
%the \NP\ $\rightarrow$ \NP  \PP\ rule.






%\includegraphics[scale=.4]{moore/figures/evening.ps}


%\end{frame}

%%%%%%%%%%%%%%%%%%%%%%%%%%%%%%%%%%%%%%%%%%%%%%%%%%
% Slide 29
%%%%%%%%%%%%%%%%%%%%%%%%%%%%%%%%%%%%%%%%%%%%%%%%%%

%\begin{frame}[fragile]
%  \frametitle{Useful Properties}

%  \begin{itemize}
    
%  \item Error handling
    
%  \item Alternative control strategies
%\end{itemize}
%\end{frame}

%%%%%%%%%%%%%%%%%%%%%

%\begin{frame}[fragile]
%  \frametitle{Error handling}


%  \begin{itemize}
    

%  \item What happens when we look at the contents of the last table
%    column and don't find a \Se\ $\rightarrow \alpha \, \bigdot$ rule?
%  \item Is  there some way to proceed?
%  \item Yes. The chart contains every constituent and combination of
%    constituents that could have been found for that input given the
%    grammar.
%  \item Useful for partial (or shallow) parsing, also called
%    \ling{chunking}, which is frequently used in information extraction
%    and question answering.
%\end{itemize}

%\end{frame}

%%%%%%%%%%%%%%%%%%%%%%%%%%%%%%%%%%%%%%%%%%%%%%%%%%
% Slide 32
%%%%%%%%%%%%%%%%%%%%%%%%%%%%%%%%%%%%%%%%%%%%%%%%%%

%\begin{frame}[fragile]
%  \frametitle{Alternative Control Strategies}


%Its trivial to take the Earley top-down strategy and change it to
%bottom-up or \ldots

%\ldots to change it to a best-first (A*) strategy based on the
%probabilities of the constituents. Simply compute and store the
%probabilities of constituents in the chart as you are parsing. Then
%instead of expanding the columns/states in a fixed order you allow the
%probabilities to control the order of expansion.


%\end{frame}

\section{Reading}

%%%%%%%%%%%%%%%%%%%%%%%%%%%%%%%%%%%%%%%%%%%%%%%%%%
% Slide 33
%%%%%%%%%%%%%%%%%%%%%%%%%%%%%%%%%%%%%%%%%%%%%%%%%%

\begin{frame}[fragile]
  \frametitle{Reading}

{\large 
  \begin{itemize}
    
  \item Read section 10.4 of J\&M
 %on finite state parting methods (tutorial).
% Chapter 12 for next time.

    
  \item Read the NLTK-Lite Tutorial on Chart Parsing

%  \item Assignment 2: Sorry, it's late. I'll try to mail out details about it
%    tomorrow.

  \end{itemize}

}

%Start on the new homework!

\end{frame}

%%%%%%%%%%%%%%%%%%%%%%%%%%%%%%%%%%%%%%%%%%%%%%%%%%
% Slide 34
%%%%%%%%%%%%%%%%%%%%%%%%%%%%%%%%%%%%%%%%%%%%%%%%%%
%\begin{frame}[fragile]
%  \frametitle{Acknowledgements}


%Some of this material was taken from the course web pages
%of James Martin. 

%\end{frame}

\end{document}

%%%%%%%%%%%%%%%%%%%%%%%%%%%%%%%%%%%%%%%%%%%%%%%%%%
% Slide 35
%%%%%%%%%%%%%%%%%%%%%%%%%%%%%%%%%%%%%%%%%%%%%%%%%%

\begin{frame}[fragile]
  \frametitle{Memoization and Dynamic Programming}

  
\item As we've seen, dynamic programming techniques make explicit use of
tables to keep track of solutions to previously encountered problems.

\item Correspondingly, dynamic programming algorithms are oriented around
systematically \Em{filling} these tables.

\item Memoization is an alternative technique that achieves the same
results but allows the algorithm


%%% Local Variables: 
%%% mode: latex
%%% TeX-master: "chart1-lec"
%%% End: 
