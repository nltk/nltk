%%%%%%%%%%%%%%%%%%%%%%%%%%%%%%%%%%%%%%%%
% Revised: Time-stamp: <2005-11-14 11:59:46 ewan>
% 
% $Log$
% Revision 1.1  2006/02/08 15:36:53  ehk
% First import of beamer.sty slides. Some files still need renaming.
%
% Revision 1.4  2005/11/14 12:03:33  ewan
% Reworked examples, esp for BIO
%
% Revision 1.3  2005/11/14 10:27:48  ewan
% Reordered some material.
%
% Revision 1.2  2005/11/10 17:00:09  ewan
% *** empty log message ***
%
% Revision 1.1  2005/11/10 16:58:00  ewan
% *** empty log message ***
%
% Revision 1.1  2005/11/08 15:28:01  ewan
% *** empty log message ***
%
% Revision 1.2  2004/10/21 16:38:06  ewan
% *** empty log message ***
%
% Revision 1.1  2004/10/21 11:26:22  ewan
% renamed
%
% Revision 1.1  2004/08/16 16:24:28  ewan
% *** empty log message ***
%
% Revision 1.1  2004/08/09 21:56:09  neilb
% move from teaching/modules/icl to teaching/courses/icl. No corrections done
%
% Revision 1.3  2003/11/25 12:23:35  ewan
% First import into CVS
%
% Revision 1.1  2003/10/14 15:23:23  ewan
% *** empty log message ***
%
% Revision 1.1  2002/11/27 14:28:50  ewan
% First import into CVS
%
%
%
%
%%%%%%%%%%%%%%%%%%%%%%%%%%%%%%%%%%%%%%%%
\title{Chunking}
\author{Ewan Klein \newline \mbox{ }ewan@inf.ed.ac.uk\mbox{ }}
\date{ICL --- 14 November 2005}



\newcommand{\Rule}{\rule{\textwidth}{1pt}}
\newcommand{\SqB}[1]{[#1]}

\newcommand{\bigdot}{\mbox{\begin{scriptsize}{\ensuremath{\bullet}}\end{scriptsize}}}

\newcommand{\chdot}{\bigdot}

%%%%%%%%%%%%%%%%%%%%%%%%%%%%%%%%%%%%%%%%
% Title:  defs
%
% Doc: /home/ewan/Teach/Intro/src/defs.tex
% Original Author:     Ewan Klein (ewan@zangfu)
% Created:             Tue Oct  8 1996
% Revised: Time-stamp: <2005-10-28 16:38:35 ewan>
% $Log$
% Revision 1.1  2006/03/09 09:13:21  ehk
% *** empty log message ***
%
% Revision 1.1  2005/11/03 10:42:49  ewan
% *** empty log message ***
%
% Revision 1.1  2004/09/28 07:28:46  ewan
% *** empty log message ***
%
% Revision 1.2  1996/10/11  16:01:58  ewan
% Installed in /projects/ltg/users/colin/Teach/Intro
%
% Revision 1.2  1996/10/09  03:41:25  ewan
% *** empty log message ***
%
%%%%%%%%%%%%%%%%%%%%%%%%%%%%%%%%%%%%%%%%
%\newcommand{\iupp}{\^\i}
%\newcommand{\nasa}{\diatop[\~|a]}
%\newcommand{\nase}{\diatop[\~|\niepsilon]}
%\newcommand{\naso}{\diatop[\~|\openo]}
%\newcommand{\slt}{\diatop[/|t]}
%\newcommand{\hand}{\ding{43}}
%\newcommand{\hand}{$\Rightarrow$}
\newcommand{\bad}{\sqz{*}}
\newcommand{\cat}[1]{\textit{#1\/}} %linguistic categories
\newcommand{\con}[1]{\textbf{#1\/}} %constituents etc in examples
\newcommand{\dash}{$\underline{\hspace{1em}}$}
\newcommand{\Easy}{\ling{easy}-Constructions}
\newcommand{\Feat}[1]{\textsc{#1}}
\newcommand{\feat}[1]{\textsc{#1}}
\newcommand{\featval}[2]{\textsc{#1}\ \textsc{#2}}
\newcommand{\fun}[1]{\mbox{\textsl{#1}}}
\newcommand{\mygloss}[1]{`{#1}'}
\newcommand{\group}[1]{$\underbrace{\mbox{#1}}$}
\newcommand{\hpsg}{\textsc{hpsg}}
\newcommand{\latin}[1]{\textit{#1\/}} 
%\newcommand{\lextab}{\begin{tabular}{l l @{:\extracolsep{.5in}} l}}
\newcommand{\lextab}{\begin{tabular}{l l l}}
\newcommand{\lingloss}[2]{\ling{#1} \gloss{#2}}
\newcommand{\ling}[1]{\mbox{\textit{#1}}}
\newcommand{\quot}[1]{\textit{#1\/}}
\newcommand{\scare}[1]{`#1'}
\newcommand{\str}[1]{$ #1 $}
\newcommand{\sxref}[2]{(\ref{ex:#1}\ref{ex:#1#2})} %(3a) type example references
\newcommand{\term}[1]{\textsc{#1}}
\newcommand{\type}[1]{\textit{#1\/}}
\newcommand{\val}[1]{\textsc{#1}}
\newcommand{\whIs}{\ling{wh}-Interrogatives}
\newcommand{\whs}{\ling{wh} expressions}
\newcommand{\Whs}{\ling{Wh} expressions}
\newcommand{\xref}[1]{(\ref{ex:#1})} %(3) type example references
%
%\newcommand{\AP}{\textsc{ap\xspace}}
%\newcommand{\NP}{\textsc{np\xspace}}
%\newcommand{\PP}{\textsc{pp\xspace}}
%\newcommand{\VP}{\textsc{vp\xspace}}
%%%%
% changed \textsc to \textsf, 2002-12-02
%%%
\newcommand{\AP}{\textsf{AP}}
\newcommand{\NP}{\textsf{NP}}
\newcommand{\PP}{\textsf{PP}}
\newcommand{\VP}{\textsf{VP}}
\newcommand{\Adj}{\textsf{Adj}}
\newcommand{\Adv}{\textsf{Adv}}
\newcommand{\Aux}{\textsf{Aux}}
\newcommand{\Nom}{\textsf{Nom}}
\newcommand{\Conj}{\textsf{Conj}}
\newcommand{\Pro}{\textsf{Pro}}
\newcommand{\Card}{\textsf{Card}}
\newcommand{\Quant}{\textsf{Quant}}
\newcommand{\Ord}{\textsf{Ord}}
\newcommand{\AdvP}{\textsf{AdvP}}
\newcommand{\A}{\textsf{A}}
\newcommand{\Det}{\textsf{Det}}
\newcommand{\N}{\textsf{N}}
\newcommand{\PropN}{\textsf{PropN}}
\newcommand{\Prep}{\textsf{P}}
\newcommand{\V}{\textsf{V}}
\newcommand{\Se}{\textsf{S}}
\newcommand{\Ntbar}{$\overline{\overline{\overline{\mbox{N}}}}$}
\newcommand{\Ndbar}{$\overline{\overline{\mbox{N}}}$}
\newcommand{\Nbar}{$\overline{\mbox{N}}$}
\newcommand{\Atbar}{$\overline{\overline{\overline{\mbox{A}}}}$}
\newcommand{\Adbar}{$\overline{\overline{\mbox{A}}}$}
\newcommand{\Abar}{$\overline{\mbox{A}}$}
\newcommand{\Ptbar}{$\overline{\overline{\overline{\mbox{P}}}}$}
\newcommand{\Pdbar}{$\overline{\overline{\mbox{P}}}$}
\newcommand{\Pbar}{$\overline{\mbox{P}}$}
\newcommand{\Vtbar}{$\overline{\overline{\overline{\mbox{V}}}}$}
\newcommand{\Vdbar}{$\overline{\overline{\mbox{V}}}$}
\newcommand{\Vbar}{$\overline{\mbox{V}}$}
\newcommand{\Xtbar}{$\overline{\overline{\overline{\mbox{X}}}}$}
\newcommand{\Xdbar}{$\overline{\overline{\mbox{X}}}$}
\newcommand{\Xbar}{$\overline{\mbox{X}}$}
\newcommand{\PS}{\textsc{ps}}
\newcommand{\PSG}{\textsc{psg}}
\newcommand{\Vintrans}{V$_{intrans}$}
\newcommand{\Vtrans}{V$_{trans}$}
\newcommand{\Vsent}{V$_{sent}$}
\newcommand{\XP}{\textsf{XP}}
%
%\newcommand{\W}[1]{\makebox[.75in]{#1}}
%
\newcommand{\cf}{\textsl{c\,f\xspace}}
\newcommand{\ie}{\textsl{i\,e\xspace}}
\newcommand{\eg}{\textsl{e\,g\xspace}}
%
\newcommand{\Book}[1]{\textit{#1\/}}
%
\newenvironment{gram}{%
    \begin{verse}}%
    {\end{verse}}

% Lists
\newenvironment{shortlist}{%
    \begin{itemize}%
        \setlength{\parsep}{0pt}
        \setlength{\itemsep}{0pt}}{%
    \end{itemize}}

\newenvironment{enum}{%
    \begin{enumerate}%
        \setlength{\parsep}{0pt}
        \setlength{\itemsep}{0pt}}{%
    \end{enumerate}}

\newenvironment{desc}{%
    \begin{description}%
        \setlength{\parsep}{0pt}
        \setlength{\itemsep}{0pt}}{%
    \end{description}}


%%%%%%%%%%%%%%%%%%%%%%%%%%%%%%%%%%%%%%%%%%%%%%%%%%%%%%%%%%%%


\usepackage{color}
\usepackage{amsmath}
\usepackage{graphicx}
\usepackage{alltt}


% Shading
\definecolor{light}{gray}{.80}
\newcommand{\Hilite}[1]{\colorbox{yellow}{#1}}
\newcommand{\Shade}[1]{\colorbox{light}{#1}}

\newcommand{\Em}[1]{\textcolor{red}{#1}}
\newcommand{\Dim}[1]{\textcolor{gray}{#1}}

\setlength{\parskip}{0in}
\setlength{\parindent}{0in}



\usepackage{synttree}
\branchheight{5ex}



\begin{document}
\frame{\titlepage}

\mode<article>{\section[Outline]{ICL/Chunking/2005-11-14}}
\mode<presentation>{
  \section[Outline]{}
}

\frame{\tableofcontents}

\section{Motivation}

%%%%%%%%%%%%%%%%%%%%%%%%%%%%%%%%%%%%%%%%%%%%%%%%%%
% Slide 1
%%%%%%%%%%%%%%%%%%%%%%%%%%%%%%%%%%%%%%%%%%%%%%%%%%

% \begin{frame}[fragile]
%   \frametitle{Today}


%   \begin{itemize}
% %  \item Problems with full parsing
%   \item Motivation
%   \item What are chunks?
%   \item Chunking in CASS
%   \item Chunking in NLTK
% \end{itemize}

% \end{frame}


\begin{frame}[fragile]
  \frametitle{Problems with Full Parsing, 1}

Goal: Build a complete parse tree for a sentence.
  \begin{itemize}
    \item Coverage and ambiguity:
      \begin{itemize}
        \item No complete grammar of any language
        \item Sapir: ``All grammars leak''
        \item As coverage increases, so does ambiguity.
        \item Problem of \Em{ranking} parses by degree of `plausibility'
      \end{itemize}
    \item Low accuracy
      \begin{itemize}
        \item Unbounded dependencies hard to parse
        \item Errors tend to propagate
      \end{itemize}
  \end{itemize}

\end{frame}

\begin{frame}[fragile]
  \frametitle{Problems with Full Parsing, 2}

  \begin{itemize}
 
    \item Speed:
      \begin{itemize}
      \item Complexity of rule-based chart parsing is O($n^3$) in length
        of sentence, multiplied by factor O($G^2$), where $G$ is size
        of grammar.
      \item Practical results are often better, but still slow for
        parsing large (e.g., billion words) corpora in reasonable
        time.
      \item Finite state machines have worst-case complexity O($n$) in
        length of string.
      \end{itemize}
  \end{itemize}

\end{frame}


% \begin{frame}[fragile]
%  \frametitle{Full Parsing: Efficiency}

%  \begin{itemize}
%  \item Exponential solution space
%  \item Large relevant processing context
%    \begin{itemize}
%      \item unbounded dependencies
%      \item need to process lexical content of each word
%    \end{itemize}
%  \item Typically too slow to deal with very large corpora (e.g., the web)
%  \end{itemize}

% \end{frame}

\begin{frame}[fragile]
  \frametitle{Motivations for Parsing}

Why parse sentences in the first place?
  \begin{itemize}
    \item Parsing is usually an intermediate stage in a larger
      processing framework.
    \item Full parsing is a \Em{sufficient} but not \Em{necessary}
      step for many NLP tasks.
    \item Full parsing often provides more information than we need or
      can deal with.
  \end{itemize}

\end{frame}

\begin{frame}[fragile]
  \frametitle{Partial Parsing / Chunking}

Assign a \Em{partial structure} to a sentence.
\begin{itemize}
  \item Don't try to deal with all of language
  \item Don't attempt to resolve all semantically significant decisions
  \item Use deterministic grammars for easy-to-parse
    pieces, and other methods for other pieces, depending on task.
  \item ``easy to parse'' = no ambiguity \& no recursion

    
\item Partial parsing is usually:
  \begin{itemize}
  \item easier to implement
  \item more robust
  \item faster
  \end{itemize}
\end{itemize}

 

\end{frame}


\section{What are chunks?}

\begin{frame}[fragile]
  \frametitle{Chunking, 1}

Goal: Divide a sentence into a sequence of \Em{chunks}.

  \begin{itemize}
  \item   Abney (1994):
\begin{verbatim}
[when I read] [a sentence], [I read it] 
[a chunk] [at a time]
\end{verbatim}
     \item Chunks are non-overlapping regions of text:\\
         \verb![walk] [straight past] [the lake]!
%         {} [you] [will take] [the road] that [is] opposite you
        \item (Usually) each chunk contains a \textbf{head}, with the possible
          addition of some \Em{preceding} function words and modifiers\\
\begin{alltt}
[\Hilite{walk}] [straight \Hilite{past}] [the \Hilite{lake}]
\end{alltt}
 
        \item Chunks are non-recursive:
          \begin{itemize}
            \item A chunk cannot contain another chunk of the same category
          \end{itemize}

  \end{itemize}

\end{frame}

\begin{frame}[fragile]
  \frametitle{Chunking, 2}

  \begin{itemize}
        \item<1-> Chunks are non-exhaustive
          \begin{itemize}
          \item Some words in a sentence may not be grouped into a
            chunk\\
            \verb![take] [the second road] that [is] on [the left hand side]!
          \end{itemize}
        \item<2-> \NP\ postmodifiers (e.g., \PP s, relative clauses) are often recursive and/or structurally ambiguous:
          \begin{itemize}
          \item they are \Em{not} included in noun chunks.
          \end{itemize}
        \item<3-> Chunks are typically subsequences of constituents (they
          don't cross constituent boundaries)
        \item<4-> \Em{noun groups} --- everything in \NP\ up to and
          including the head noun
        \item<5-> \Em{verb groups} --- everything in \VP\ (including
          auxiliaries) up to and including the head verb
  \end{itemize}

\end{frame}

% \begin{frame}[fragile]
%   \frametitle{More Chunk Parsing Examples}

%   \begin{itemize}
%      \item Different structures useful for different tasks:
%       \begin{itemize}
%         \item Partial constituent structure\\
%           {}[I] [saw] [two dogs] that [were running] in [the park]
%         \item Prosodic phrases ($\phi$-phrases)\\
%           {}[I s\'aw] [a tall m\'an] [in the p\'ark]
%       \end{itemize}
%     \end{itemize}

% Can also look just for \Em{noun groups} --- everything in \NP\ up to
% and including the head noun; or just for \Em{verb groups} ---
% everything in \VP\ (including auxiliaries) up to and including the head verb.

% \end{frame}

\begin{frame}[fragile]
  \frametitle{Chunk Parsing: Accuracy}

Chunk parsing attempts to do less, but does it more accurately.
  \begin{itemize}
  \item Smaller solution space
  \item Less word-order flexibility \Em{within} chunks than
    \Em{between} chunks.
  \item Better locality:
    \begin{itemize}
      \item doesn't attempt to deal with unbounded dependencies
      \item less context-dependence
      \item doesn't attempt to resolve ambiguity --- only do those
        things which can be done reliably\\
        {}[the boy] [saw] [the man] [with a telescope]
    \end{itemize}
  \item less error propagation
  
    
  \end{itemize}

\end{frame}

\begin{frame}[fragile]
 \frametitle{Chunk Parsing: Domain Independence}

Chunk parsing can be relatively domain independent, in that
 \begin{itemize}
   \item Dependencies involving lexical or semantic information tend to
     occur at levels `higher' than chunks:
     \begin{itemize}
       \item attachment of PPs and other modifiers
       \item argument selection
       \item constituent re-ordering
     \end{itemize}
 \end{itemize}

\end{frame}

\begin{frame}[fragile]
  \frametitle{Chunk Parsing: Efficiency}

Chunk parsing is more efficient:

  \begin{itemize}
    \item smaller solution space
    \item relevant context is small and local
    \item chunks are non-recursive
    \item can be implement with a finite state automaton (FSA)
    \item can be applied to very large text sources
  \end{itemize}

\end{frame}

\begin{frame}[fragile]
 \frametitle{Psycholinguistic Motivations}

 \begin{itemize}
   \item Chunks as processing units --- evidence that humans tend to
     read texts one chunk at a time
   \item Chunks are phonologically relevant
     \begin{itemize}
       \item prosodic phrase breaks
       \item rhythmic patterns
     \end{itemize}
   \item Chunking might be a first step in full parsing
 \end{itemize}

\end{frame}


\section{Chunking in NLTK-Lite}

\begin{frame}[fragile]
  \frametitle{Chunking with Regular Expressions, 1}

  \begin{itemize}
  \item Assume input is tagged.
    \item Identify chunks (e.g., noun groups) by sequences of tags:
\smallskip

      \begin{tabular}[t]{llllllll}
        announce &  any &  new & policy & measures & in & his & \ldots \\
        VB       &  DT  &  JJ  & NN     & NNS      & IN & PRP\$ & \\
      \end{tabular}
  \end{itemize}

\end{frame}

\begin{frame}[fragile]
  \frametitle{Chunking with Regular Expressions, 2}

  \begin{itemize}
  \item<1-> Assume input is tagged.
  \item<1-> Identify chunks (e.g., noun groups) by sequences of tags:
\smallskip

    \begin{tabular}[t]{llllllll}
        announce &  \Hilite{any} &  \Hilite{new} & \Hilite{policy} & \Hilite{measures} & in & his & \ldots \\
        VB       &  DT  &  JJ  & NN     & NNS      & IN & PRP\$ & \\
    \end{tabular}
  \item<2-> Define rules in terms of \Em{tag patterns}
  \item<3-> 
\begin{verbatim}
rule = parse.ChunkRule('<DT><JJ><NN><NNS>', 
                       'Modified plural NPs')
\end{verbatim}


  \end{itemize}

\end{frame}


\begin{frame}[fragile]
  \frametitle{Chunking with Regular Expressions, 3}

  \begin{itemize}
  \item 
\begin{verbatim}
rule = parse.ChunkRule('<DT><JJ><NN><NNS>', 
                       'Modified plural NPs')
\end{verbatim}
  \item<1-> Extending the example:
\smallskip

    \begin{tabular}[t]{llllllll}
        in &  \Hilite{his} &  \Hilite{Mansion} & \Hilite{House} & \Hilite{speech}\\
        IN &  PRP\$        &  NNP             & NNP             & NN \\
    \end{tabular}
  \item<2-> \texttt{DT} or \texttt{PRP\$}: \verb!'<DT|PRP$><JJ><NN><NNS>'!
  \item<3-> \texttt{JJ} and \texttt{NN} are optional: \verb!'<DT|PRP$><JJ>*<NN>*<NNS>'!
  \item<4-> we can have \texttt{NNP}s: \verb!'<DT|PRP$><JJ>*<NNP>*<NN>*<NNS>'!
  \item<5-> \texttt{NN} or \texttt{NNS}: \verb!'<DT|PRP$><JJ>*<NNP>*<NN>*<NN|NNS>'!
  \end{itemize}

\end{frame}


\begin{frame}[fragile]
  \frametitle{Tag Patterns in Chunk Rules}

  \begin{itemize}
        \item NLTK-Lite tag patterns are a special kind of Regular Expression:
          \begin{itemize}
          \item<1-> Use \verb!'< >'! for grouping instead of \verb!'( )'!, e.g.\\
            \verb!'<JJ>*'!, \verb!'<NN|NNS>*'!
          \item<2-> Wildcard \verb!'.'! never matches beyond tag boundaries, e.g.\\
            \verb!'<NN.*>'! matches \verb!'<NN>'! and \verb!'<NNS>'!, but not \verb!'<NN|JJ>'!
          \item<3-> Whitespace is ignored in tag patterns, e.g.\\
             \verb!'<NN | JJ>'! is equivalent to \verb!'<NN|JJ>'!

          \end{itemize}



  \end{itemize}

\end{frame}

% \begin{frame}[fragile]
%   \frametitle{Chunk Parsing in NLTK}

%   \begin{itemize}
%         \item Regular expression matching\\
% \begin{verbatim}
% rule1 = Chunkrule('<DT|NN>+', 
%      'Chunk sequences of NN and DT')

% RegexpChunkparser([rule], 
%                    SUBTOKENS='WORDS', 
%                    chunk_node='NP', 
%                    top_node='S')
% \end{verbatim}

%   \end{itemize}

% \end{frame}

\section{Chunking in Cass}

\begin{frame}[fragile]
  \frametitle{Chunk Grammars}

Approach adopted in Cass (Abney)
\begin{itemize}
  \item Recognition carried out by a cascade of FSAs --- output of one is
    the input to another
    \begin{description}
      \item [Level 0:] tagged words
      \item [Level 1:] all sequences at level 0 that match a given
        pattern are replaced by appropriate label
        \begin{itemize}
          \item e.g., date expressions replaced by the label \texttt{Date}
        \end{itemize}
      \item [Level $n$:] do something with output of Level $n-1$
    \end{description}
  \item Strings that don't match a pattern are just passed on unchanged
\end{itemize}

\end{frame}

\begin{frame}[fragile]
  \frametitle{CASS RegEx Grammar}

Automata defined by a `regular expression grammar'
\begin{verbatim}
:chunks
 nx -> DT? NN+
 vx -> VBZ | VBD | BE VBG
:phrases
 vp -> vx nx*
 pp -> IN nx
:clause
 c -> pp* nx pp* vp pp*
\end{verbatim}

\end{frame}

\begin{frame}[fragile]
  \frametitle{CASS Example}

{\small
\begin{semiverbatim}
\uncover<1->{take/VBP the/DT road/NN on/IN the/DT left/NN}

\uncover<2->{[vx take/VBP] [nx the/DT road/NN] on/IN [nx the/DT left/NN]}

\uncover<3->{[vx take/VBP] [nx the/DT road/NN] [pp on/IN [nx the/DT left/NN]]}

\uncover<4->{[c [vx take/VBP] [nx the/DT road/NN] [pp on/IN [nx the/DT left/NN]]}
\end{semiverbatim}
}

\end{frame}



\section{Chunking as Tagging}

\begin{frame}[fragile]
  \frametitle{CONLL Notation for Chunks}

\begin{itemize}

\item<1-> Instead of using bracketing, as in
\verb!announce [any new policy measures] in [his ...!
\item<2-> we \textbf{tag} words according to where they are in a chunk:
\smallskip

      \begin{tabular}[t]{llllllll}
        announce &  any &  new & policy & measures & in & his & \ldots \\
        VB       &  DT  &  JJ  & NN     & NNS      & IN & PRP\$ & \\
        \Hilite{O}        &  \Hilite{B-NP}& \Hilite{I-NP} & \Hilite{I-N}P   & \Hilite{I-NP}     & \Hilite{O}  & \Hilite{B-NP} & \\
      \end{tabular}

where \texttt{B-NP} is `Begin noun chunk', \texttt{I-NP} is `Inside
noun chunk' and \texttt{O} is `Outside any chunk'.

  \item<3-> Known both as BIO and IOB tagging
  \item<4-> Used in CoNNL shared tasks
  \item<5-> Allows \Em{off-the-shelf statistical taggers to be used for chunking} as well as POS tagging
\end{itemize}
\end{frame}





\section{Summary and Reading}

\begin{frame}[fragile]
  \frametitle{Summary}

  \begin{itemize}
  \item<1-> Chunking is less ambitious than full parsing, but more efficient.
  \item<2-> Maybe sufficient for many practical tasks:
      \begin{itemize}
        \item Information Extraction
        \item Question Answering
        \item Extracting subcatgorization frames
        \item Providing features for machine learning, e.g., for
          building Named Entity recognizers.
      \end{itemize}
  \item<3-> Two main approaches:
    \begin{enum}
      \item Regular expressions over tag sequences
      \item Tagging with IOB tags
    \end{enum}
  \item<4-> Cass extends regular expression approach using a \Em{cascade} of
finite state transducers.
    
  \end{itemize}

\end{frame}

\begin{frame}[fragile]
  \frametitle{Reading}

  \begin{itemize}

  \item Jurafsky and Martin, Section 10.5
  \item NLTK-Lite Chunk Parsing Tutorial
  \item Steven Abney. Parsing By Chunks. In: Robert Berwick, Steven
    Abney and Carol Tenny (eds.), Principle-Based Parsing. Kluwer
    Academic Publishers, Dordrecht. 1991.
  \item Steven Abney. Partial Parsing via Finite-State Cascades. J.\ of
    Natural Language Engineering, 2(4): 337-344. 1996.
  \item Abney's publications:
    \url{http://www.vinartus.net/spa/publications.html}
   
  \end{itemize}

\end{frame}

\begin{frame}[fragile]
  \frametitle{Extra Tutorial}

  \begin{itemize}

  \item Extra tutorial on writing tag patterns
  \item 5.00pm Tuesday 15th Nov, HCRC Seminar Room, 2 Buccleuch Place
   
  \end{itemize}

\end{frame}
\end{document}



%%% Local Variables: 
%%% mode: latex
%%% TeX-master: "chunk-lec"
%%% End: 
