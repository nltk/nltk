% for slides
\documentclass{beamer}

% for handout
% \documentclass[handout]{beamer}
% \usepackage{pgfpages}
% \pgfpagesuselayout{4 on 1}[a4paper,landscape,scale=0.9]

\mode<handout>
{
  \usetheme{default}
  \usepackage{fullpage}
  \usepackage{pgf}
  \usepackage{hyperref}
  \setjobnamebeamerversion{introduction.beamer}
}

\mode<article>
{
  \usepackage{fullpage}
  \usepackage{pgf}
  \usepackage{hyperref}
  \setjobnamebeamerversion{introduction.beamer}
}

\mode<presentation>
{
  \usetheme{Warsaw}
  \setbeamercovered{transparent}
  % If you wish to uncover everything in a step-wise fashion, uncomment
  % the following command: 
  \beamerdefaultoverlayspecification{<+->}

}


\usepackage[english]{babel}
\usepackage[latin1]{inputenc}
\usepackage{times}
\usepackage[T1]{fontenc}
\usepackage{alltt}

\title{NLTK: \\ Introduction to Natural Language Processing}

\author{Steven Bird \and Edward Loper \and Ewan Klein}
\institute{
  University of Melbourne, AUSTRALIA
  \and
  University of Pennsylvania, USA
  \and
  University of Edinburgh, UK
}

\date{\today}

\subject{Natural Language Toolkit}

%%%%%%%%%%%%%%%%%%%%%%%%%%%%%%%%%%%%%%%%%%%%%%%%%%%%%%%%%%%%%%%%%%%%%%%%%%%%%%%%%%
%%%%%%%%%%%%%%%%%%%%%%%%%%%%%%%%%%%%%%%%%%%%%%%%%%%%%%%%%%%%%%%%%%%%%%%%%%%%%%%%%%

\begin{document}

\section{Introduction}

\begin{frame}
  \titlepage
\end{frame}

%%%%%%%%%%%%%%%%%%%%%%%%%%%%%%%%%%%%%%%%%%%%%%%%%%%%%%%%%%%%%%%%%%%%%%%%%%%%%%%%%%

\section{Why Language Processing is Easy}

\begin{frame}
  \frametitle{Questions}
  \begin{itemize}
    \item How do we write programs to manipulate natural language?
    \item What questions about language could we answer?
    \item How would the programs work?
    \item What data would they need?
    \item First: what do they look like?
  \end{itemize}
\end{frame}

\subsection{Searching Pronunciation Dictionary}

\begin{frame}[fragile]
  \frametitle{Searching Pronunciation Dictionary}
  \tiny

\begin{verbatim}
    >>> from nltk_lite.corpora import cmudict
    >>> from string import join
    >>> for word, num, pron in cmudict.raw():
    ...     stress_pattern = join(c for c in join(pron) if c in "012")
    ...     if stress_pattern.endswith("1 0 0 0 0"):
    ...         print word, "/", join(pron)
    ACCUMULATIVELY / AH0 K Y UW1 M Y AH0 L AH0 T IH0 V L IY0
    AGONIZINGLY / AE1 G AH0 N AY0 Z IH0 NG L IY0
    CARICATURIST / K EH1 R AH0 K AH0 CH ER0 AH0 S T
    CIARAMITARO / CH ER1 AA0 M IY0 T AA0 R OW0
    CUMULATIVELY / K Y UW1 M Y AH0 L AH0 T IH0 V L IY0
    DEBENEDICTIS / D EH1 B EH0 N AH0 D IH0 K T AH0 S
    DELEONARDIS / D EH1 L IY0 AH0 N AA0 R D AH0 S
    FORMALIZATION / F AO1 R M AH0 L AH0 Z EY0 SH AH0 N
    GIANNATTASIO / JH AA1 N AA0 T AA0 S IY0 OW0
    HYPERSENSITIVITY / HH AY2 P ER0 S EH1 N S AH0 T IH0 V AH0 T IY0
    IMAGINATIVELY / IH2 M AE1 JH AH0 N AH0 T IH0 V L IY0
    INSTITUTIONALIZES / IH2 N S T AH0 T UW1 SH AH0 N AH0 L AY0 Z AH0 Z
    INSTITUTIONALIZING / IH2 N S T AH0 T UW1 SH AH0 N AH0 L AY0 Z IH0 NG
    MANGIARACINA / M AA1 N JH ER0 AA0 CH IY0 N AH0
    SPIRITUALIST / S P IH1 R IH0 CH AH0 W AH0 L AH0 S T
    SPIRITUALISTS / S P IH1 R IH0 CH AH0 W AH0 L AH0 S T S
    SPIRITUALISTS / S P IH1 R IH0 CH AH0 W AH0 L AH0 S S
    SPIRITUALISTS / S P IH1 R IH0 CH AH0 W AH0 L AH0 S
    SPIRITUALLY / S P IH1 R IH0 CH AH0 W AH0 L IY0
    UNALIENABLE / AH0 N EY1 L IY0 EH0 N AH0 B AH0 L
    UNDERKOFFLER / AH1 N D ER0 K AH0 F AH0 L ER0
\end{verbatim}
\end{frame}

\subsection{Minimal Sets from Lexicon}
\begin{frame}[fragile]
  \frametitle{Minimal Sets from Lexicon}
  \tiny
\begin{verbatim}
    >>> from nltk_lite.corpora import shoebox
    >>> from nltk_lite.utilities import MinimalSet
    >>> length, position, min = 4, 1, 3
    >>> lexemes = [field[1].lower() for entry in shoebox.raw('rotokas')
    ...		   for field in entry if field[0] == 'lx']
    >>> ms = MinimalSet()
    >>> for lex in lexemes:
    ...     if len(lex) == length:
    ...         context = lex[:position] + '_' + lex[position+1:]
    ...         target = lex[position]
    ...         ms.add(context, target, lex)
    >>> for context in ms.contexts(3):
    ...     for target in ms.targets():
    ...         print "%-4s" % ms.display(context, target, "-"),
    ...     print
    kasi -    kesi kusi kosi
    kava -    -    kuva kova
    karu kiru keru kuru koru
    kapu kipu -    -    kopu
    karo kiro -    -    koro
    kari kiri keri kuri kori
    kapa -    kepa -    kopa
    kara kira kera -    kora
    kaku -    -    kuku koku
    kaki kiki -    -    koki
\end{verbatim}
\end{frame}


\subsection{Modelling Text Genres}
\begin{frame}[fragile]
  \frametitle{Modelling Text Genres}
  \tiny
\begin{verbatim}
    >>> from nltk_lite.corpora import genesis
    >>> from nltk_lite.probability import ConditionalFreqDist
    >>> from nltk_lite.utilities import print_string
    >>> cfdist = ConditionalFreqDist()
    >>> prev = None
    >>> for word in genesis.raw():
    ...     word = word.lower()
    ...     cfdist[prev].inc(word)
    ...     prev = word
    >>> words = []
    >>> prev = 'lo,'
    >>> for i in range(99):
    ...     words.append(prev)
    ...     for word in cfdist[prev].sorted_samples():
    ...         if word not in words:
    ...             break
    ...     prev = word
    >>> print_string(join(words))
    lo, it came to the land of his father and he said, i will not be a
    wife unto him, saying, if thou shalt take our money in their kind,
    cattle, in thy seed after these are my son from off any more than all
    that is this day with him into egypt, he, hath taken away unawares to
    pass, when she bare jacob said one night, because they were born two
    hundred years old, as for an altar there, he had made me out at her
    pitcher upon every living creature after thee shall come near her:
    yea,
\end{verbatim}
\end{frame}

\subsection{Exploring Syntax}
\begin{frame}[fragile]
  \frametitle{Exploring Syntax}
  \tiny
\begin{verbatim}
    >>> from nltk_lite.corpora import treebank
    >>> from string import join
    >>> def vp_conj(tree):
    ...     if tree.node == 'VP' and len(tree) == 3 and tree[1].leaves() == ['but']:
    ...         return True
    ...     else:
    ...         return False
    >>> for tree in treebank.parsed():
    ...     for vp1,conj,vp2 in tree.subtrees(vp_conj):
    ...         print join(child.node for child in vp1), "*BUT*", join(child.node for child in vp2)
    VBP ADVP-TMP PP-PRD PP *BUT* VBP VP
    VBZ VP *BUT* VBZ NP PP-CLR
    PP-TMP VBZ VP *BUT* VBD ADVP-TMP S
    VBZ SBAR *BUT* VBZ SBAR
    VBD SBAR *BUT* VBD RB VP
    VBD SBAR *BUT* VBD S
    VBP NP-PRD *BUT* VBP RB ADVP-TMP VP
    VBN PP PP-TMP *BUT* ADVP-TMP VBN NP
    MD VP *BUT* VBZ NP SBAR-ADV
    VBD ADVP-CLR *BUT* VBD NP
    VBN NP PP *BUT* VBN NP PP SBAR-PRP
    VBD NP *BUT* MD RB VP
    VBD NP PP-CLR *BUT* VBD PRT NP
    VBZ S *BUT* MD VP
\end{verbatim}
\end{frame}

\section{The Language Challenge}

\subsection{The Richness of Language}

\begin{frame}
\frametitle{The Richness of Language}
\begin{itemize}
\item basic needs and lofty aspirations; technical know-how and
  flights of fantasy
\item ideas are shared over great separations of distance and time
\end{itemize}

\scriptsize

\begin{enumerate}
\item Overhead the day drives level and grey, hiding the sun by a flight
   of grey spears.  (William Faulkner, \textit{As I Lay Dying}, 1935)
\item When using the toaster please ensure that the exhaust fan is turned
   on. (sign in dormitory kitchen)
\item Amiodarone weakly inhibited CYP2C9, CYP2D6, and CYP3A4-mediated
   activities with Ki values of 45.1-271.6 $\mu$M (Medline)
\item Iraqi Head Seeks Arms (spoof headline, \url{http://www.snopes.com/humor/nonsense/head97.htm}
\item The earnest prayer of a righteous man has great power and wonderful
   results. (James 5:16b)
\item Twas brillig, and the slithy toves did gyre and gimble in the wabe
   (Lewis Carroll, \textit{Jabberwocky}, 1872)
\item There are two ways to do this, AFAIK :smile:  (internet discussion archive)
\end{enumerate}
\end{frame}

\begin{frame}[fragile]
\frametitle{Disciplines Studying Language}
\begin{enumerate}
\item linguistics
\item translation
\item literary criticism
\item philosophy
\item anthropology
\item psychology
\item law
\item hermeneutics
\item forensics
\item telephony
\item pedagogy
\item archaeology
\item cryptanalysis
\item speech pathology
\end{enumerate}
\end{frame}

\subsection{Language in Computing}
\begin{frame}
\frametitle{Language in Computing}
\begin{itemize}
\item new waves of computing technology: new challenges for language
  analysis
\item early machine languages
\item high-level programming languages, automatically parsed and
  interpreted
\item databases: e.g. \texttt{SELECT age FROM employee}
\item ubiquitous computing
\item multimodality: text, speech, dialogue, pen gestures
\item building new systems for natural linguistic interaction
  requires sophisticated language analysis
\end{itemize}
\end{frame}

\subsection{Language on the Web}
\begin{frame}
\frametitle{Language on the Web}
\begin{itemize}
\item explosion of text and multimedia content
\item large and growing fraction of work and leisure time
\item questions:
  \begin{enumerate}
  \item \textit{What tourist sites can I visit between Philadelphia and Pittsburgh on a
    limited budget?}
  \item \textit{What do expert critics say about Canon digital
      cameras?}
  \item \textit{What predictions about the steel market were made by
      credible commentators in the past week?}
  \end{enumerate}
\item requires a combination of language processing tasks, e.g.
  information extraction, inference, and summarisation
\item scale: high-performance computing
\end{itemize}
\end{frame}

\begin{frame}
\frametitle{Web as Corpus: Absolutely vs Definitely}
\scriptsize

\begin{tabular}{l|llll}
\textbf{Google hits}
& \textbf{adore}
& \textbf{love}
& \textbf{like}
& \textbf{prefer} \\ \hline
absolutely  & 289,000 & 905,000 & 16,200  & 644 \\
definitely  & 1,460   & 51,000  & 158,000 & 62,600 \\
ratio       & 198/1   & 18/1    & 1/10    & 1/97
\end{tabular}

\url{http://itre.cis.upenn.edu/~myl/languagelog/archives/002022.html}
\end{frame}

\subsection{Natural Language Processing}

\begin{frame}
\frametitle{Natural Language Processing}
\small
\begin{itemize}
\item experiencing rapid growth as theories and methods deployed in
new technologies
\item people from many fields should have a working knowledge of NLP
\item academia: from humanities computing and corpus linguistics through to computer science and
artificial intelligence
\item industry: human-computer interaction, business information analysis, and web
software development
\item NLTK contains a carefully-balanced selection of theoretical foundations and
    practical application, equips people to:
  \begin{enumerate}
  \item work with large datasets
  \item create robust models of linguistic phenomena
  \item deploy them in working language technologies
  \end{enumerate}
\item importance of practical experience in language processing
\item download NLTK from \texttt{nltk.sourceforge.net} and install
\end{itemize}
\end{frame}

\section{The Python Programming Language}

\subsection{Key Features}

\begin{frame}
\frametitle{Python: Key Features}
\begin{itemize}
\item simple yet powerful, shallow learning curve
\item object-oriented: encapsulation, re-use
\item scripting language, facilitates interactive exploration
\item excellent functionality for processing linguistic data
\item extensive standard library, incl graphics, web, numerical processing
\item downloaded for free from \url{http://www.python.org/}
\end{itemize}
\end{frame}

\subsection{Python Example}
\begin{frame}[fragile]
\frametitle{Python Example}

\begin{verbatim}
 import sys
 for line in sys.stdin.readlines():
     for word in line.split():
         if word.endswith('ing'):
             print word
\end{verbatim}

\begin{enumerate}
\item whitespace: nesting lines of code; scope
\item object-oriented: attributes, methods (e.g. \texttt{line})
\item readable
\end{enumerate}
\end{frame}

\subsection{Comparison with Perl}
\begin{frame}[fragile]
\frametitle{Comparison with Perl}

\begin{verbatim}
  while (<>) {
      foreach my $word (split) {
          if ($word =~ /ing$/) {
              print "$word\n";
          }
      }
  }
\end{verbatim}

\begin{enumerate}
\item syntax is obscure: \textit{what are:} \verb|<> $ my split| ?
\item ``it is quite easy in Perl to write programs that simply
  look like raving gibberish, even to experienced Perl programmers''
  (Hammond \textit{Perl Programming for Linguists} 2003:47)
\item large programs difficult to maintain, reuse
\end{enumerate}
\end{frame}

\subsection{What NLTK adds to Python}
\begin{frame}
\frametitle{What NLTK adds to Python}

NLTK defines a basic infrastructure that can be used to build NLP
programs in Python.  It provides:

\begin{itemize}
\item Basic classes for representing data relevant to natural language
  processing
\item Standard interfaces for performing tasks, such
  as tokenization, tagging, and parsing
\item Standard implementations for each task, which
  can be combined to solve complex problems
\item Extensive documentation, including tutorials
  and reference documentation
\end{itemize}
\end{frame}

\begin{frame}[fragile]
\frametitle{Installing Python and NLTK}
\begin{enumerate}
\item Install Python, Numeric
\item Install NLTK-Lite, NLTK-Lite-Corpora
\item Set environment variable \verb|NLTK_LITE_CORPORA|
\end{enumerate}

\textit{For detailed instructions, see:}

\begin{itemize}
\item  http://nltk.sourceforge.net/install.html
\item  CDROM: /webpage/install.html
\end{itemize}

\end{frame}

\end{document}
