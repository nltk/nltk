\documentclass{beamer}             % for slides
% \documentclass[handout]{beamer}    % for handout
\mode<handout>
{
  \usetheme{default}
  \usepackage{fullpage}
  \usepackage{pgf}
  \usepackage{hyperref}
  \usepackage{pgfpages}
  \pgfpagesuselayout{4 on 1}[a4paper,landscape,scale=0.9]
  \setjobnamebeamerversion{handout.beamer}
}

\mode<article>
{
  \usepackage{fullpage}
  \usepackage{pgf}
  \usepackage{hyperref}
  \setjobnamebeamerversion{notes.beamer}
}

\mode<presentation>
{
  \usetheme{Warsaw}
  \setbeamercovered{transparent}
  % If you wish to uncover everything in a step-wise fashion, uncomment
  % the following command: 
  \beamerdefaultoverlayspecification{<+->}

}


\usepackage[english]{babel}
\usepackage[latin1]{inputenc}
\usepackage{times}
\usepackage[T1]{fontenc}

\date{\today}

\subject{Natural Language Toolkit}



\title{Chunk Parsing}

\author{Steven Bird \and Edward Loper \and Ewan Klein}
\institute{
  University of Melbourne, AUSTRALIA
  \and
  University of Pennsylvania, USA
  \and
  University of Edinburgh, UK
}

%%%%%%%%%%%%%%%%%%%%%%%%%%%%%%%%%%%%%%%%%%%%%%%%%%%%%%%%%%%%%%%%%%%%%%%%%%%%%%%%%%
%%%%%%%%%%%%%%%%%%%%%%%%%%%%%%%%%%%%%%%%%%%%%%%%%%%%%%%%%%%%%%%%%%%%%%%%%%%%%%%%%%

\begin{document}

\frame{\titlepage}

\section{Introduction}

\subsection{What is it?}

\begin{frame}
\begin{itemize}
\item chunk parsing:
  \begin{itemize}
    \item efficient and robust approach to parsing natural language
    \item a popular alternative to the full parsing
  \end{itemize}
\item chunks:
  \begin{itemize}
  \item non-overlapping regions of text
  \item contain a head word (e.g. noun)
  \item adjacent modifiers and function words
  \end{itemize}
\item motivations:
  \begin{itemize}
  \item extract information
  \item ignore information
  \end{itemize}
\end{itemize}
\end{frame}
\subsection{Motivation}

<\pgfdeclareimage[width=12cm]{chunk-coref}{../images/chunk-coref}
\begin{frame}
  \frametitle{Extracting Information: Coreference Annotation}
  \centerline{\pgfuseimage{chunk-coref}}
\end{frame}

\pgfdeclareimage[width=12cm]{chunk-muc}{../images/chunk-muc}
\begin{frame}
  \frametitle{Extracting Information: Message Understanding}
  \centerline{\pgfuseimage{chunk-muc}}
\end{frame}

\begin{frame}[fragile]
  \frametitle{Ignoring Information: Lexical Acquisition}

  \begin{itemize}
  \item studying syntactic patterns, e.g. finding verbs in a corpus, displaying possible arguments
  \item e.g. \texttt{gave}, in 100 files of the Penn Treebank corpus
  \item replaced internal details of each noun phrase with \texttt{NP}

\begin{verbatim}
  gave NP
  gave up NP in NP
  gave NP up
  gave NP help
  gave NP to NP
\end{verbatim}
    
  \item use in lexical acquisition, grammar development
  \end{itemize}
\end{frame}

\subsection{Analogy with Tokenization and Tagging}

\pgfdeclareimage[width=8cm]{chunk-segmentation}{../images/chunk-segmentation}
\begin{frame}[fragile]
  \frametitle{Analogy with Tokenization and Tagging}

  \begin{itemize}
  \item fundamental in NLP: segmentation and labelling
  \item tokenization and tagging \\[2ex]
  \centerline{\pgfuseimage{chunk-segmentation}}
  \item other similarities: skipping material; finite-state; application specific
  \end{itemize}
\end{frame}

\subsection{Chunking vs Parsing}

\begin{frame}[fragile]
  \frametitle{Chunking vs Parsing}
  \scriptsize

  \begin{enumerate}
  \item Parsing
\begin{verbatim}
  [
    [ G.K. Chesterton ],
    [
      [ author ] of
      [
        [ The Man ] who was
        [ Thursday ]
      ]
    ]
  ]
\end{verbatim}

  \item Chunking:
\begin{verbatim}
  [ G.K. Chesterton ],
  [ author ] of
  [ The Man ] who was
  [ Thursday ]
\end{verbatim}
\end{enumerate}
\end{frame}

\begin{frame}
  \frametitle{Chunking vs Parsing}
  \begin{enumerate}
  \item flat vs nested
  \item context
  \item robustness
  \item efficiency
  \item methodology
  \end{enumerate}
\end{frame}

\begin{frame}[fragile]
  \frametitle{Perfection is unattainable}

\begin{verbatim}
  1. Prepositional phrase:
  [
    [ I ]
    [ turned ]
    [ off the spectroroute ]
  ]

  2. Verb-particle construction:
  [
    [ I ]
    [ turned off ]
    [ the spectroroute ]
  ]
\end{verbatim}
\end{frame}


\section{Accessing Chunked Corpora}

\subsection{Representing Chunks: Tags vs Trees}

\pgfdeclareimage[width=8cm]{chunk-tagrep}{../images/chunk-tagrep}
\begin{frame}
  \frametitle{Tag Representation}
  \centerline{\pgfuseimage{chunk-tagrep}}
\end{frame}

\pgfdeclareimage[width=8cm]{chunk-treerep}{../images/chunk-treerep}
\begin{frame}
  \frametitle{Tree Representation}
  \centerline{\pgfuseimage{chunk-treerep}}
\end{frame}

\begin{frame}[fragile]
  \frametitle{Chunk Structures}
\begin{verbatim}
  (S: (NP: 'I')
      'saw'
      (NP: 'the' 'big' 'dog')
      'on'
      (NP: 'the' 'hill'))
\end{verbatim}
  \begin{itemize}
  \item Demonstration: reading chunk structures from Treebank and CoNLL-2000 corpora
  \end{itemize}
\end{frame}

\section{Chunk Parsing}

\begin{frame}
  \frametitle{Chunk Parsing}

To Do

\end{frame}

\subsection{Chunking with Regular Expressions}

\subsection{Developing Chunk Parsers}

\subsection{More Chunking Rules}

\section{Evaluating Chunk Parsers}

\begin{frame}
  \frametitle{Evaluating Chunk Parsers}

To Do

\end{frame}

\end{document}

