% for slides
\documentclass{beamer}

% for handout
% \documentclass[handout]{beamer}
% \usepackage{pgfpages}
% \pgfpagesuselayout{4 on 1}[a4paper,landscape,scale=0.9]

\mode<handout>
{
  \usetheme{default}
  \usepackage{fullpage}
  \usepackage{pgf}
  \usepackage{hyperref}
  \setjobnamebeamerversion{field.beamer}
}

\mode<article>
{
  \usepackage{fullpage}
  \usepackage{pgf}
  \usepackage{hyperref}
  \setjobnamebeamerversion{field.beamer}
}

\mode<presentation>
{
  \usetheme{Warsaw}
  \setbeamercovered{transparent}
  % If you wish to uncover everything in a step-wise fashion, uncomment
  % the following command: 
  \beamerdefaultoverlayspecification{<+->}

}


\usepackage[english]{babel}
\usepackage[latin1]{inputenc}
\usepackage{times}
\usepackage[T1]{fontenc}
\usepackage{alltt}

\title{Accessing and Analyzing Linguistic Field Data}

\author{Steven Bird}
\institute{
  University of Melbourne, AUSTRALIA
}

\date{\today}

\subject{Natural Language Toolkit}

%%%%%%%%%%%%%%%%%%%%%%%%%%%%%%%%%%%%%%%%%%%%%%%%%%%%%%%%%%%%%%%%%%%%%%%%%%%%%%%%%%
%%%%%%%%%%%%%%%%%%%%%%%%%%%%%%%%%%%%%%%%%%%%%%%%%%%%%%%%%%%%%%%%%%%%%%%%%%%%%%%%%%

\begin{document}

\section{Introduction}

\begin{frame}
  \titlepage
\end{frame}

%%%%%%%%%%%%%%%%%%%%%%%%%%%%%%%%%%%%%%%%%%%%%%%%%%%%%%%%%%%%%%%%%%%%%%%%%%%%%%%%%%

\section{Introduction}

\begin{frame}
  \frametitle{Introduction}
  \begin{itemize}
  \item variety of data types, e.g.
  \item \textbf{lexicon:} database of words, minimally containing part
    of speech information and glosses
  \item \textbf{paradigm:} any kind of rational
    tabulation of words or phrases to illustrate contrasts and systematic
    variation
  \item \textbf{text:} any larger unit such as a narrative or a
    conversation
  \item descriptions: field notes, grammars and analytical papers
  \item complex inter-relations (new word; lexicon update, paradigm;
    add to field notes; extend grammar, ...)
  \item tasks: sorting, tabulating, garnering statistics, searching,
  verifying transcriptions
  \item principle challenge is computational
  \end{itemize}

\end{frame}

\subsection{Tools and Technologies}

\begin{frame}
\frametitle{Tools and Technologies}
\begin{itemize}
\item General-purpose tools
  \begin{itemize}
  \item word processors
  \item spreadsheets
  \item databases
  \end{itemize}
\item Special prupose tools
  \begin{itemize}
  \item Shoebox
  \end{itemize}
\item broader question: data management in support of best practices
\item Bird, Steven and Gary Simons (2003).  Seven Dimensions of Portability
    for Language Documentation and Description, \textit{Language} 79: 557-582.
\end{itemize}
\end{frame}

\section{General Purpose Tools}

\subsection{Word Processors}

\begin{frame}
\frametitle{General Purpose Tools: Word Processors}

\begin{itemize}
\item office software widely used in computer-based language
  documentation work
\item familiar, ready availability (though may be too expensive?)
\item Word processors for managing dictionaries:
  \begin{itemize}
  \item consistency of content and format --- time consuming
  \item how would we check the following constraint: \\
    \textit{each entry must have a part-of-speech field, drawn from a set of 20
     possibilities, displayed after the pronunciation field, and rendered
     in 11-point bold}
  \item generally a bad idea!
  \end{itemize}
\item save to a non-proprietary format (e.g. RTF, HTML, or XML)
\item write programs to do this checking automatically
\end{itemize}
\end{frame}

\begin{frame}[fragile]
\frametitle{Example: Microsoft Word}
\begin{itemize}
\item sleep \verb|[sli:p]| \textbf{vi} \textit{condition of body and mind...}
\item We can enter this in MSWord, then "Save as Web Page"
\item resulting HTML:

\begin{verbatim}
    <p class=MsoNormal>sleep <span style='mso-spacerun:yes'> </span>[<span
    class=SpellE>sli:p</span>]<span style='mso-spacerun:yes'>  </span><b><span
    style='font-size:11.0pt'>vi</span></b><span style='mso-spacerun:yes'>  </span><i>a
    condition of body and mind ...o:p></o:p></i></p>
\end{verbatim}

\item Observations:
  \begin{itemize}
  \item entry: HTML paragraph using the \verb|<p>| element
  \item pos: inside a \verb|<span style='font-size:11.0pt'>| element
  \end{itemize}
\end{itemize}
\end{frame}

\begin{frame}[fragile]
\frametitle{Example: Validating Dictionaries stored in MSWord}

\begin{verbatim}
    >>> import re
    >>> legal_pos = set(['n', 'v.t.', 'v.i.', 'adj', 'det'])
    >>> pattern = re.compile(r"'font-size:11.0pt'>([a-z.]+)<")
    >>> document = open("dict.htm").read()
    >>> used_pos = set(re.findall(pattern, document))
    >>> illegal_pos = used_pos.difference(legal_pos)
    >>> print list(illegal_pos)
    ['v.intr', 'v.i', 'intrans']
\end{verbatim}
\end{frame}

\begin{frame}[fragile]
\frametitle{Example: Converting MSWord dictionary to Excel}

\begin{verbatim}
    >>> import re
    >>> document = open("dict.htm").read()
    >>> document = re.sub("[\r\n]", "", document)
    >>> word_pattern = re.compile(r">([\w]+)")
    >>> pron_pattern = re.compile(r"\[.*>([a-z:]+)<.*\]")
    >>> for entry in document.split("<p"):
    ...     word_match = word_pattern.search(entry)
    ...     pron_match = pron_pattern.search(entry)
    ...     if word_match and pron_match:
    ...         lex = word_match.group(1)
    ...         pos = pron_match.group(1)
    ...         print '"%s","%s"' % (lex, pos)
    "sleep","sli:p"
    "walk","wo:k"
    "wake","weik"
\end{verbatim}
\end{frame}

\begin{frame}
\frametitle{Spreadsheets}
\begin{itemize}
\item wordlists, paradigms
\item e.g. comparative wordlist (\url{http://www.rosettaproject.org/}):
  \begin{itemize}
  \item row for each cognate set
  \item column for each language
  \end{itemize}
\item export spreadsheets in \textit{CSV format}
\item read them using Python's \texttt{csv} module
\item e.g. find cognates having an edit-distance of at least three from each other
\end{itemize}
\end{frame}

\begin{frame}
\frametitle{Databases}
\begin{itemize}
  \item lexicons are stored in a full-fledged relational databases
  \item databases can implement many well-formedness constraints\\
    e.g. ensure that all parts-of-speech come from an
    \textit{enumerated type}
  \item However, the relational model is often too restrictive
  \item optional, repeatable fields (e.g.
    dictionary sense definitions and example sentences).
  \item SQL cannot express many linguistically-motivated queries\\
    \textit{Find all words that appear in example sentences for which no dictionary entry is provided}
\end{itemize}
\end{frame}

\begin{frame}[fragile]
\frametitle{Database Example}  

\begin{verbatim}
    >>> import csv
    >>> lexemes = []
    >>> defn_words = []
    >>> for row in csv.reader(open("dict.csv")):
    ...     lexeme, pron, pos, defn = row
    ...     lexemes.append(lexeme)
    ...     defn_words += defn.split()
    >>> undefined = list(set(defn_words).difference(set(lexemes)))
    >>> undefined.sort()
    >>> print undefined
    ['...', 'a', 'and', 'body', 'by', 'cease', 'condition', 'down', 'each', 'foot', 'lifting', 'mind', 'of', 'progress', 'setting', 'to']
\end{verbatim}
\end{frame}

\section{Processing Shoebox Data}

\end{document}

Over the last two decades, several dozen tools have been developed
that provide specialized support for linguistic data management.
(Please see Bird and Simons 2003 for a detailed list of such tools.)
Perhaps the single most popular tool for managing linguistic field
data is *Shoebox*.  Together with its more recent incarnation
(*Toolbox*), Shoebox uses a simple file format which we can easily
read and write, permitting us to apply computational methods to
linguistic field data.  In this section we discuss a variety of
techniques for manipulating Shoebox data in ways that are not supported
by the Shoebox software.

A Shoebox file consists of a collection of *entries* (or *records*),
where each record is made up of one or more *fields*.  Here is an
example of an entry taken from a Shoebox dictionary of Rotokas.
(Rotokas is an East Papuan language spoken on the island of
Bougainville; this data was provided by Stuart Robinson)::

  \lx kaa
  \ps N.M
  \cl isi
  \ge cooking banana
  \gp banana bilong kukim
  \sf FLORA
  \dt 12/Feb/2005
  \ex Taeavi iria kaa isi kovopaueva kaparapasia.
  \xp Taeavi i bin planim gaden banana bilong kukim tasol long paia.
  \xe Taeavi planted banana in order to cook it.

Each field consists of a field name (e.g. ``lx``, for lexeme), and a
value (e.g. ``kaa``).  Other fields are:
``ps`` part-of-speech;
``cl`` classifier;
``ge`` English gloss;
``gp`` Pidgin English gloss;
``sf`` Semantic field;
``dt`` Date last edited;
``ex`` Example sentence;
``xp`` Pidgin translation of example;
``xe`` English translation of example.
These field names are preceded by a backslash, and must always appear
at the start of a line.  The characters of the field names must be
alphabetic.  The field name is separated from the field's contents by
whitespace.  The contents can be arbitrary text, and can continue over
several lines (but cannot contain a line-initial backslash).

Accessing Shoebox Data
----------------------

We can use the ``shoebox.raw()`` method to access a Shoebox file to
perform various operations that involve single-pass scanning of a
lexicon.  For example, here we compute the average number of fields
for each entry:

    >>> from nltk_lite.corpora import shoebox
    >>> sum_size = num_entries = 0
    >>> for entry in shoebox.raw('rotokas'):
    ...     num_entries += 1
    ...     sum_size += len(entry)
    >>> print sum_size/num_entries
    10

As we will see below, we can also use the ``raw()`` method in
functions that add or remove fields, or to examine sequences of fields.

This raw method reads each field into a list, preserving the order of
fields.  Next we look at the ``shoebox.dictionary()`` method, which
reads an entry into a Python dictionary.  The following line generates
a dictionary for each entry, and stores the result in the list ``entries``:

    >>> entries = list(shoebox.dictionary('rotokas'))

We can index into this list, thus ``entries[3]`` returns entry number
3 (which is actually the fourth entry counting from zero).

    >>> from pprint import pprint
    >>> pprint(entries[3])
    {'cl': 'isi',
     'dt': '12/Feb/2005',
     'ex': 'Taeavi iria kaa isi kovopaueva kaparapasia.',
     'ge': 'cooking banana',
     'gp': 'banana bilong kukim',
     'lx': 'kaa',
     'ps': 'N.M',
     'sf': 'FLORA',
     'xe': 'Taeavi planted banana in order to cook it.',
     'xp': 'Taeavi i bin planim gaden banana bilong kukim tasol long paia.'}

Simple Entry Processing
-----------------------

The simplest approach to processing a Shoebox file is to scan through
each entries, and for each entry, to scan through each field.  As we
have seen, the ``shoebox.raw()`` method returns entries, where each
entry is just a sequence of fields.  Suppose we wanted to create a list of
all the lexemes.  We can do this as follows, starting by initialising
``lexemes`` to be the empty list.

    >>> lexemes = []
    >>> for entry in shoebox.raw('rotokas'):
    ...     for field in entry:
    ...         if field[0] == 'lx':
    ...             normalised_lexeme = field[1].lower()
    ...             lexemes.append(normalised_lexeme)

Note that we can construct the ``lexemes`` list much more economically
using Python's *list comprehension* syntax, as follows:

    >>> lexemes = [field[1].lower()
    ...              for entry in shoebox.raw('rotokas')
    ...	               for field in entry if field[0] == 'lx']


Each field is stored as a tuple, e.g. ``('lx', 'kakate')``.  For each
field in each entry, we check to see if the field name is ``lx``.  If
it is, we convert the field's contents to lowercase, and append it to
the ``lexemes`` list.  Observe that this process does not store the
entire lexicon in memory.  Instead, the information we want is
extracted during a single scan of the lexicon.

**Adding New Fields:**
It is often convenient to add new fields that are derived from
existing ones.  Such fields often facilitate analysis.
For example, let us define a function which maps a string of
consonants and vowels to the corresponding CV sequence,
e.g. ``kakapua`` would map to ``CVCVCVV``.

    >>> import re
    >>> def cv(s):
    ...     s = s.lower()
    ...     s = re.sub(r'[^a-z]',     r'_', s)
    ...     s = re.sub(r'[aeiou]',    r'V', s)
    ...     s = re.sub(r'[^V_]',     r'C', s)
    ...     return (s)

This mapping has four steps.  First, the string is converted to lowercase,
then we replace any non-alphabetic characters ``[^a-z]`` with an underscore.
Next, we replace all vowels with ``V``.  Finally, anything that is not
a ``V`` or an underscore must be a consonant, so we replace it with a ``C``.
Now, we can scan the lexicon and add a new ``cv`` field after every ``lx``
field.  Here we will do it for a single entry only:

    >>> raw_entries = list(shoebox.raw('rotokas'))
    >>> for field in raw_entries[50]:
    ...     print "\\%s %s" % field
    ...	    if field[0] == "lx":
    ...	        print "\\cv %s" % cv(field[1])
    \lx kaeviro
    \cv CVVCVCV
    \ps V.A
    \ge lift off
    \ge take off
    \gp go antap
    \nt used to describe action of plane
    \dt 12/Feb/2005
    \ex Pita kaeviroroe kepa kekesia oa vuripierevo kiuvu.
    \xp Pita i go antap na lukim haus win i bagarapim.
    \xe Peter went to look at the house that the wind destroyed.

**Removing Fields:**
We can also use this technique to make copies of Shoebox data
that lack particular fields.  For example, we may want to sanitise our
lexical data before giving it to others, by removing unnecessary
fields (e.g. fields containing personal comments.)

    >>> retain = ('lx', 'ps')
    >>> raw_entries = list(shoebox.raw('rotokas'))
    >>> for entry in raw_entries[50:55]:
    ...     for field in entry:
    ...         if field[0] in retain:
    ...             print "\\%s %s" % field
    ...     print
    \lx kaeviro
    \ps V.A
    <BLANKLINE>
    \lx kagave
    \ps N.F
    <BLANKLINE>
    \lx kaie
    \ps V.A
    <BLANKLINE>
    \lx kaiea
    \ps N.N
    <BLANKLINE>
    \lx kaikaio
    \ps N.N
    <BLANKLINE>

**Formatting Entries:**
We can use the ``shoebox.dictionary()`` method to print a formatted
version of a lexicon.  It allows us to request specific fields without
needing to be concerned with their relative ordering in the original file.

    >>> entries = list(shoebox.dictionary('rotokas'))
    >>> for entry in entries[70:80]:
    ...     lex = entry['lx']
    ...     pos = entry['ps']
    ...     dfn = entry['ge']
    ...     if 'eng' in entry:
    ...         dfn = entry['eng']
    ...     print "%s (%s) '%s'" % (lex, pos, dfn)
    kakapikoto (N.N2) 'newborn baby'
    kakapu (V.B) 'place in sling for purpose of carrying'
    kakapua (N.N) 'sling for lifting'
    kakara (N.N) 'bracelet'
    Kakarapaia (N.PN) 'village name'
    kakarau (N.F) 'stingray'
    Kakarera (N.PN) 'name'
    Kakareraia (N.???) 'name'
    kakata (N.F) 'cockatoo'
    kakate (N.F) 'bamboo tube for water'

.. sidebar:: Producing CSV output

    We could have produced comma-separated value (CSV) format with a
    slightly different print statement: ``print '"%s";"%s";"%s"\n' % (lex, pos, dfn)``

We can use the same idea to generate HTML tables instead of plain text.
This would be useful for publishing a Shoebox lexicon on the web.
It produces HTML elements ``<table>``, ``<tr>`` (table row), and
``<td>`` (table data).

    >>> html = "<table>\n"
    >>> for entry in entries[70:80]:
    ...     lex = entry['lx']
    ...     pos = entry['ps']
    ...     dfn = entry['ge']
    ...     if 'eng' in entry:
    ...         dfn = entry['eng']
    ...     html += "  <tr><td>%s</td><td>%s</td><td>%s</td></tr>\n" % (lex, pos, dfn)
    >>> html += "</table>"
    >>> print html
    <table>
      <tr><td>kakapikoto</td><td>N.N2</td><td>newborn baby</td></tr>
      <tr><td>kakapu</td><td>V.B</td><td>place in sling for purpose of carrying</td></tr>
      <tr><td>kakapua</td><td>N.N</td><td>sling for lifting</td></tr>
      <tr><td>kakara</td><td>N.N</td><td>bracelet</td></tr>
      <tr><td>Kakarapaia</td><td>N.PN</td><td>village name</td></tr>
      <tr><td>kakarau</td><td>N.F</td><td>stingray</td></tr>
      <tr><td>Kakarera</td><td>N.PN</td><td>name</td></tr>
      <tr><td>Kakareraia</td><td>N.???</td><td>name</td></tr>
      <tr><td>kakata</td><td>N.F</td><td>cockatoo</td></tr>
      <tr><td>kakate</td><td>N.F</td><td>bamboo tube for water</td></tr>
    </table>

Exploration
-----------

In this section we consider a variety of analysis tasks.

**Reduplication:**
First, we will develop a program to find reduplicated words.  In order
to do this we need to store all lexemes, along with the English glosses.
We need to keep the glosses so that they can be displayed alongside the
wordforms.  The following code defines a Python dictionary ``lexgloss``
which maps lexemes to their English glosses:

    >>> lexgloss = {}
    >>> for entry in shoebox.dictionary('rotokas'):
    ...     if 'lx' in entry and entry['ps'][0] == 'V':
    ...         lexgloss[entry['lx']] = entry['ge']

Next, for each lexeme ``lex``, we will check if the lexicon contains the
reduplicated form ``lex+lex``.  If it does, we report both forms along with
their glosses.

    >>> for lex in lexgloss:
    ...     if lex+lex in lexgloss:
    ...         print "%s (%s); %s (%s)" % (lex, lexgloss[lex], lex+lex, lexgloss[lex+lex])
    kuvu (fill.up); kuvukuvu (stamp the ground)
    kitu (save); kitukitu (scrub clothes)
    kopa (ingest); kopakopa (gulp.down)
    kasi (burn); kasikasi (angry)
    koi (high pitched sound); koikoi (groan with pain)
    kee (chip); keekee (shattered)
    kauo (jump); kauokauo (jump up and down)
    kea (deceived); keakea (lie)
    kove (drop); kovekove (drip repeatedly)
    kape (unable to meet); kapekape (grip with arms not meeting)
    kapo (fasten.cover.strip); kapokapo (fasten.cover.strips)
    koa (skin); koakoa (remove the skin)
    kipu (paint); kipukipu (rub.on)
    koe (spoon out a solid); koekoe (spoon out)
    kovo (work); kovokovo (surround)
    kiru (have sore near mouth); kirukiru (crisp)
    kotu (bite); kotukotu (grind teeth together)
    kavo (collect); kavokavo (work black magic)
    kuri (scrape); kurikuri (scratch repeatedly)
    karu (unhook); karukaru (open)
    kare (return); karekare (return)
    kari (break); karikari (shred)
    kiro (write); kirokiro (write)
    kae (carry); kaekae (tempt)
    koru (make return); korukoru (obstruct)
    ku (finished with); kuku (spoonfeed)
    kosi (exit); kosikosi (exit)

**Complex Search Criteria:**
Phonological description typically identifies the segments, alternations,
syllable canon and so forth.  It is relatively
straightforward to count up the occurrences of all the different types
of CV syllables that occur in lexemes.

In the following example, we first import the regular expression and
probability modules.  Then we iterate over the lexemes to find all
sequences of a non-vowel ``[^aeiou]`` followed by a vowel ``[aeiou]``.

    >>> from nltk_lite.tokenize import regexp
    >>> from nltk_lite.probability import FreqDist
    >>> fd = FreqDist()
    >>> for lex in lexemes:
    ...     for syl in regexp(lex, pattern=r'[^aeiou][aeiou]'):
    ...         fd.inc(syl)

Now, rather than just printing the syllables and their frequency
counts, we can tabulate them to generate a useful display.

    >>> for vowel in 'aeiou':
    ...     for cons in 'ptkvsr':
    ...          print '%s%s:%4d ' % (cons, vowel, fd.count(cons+vowel)),
    ...     print
    pa:  84  ta:  43  ka: 414  va:  87  sa:   0  ra: 185 
    pe:  32  te:   8  ke: 139  ve:  25  se:   1  re:  62 
    pi:  97  ti:   0  ki:  88  vi:  96  si:  95  ri:  83 
    po:  31  to: 140  ko: 403  vo:  42  so:   3  ro:  86 
    pu:  49  tu:  35  ku: 169  vu:  44  su:   1  ru:  72 

Consider the ``t`` and ``s`` columns, and observe that ``ti`` is not
attested, while ``si`` is frequent.  This suggests that a phonological
process of palatalisation is operating in the language.  We would then
want to consider the other syllables involving ``s``
(e.g. the single entry having *su*, namely *kasuari* 'cassowary' is a loanword).

**Prosodically-motivated search:**
A phonological description may include an examination of the segmental
and prosodic constraints on well-formed morphemes and lexemes.  For
example, we may want to find trisyllabic verbs ending in a long vowel.
Our program can make use of the fact that syllable onsets are
obligatory and simple (only consist of a single consonant).
First, we will encapsulate the syllabic counting part in a separate
function.  It gets the CV template of the word ``cv(word)`` and counts
the number of consonants it contains:

    >>> def num_syls(word):
    ...     template = cv(word)
    ...     num_cons = template.count('C')
    ...     return num_cons

We also encapsulate the vowel test in a function, as this improves the readability
of the final program.  This function returns the value ``True`` just in case ``char``
is a vowel.

    >>> def is_vowel(char):
    ...     return (char in 'aeiou')

Over time we may create a useful collection of such functions.  We can save them
in a file ``utilities.py``, and then at the start of each program we can simply
import all the functions in one go using ``from utilities import *``.
We take the entry to be a verb if the first letter of its part of speech is a ``V``.
Here, then, is the program to display trisyllabic verbs ending in a long vowel:

    >>> for entry in shoebox.dictionary('rotokas'):
    ...     if 'lx' in entry:
    ...         lex = entry['lx']
    ...         pos = entry['ps']
    ...         if num_syls(lex) == 3 and is_vowel(lex[-1]) and is_vowel(lex[-2]) and pos[0] == 'V':
    ...             dfn = entry['ge']
    ...             print "%s (%s) '%s'" % (lex, pos, dfn)
    kaetupie (V.B) 'tighten'
    kakupie (V.B) 'yodel'
    kapatau (V.B) 'add to'
    kapuapie (V.B) 'wound'
    kapupie (V.B) 'close tight'
    kapuupie (V.B) 'close'
    karepie (V.B) 'return'
    karivai (V.A) 'have an appetite'
    kasipie (V.B) 'care for'
    kaukaupie (V.B) 'intense sunlight'
    kavorou (V.A) 'intercept'
    kavupie (V.B) 'leave.behind'
    kekepie (V.B) 'show'
    keruria (V.A) 'determined'
    ketoopie (V.B) 'make sprout from seed'
    koatapie (V.B) 'accept'
    koetapie (V.B) 'satisfy curiosity'
    kokovae (V.A) 'sing'
    kokovua (V.B) 'shave the hair line'
    kopiipie (V.B) 'kill'
    korupie (V.B) 'take outside'
    kosipie (V.B) 'make exit'
    kovopie (V.B) 'use to make work'
    kukuvai (V.B) 'cover the head from rain or sun'
    kuvaupie (V.B) 'leave alone'
    kuverea (V.A) 'not all right'


**Finding Minimal Sets:**
In order to establish a contrast segments (or lexical properties, for
that matter), we would like to find pairs of words which are identical
except for a single property.  For example, the words pairs *mace* vs
*maze* and *face* vs *faze*, and many others like them, demonstrate the
existence of a phonemic distinction between *s* and *z* in English.
NLTK-Lite provides flexible support for constructing minimal sets, using
the ``MinimalSet()`` class.  This class needs three pieces of information
for each item to be added:
``context``: the material that must be fixed across all members of a minimal set;
``target``: the material that changes across members of a minimal set;
``display``: the material that should be displayed for each item.


=========================  ==================  ===============  ===========
Examples of Minimal Set Parameters
---------------------------------------------------------------------------
Minimal Set                Context             Target           Display
=========================  ==================  ===============  ===========
*bib*, *bid*, *big*        first two letters   third letter     word
*deal (N)*, *deal (V)*     whole word          pos              word (pos)
=========================  ==================  ===============  ===========

We begin by creating a list of parameter values, generated from the
full lexical entries.  In our first example, we will print minimal sets
involving lexemes of length 4, with a target position of 1 (second
segment).  The ``context`` is taken to be the entire word, except for
the target segment.  Thus, if lex is ``kasi``, then context is
``lex[:1]+'_'+lex[2:]``, or ``k_si``.  Note that no parameters are
generated if the lexeme does not consist of exactly four segments.

    >>> lexemes = [entry['lx'] for entry in shoebox.dictionary('rotokas')
    ...              if 'lx' in entry]
    >>> position = 1
    >>> parameters = [(lex[:position] + '_' + lex[position+1:],
    ...                lex[position],
    ...                lex)
    ...               for lex in lexemes if len(lex) == 4]

Now, we define a function that builds creates and populates the
``MinimalSet`` object.  For each ``context, target, display`` triple, it
adds an entry to the minimal set.

    >>> from nltk_lite.utilities import MinimalSet
    >>> def build_min_set(parameters):
    ...     min_set = MinimalSet()
    ...     for context, target, display in parameters:
    ...         min_set.add(context, target, display)
    ...     return min_set

Finally, we print the table of minimal sets.  We specify that each
context was seen at least 3 times.

    >>> ms = build_min_set(parameters)
    >>> for context in ms.contexts(3):
    ...     print context + ':',
    ...     for target in ms.targets():
    ...         print "%-4s" % ms.display(context, target, "-"),
    ...     print
    k_si: kasi -    kesi -    kosi
    k_ru: karu kiru keru kuru koru
    k_pu: kapu kipu -    -    kopu
    k_ro: karo kiro -    -    koro
    k_ri: kari kiri keri kuri kori
    k_pa: kapa -    kepa -    kopa
    k_ra: kara kira kera -    kora
    k_ku: kaku -    -    kuku koku
    k_ki: kaki kiki -    -    koki

Observe in the above example that the context, target, and displayed
material were all based on the lexeme field.  However, the idea of
minimal sets is much more general.  For instance, suppose we wanted to
get a list of wordforms having more than one possible part-of-speech.
Then the target will be part-of-speech field, and the context will be
the lexeme field.  We will also display the English gloss field.

    >>> parameters = [(entry['lx'], entry['ps'][0], "%s (%s)" % (entry['ps'][0], entry['ge']))
    ...               for entry in shoebox.dictionary('rotokas') if 'lx' in entry]
    >>> ms = build_min_set(parameters)
    >>> for context in ms.contexts()[:10]:
    ...     print "%10s:" % context, "; ".join(ms.display_all(context))
      kokovara: N (unripe coconut); V (unripe)
         kapua: N (sore); V (have sores)
          koie: N (pig); V (get pig to eat)
          kovo: C (garden); N (work); V (work)
        kavori: N (lobster); V (collect crayfish or lobster)
        korita: N (cutlet?); V (cut up meat)
          keru: N (bone); V (harden like bone)
      kirokiro: N (bush used for sorcery); V (write)
        kaapie: N (fishhook); V (capture)
           kou: C (heap); V (defecate)

.. Analysing Texts:
   Use the shoebox lexicon to tag a text with part-of-speech information,
   then do POS-sensitive concordance-style searches.


Example Applications: Improving Access to Lexical Resources
-----------------------------------------------------------

A lexicon constructed as part of field-based research is a potential
*language resource* for speakers of a language.  Even when the
language in question has a standard writing system, many speakers
will not be literate in the language.  They may be able to attempt
an approximate spelling for a word, or they may prefer to access the
dictionary via an index which uses the language of wider
communication.  In this section we deal with the first of these.
The second is left to the reader as an exercise.  We will also
generate a wordfinder puzzle which can be used to test knowledge
of lexical items.

Fuzzy Spelling (notes)
~~~~~~~~~~~~~~~~~~~~~~

Confusible sets of segments: if two segments are confusible, map them
to the same integer.

    >>> group = {
    ...     ' ':0,                     # blank (for short words)
    ...     'p':1,  'b':1,  'v':1,     # labials
    ...     't':2,  'd':2,  's':2,     # alveolars
    ...     'l':3,  'r':3,             # sonorant consonants
    ...     'i':4,  'e':4,             # high front vowels
    ...     'u':5,  'o':5,             # high back vowels
    ...     'a':6                      # low vowels
    ... }

Soundex: idea of a signature.  Words with the same signature
considered confusible.  Consider first letter of a word to be so
cognitively salient that people will not get it wrong.

    >>> def soundex(word):
    ...     if len(word) == 0: return word  # sanity check
    ...     word += '    '                  # ensure word long enough
    ...     c0 = word[0].upper()
    ...     c1 = group[word[1]]
    ...     cons = filter(lambda x: x in 'pbvtdslr ', word[2:])
    ...     c2 = group[cons[0]]
    ...     c3 = group[cons[1]]
    ...     return "%s%d%d%d" % (c0, c1, c2, c3)
    >>> print soundex('kalosavi')
    K632
    >>> print soundex('ti')
    T400
  
Now we can build a soundex index of the lexicon:

    >>> soundex_idx = {}
    >>> for lex in lexemes:
    ...     code = soundex(lex)
    ...     if code not in soundex_idx:
    ...         soundex_idx[code] = set()
    ...     soundex_idx[code].add(lex)

We should sort these candidates by proximity with the target word.

    >>> from nltk_lite.utilities import edit_dist
    >>> def fuzzy_spell(target):
    ...     scored_candidates = []
    ...     code = soundex(target)
    ...     for word in soundex_idx[code]:
    ...         dist = edit_dist(word, target)
    ...         scored_candidates.append((dist, word))
    ...     scored_candidates.sort()
    ...     return [w for (d,w) in scored_candidates[:10]]

Finally, we can look up a word to get approximate matches:

    >>> fuzzy_spell('kokopouto')
    ['kokopeoto', 'kokopuoto', 'kokepato', 'koovoto', 'koepato', 'kooupato', 'kopato', 'kopiito', 'kovuto', 'koavaato']
    >>> fuzzy_spell('kogou')
    ['kogo', 'koou', 'kokeu', 'koko', 'kokoa', 'kokoi', 'kokoo', 'koku', 'kooe', 'kooku']

Wordfinder Puzzle
~~~~~~~~~~~~~~~~~

Here we will generate a grid of letters, containing words found in the
dictionary.  First we remove any duplicates and disregard the order in
which the lexemes appeared in the dictionary.  We do this by converting
it to a set, then back to a list.  Then we select the first 200 words,
and then only keep those words having a reasonable length.

    >>> words = list(set(lexemes))
    >>> words = words[:200]
    >>> words = [w for w in words if 3 <= len(w) <= 12]

Now we generate the wordfinder grid, and print it out.

    >>> from nltk_lite.misc.wordfinder import wordfinder
    >>> grid, used = wordfinder(words)
    >>> for i in range(len(grid)):
    ...     for j in range(len(grid[i])):
    ...         print grid[i][j],
    ...     print
    O G H K U U V U V K U O R O V A K U N C
    K Z O T O I S E K S N A I E R E P A K C
    I A R A A K I O Y O V R S K A W J K U Y
    L R N H N K R G V U K G I A U D J K V N
    I I Y E A U N O K O O U K T R K Z A E L
    A V U K O X V K E R V T I A A E R K R K
    A U I U G O K U T X U I K N V V L I E O
    R R K O K N U A J Z T K A K O O S U T R
    I A U A U A S P V F O R O O K I C A O U
    V K R R T U I V A O A U K V V S L P E K
    A I O A I A K R S V K U S A A I X I K O
    P S V I K R O E O A R E R S E T R O J X
    O I I S U A G K R O R E R I T A I Y O A
    R R R A T O O K O I K I W A K E A A R O
    O E A K I K V O P I K H V O K K G I K T
    K K L A K A A R M U G E P A U A V Q A I
    O O O U K N X O G K G A R E A A P O O R
    K V V P U J E T Z P K B E I E T K U R A
    N E O A V A E O R U K B V K S Q A V U E
    C E K K U K I K I R A E K O J I Q K K K

Finally we generate the words which need to be found.

    >>> for i in range(len(used)):
    ...     print "%-12s" % used[i],
    ...     if float(i+1)%5 == 0: print
    KOKOROPAVIRA KOROROVIVIRA KAEREASIVIRA KOTOKOTOARA  KOPUASIVIRA 
    KATAITOAREI  KAITUTUVIRA  KERIKERISI   KOKARAPATO   KOKOVURITO  
    KAUKAUVIRA   KOKOPUVIRA   KAEKAESOTO   KAVOVOVIRA   KOVAKOVARA  
    KAAREKOPIE   KAEPIEVIRA   KAPUUPIEPA   KOKORUUTO    KIKIRAEKO   
    KATAAVIRA    KOVOKOVOA    KARIVAITO    KARUVIRA     KAPOKARI    
    KUROVIRA     KITUKITU     KAKUPUTE     KAEREASI     KUKURIKO    
    KUPEROO      KAKAPUA      KIKISI       KAVORA       KIKIPI      
    KAPUA        KAARE        KOETO        KATAI        KUVA        
    KUSI         KOVO         KOAI       

Generating Reports
------------------

Finally, we take a look at simple methods to generate summary reports,
giving us an overall picture of the quality and organisation of the data.

Discovering Entry Patterns
~~~~~~~~~~~~~~~~~~~~~~~~~~

Print most frequent fields

    >>> fd = FreqDist()
    >>> for entry in shoebox.raw('rotokas'):
    ...     for field in entry:
    ...	        fd.inc(field[0])
    >>> fd.sorted_samples()[:10]
    ['ge', 'ex', 'xe', 'xp', 'gp', 'lx', 'ps', 'dt', 'rt', 'eng']


Discovering patterns of fields:

    >>> fd = FreqDist()
    >>> for entry in shoebox.raw('rotokas'):
    ...     marker_list = [field[0] for field in entry]
    ...     markers = ':'.join(marker_list)
    ...	    fd.inc(markers)
    >>> top_ten = fd.sorted_samples()[:10]
    >>> print '\n'.join(top_ten)
    lx:rt:ps:ge:gp:dt:ex:xp:xe
    lx:ps:ge:gp:dt:ex:xp:xe
    lx:ps:ge:gp:dt:ex:xp:xe:ex:xp:xe
    lx:rt:ps:ge:gp:dt:ex:xp:xe:ex:xp:xe
    lx:ps:ge:gp:nt:dt:ex:xp:xe
    lx:ps:ge:gp:dt
    lx:ps:ge:ge:gp:dt:ex:xp:xe:ex:xp:xe
    lx:rt:ps:ge:ge:gp:dt:ex:xp:xe:ex:xp:xe
    lx:ps:ge:ge:gp:dt:ex:xp:xe
    lx:rt:ps:ge:ge:gp:dt:ex:xp:xe


.. Parsing Entries?
   Some shoebox entries have nested structure.  Thus they correspond to a
   tree over the fields.

Looking at Timestamps
~~~~~~~~~~~~~~~~~~~~~

    >>> fd = FreqDist()
    >>> from string import split
    >>> for entry in shoebox.dictionary('rotokas'):
    ...     if 'dt' in entry:
    ...         (day, month, year) = split(entry['dt'], '/')
    ...         fd.inc((month, year))
    >>> for time in fd.sorted_samples():
    ...     print time[0], '/', time[1], ':', fd.count(time)
    Feb / 2005 : 307
    Dec / 2004 : 151
    Jan / 2005 : 123
    Feb / 2004 : 64
    Sep / 2004 : 49
    May / 2005 : 46
    Mar / 2005 : 37
    Apr / 2005 : 29
    Jul / 2004 : 14
    Nov / 2004 : 5
    Oct / 2004 : 5
    Aug / 2004 : 4
    May / 2003 : 2
    Jan / 2004 : 1
    May / 2004 : 1

To put these in time order, we need to set up a special comparison function.
Otherwise, if we just sort the months, we'll get them in alphabetical order.

    >>> month_index = {
    ...     "Jan" : 1, "Feb" : 2,  "Mar" : 3,  "Apr" : 4,
    ...     "May" : 5, "Jun" : 6,  "Jul" : 7,  "Aug" : 8,
    ...     "Sep" : 9, "Oct" : 10, "Nov" : 11, "Dec" : 12
    ... }
    >>> def time_cmp(a, b):
    ...     a2 = a[1], month_index[a[0]]
    ...     b2 = b[1], month_index[b[0]]
    ...     return cmp(a2, b2)

The comparison function says that we compare two times of the
form ``('Mar', '2004')`` by reversing the order of the month and
year, and converting the month into a number to get ``('2004', '3')``,
then using Python's built-in ``cmp`` function to compare them.

Now we can get the times found in the Shoebox entries, sort them
according to our ``time_cmp`` comparison function, and then print them
in order.  This time we print bars to indicate frequency:

    >>> times = fd.samples()
    >>> times.sort(cmp=time_cmp)
    >>> for time in times:
    ...     print time[0], '/', time[1], ':', '#' * (1 + fd.count(time)/10)
    May / 2003 : #
    Jan / 2004 : #
    Feb / 2004 : #######
    May / 2004 : #
    Jul / 2004 : ##
    Aug / 2004 : #
    Sep / 2004 : #####
    Oct / 2004 : #
    Nov / 2004 : #
    Dec / 2004 : ################
    Jan / 2005 : #############
    Feb / 2005 : ###############################
    Mar / 2005 : ####
    Apr / 2005 : ###
    May / 2005 : #####


-----------------
Language Archives
-----------------

Language technology and the linguistic sciences are confronted with a
vast array of language resources, richly structured, large and
diverse. Multiple communities depend on language resources, including
linguists, engineers, teachers and actual speakers.  Thanks to recent
advances in digital technologies, we now have unprecedented
opportunities to bridge these communities to the language resources
they need. First, inexpensive mass storage technology permits large
resources to be stored in digital form, while the Extensible Markup
Language (XML) and Unicode provide flexible ways to represent
structured data and ensure its long-term survival. Second, digital
publication on the web is the most practical and efficient means of
sharing language resources. Finally, a standard resource description
model and interchange method provided by the Open Language Archives
Community (OLAC) makes it possible to construct a *union catalog* over
multiple repositories and archives (see ``http://www.language-archives.org/``).

Managing Metadata for Language Resources
----------------------------------------

OLAC metadata extends the *Dublin Core* metadata set with descriptors
that are important for language resources.

The container for an OLAC metadata record is the element ``<olac>``.
Here is a valid OLAC metadata record from the Pacific And Regional
Archive for Digital Sources in Endangered Cultures (PARADISEC)::

  <olac:olac xsi:schemaLocation="http://purl.org/dc/elements/1.1/ http://www.language-archives.org/OLAC/1.0/dc.xsd
    http://purl.org/dc/terms/ http://www.language-archives.org/OLAC/1.0/dcterms.xsd
    http://www.language-archives.org/OLAC/1.0/ http://www.language-archives.org/OLAC/1.0/olac.xsd">
    <dc:title>Tiraq Field Tape 019</dc:title>
    <dc:identifier>AB1-019</dc:identifier>
    <dcterms:hasPart>AB1-019-A.mp3</dcterms:hasPart>
    <dcterms:hasPart>AB1-019-A.wav</dcterms:hasPart>
    <dcterms:hasPart>AB1-019-B.mp3</dcterms:hasPart>
    <dcterms:hasPart>AB1-019-B.wav</dcterms:hasPart>
    <dc:contributor xsi:type="olac:role" olac:code="recorder">Brotchie, Amanda</dc:contributor>
    <dc:subject xsi:type="olac:language" olac:code="x-sil-MME"/>
    <dc:language xsi:type="olac:language" olac:code="x-sil-BCY"/>
    <dc:language xsi:type="olac:language" olac:code="x-sil-MME"/>
    <dc:format>Digitised: yes;</dc:format>
    <dc:type>primary_text</dc:type>
    <dcterms:accessRights>standard, as per PDSC Access form</dcterms:accessRights>
    <dc:description>SIDE A<p>1. Elicitation Session - Discussion and
      translation of Lise's and Marie-Claire's Songs and Stories from
      Tape 18 (Tamedal)<p><p>SIDE B<p>1. Elicitation Session: Discussion
      of and translation of Lise's and Marie-Clare's songs and stories
      from Tape 018 (Tamedal)<p>2. Kastom Story 1 - Bislama
      (Alec). Language as given: Tiraq</dc:description>
  </olac:olac>
    

.. note:: The remainder of this section will discuss how to manipulate OLAC metadata.


---------------
Further Reading
---------------

Bird, Steven (1999).  Multidimensional exploration of online linguistic field data
    *Proceedings of the 29th Meeting of the North-East Linguistic
    Society*, pp 33-50.

Bird, Steven and Gary Simons (2003).  Seven Dimensions of Portability
    for Language Documentation and Description, *Language* 79: 557-582.

---------
Exercises
---------

1. Write a program that scans an HTML dictionary file to find entries
   having an illegal part-of-speech field, and reports the *headword* for
   each entry.

#. Obtain a comparative wordlist in CSV format, and write a program
   that prints those cognates having an edit-distance of at least three
   from each other.

#. Write a program to filter out just the date field (``dt``) without
   having to list the fields we wanted to retain.

#. Print an index of a lexicon.  For each lexical entry, construct a
   tuple of the form ``(gloss, lexeme)``, then sort and print them all.

#. Write a program to find any parts of speech (``ps`` field) that
   occurred less than ten times.  Perhaps these are typing mistakes?

#. We saw a method for discovering cases of whole-word reduplication.
   Write a function to find words that may contain partial
   reduplication.  Use the ``re.search()`` method, and the following
   regular expression: ``(..+)\1``

#. What is the frequency of each consonant and vowel contained in
   lexeme fields?

#. We saw a method for adding a ``cv`` field.  There is an interesting
   issue with keeping this up-to-date when someone modifies the content
   of the ``lx`` field on which it is based.  Write a version of this
   program to add a ``cv`` field, replacing any existing ``cv`` field.

#. Build an index of those lexemes which appear in example sentences.
   Suppose the lexeme for a given entry is *w*.
   Then add a single cross-reference field ``xrf`` to this entry, referencing
   the headwords of other entries having example sentences containing
   $w$.  Do this for all entries and save the result as a shoebox-format file.
  
#. Write a program to add a new field ``syl`` which gives a count of
   the number of syllables in the word.

#. Write a function which displays the complete entry for a lexeme.
   When the lexeme is incorrectly spelled it should display the entry
   for the most similarly spelled lexeme.
 
#. How many entries were last modified in 2004?

----

NLTK_

.. _NLTK: http://nltk.sourceforge.net/


.. To add: paradigms
